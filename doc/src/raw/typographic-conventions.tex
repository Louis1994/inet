%%%%%%%%%%%%%%%%%%%%%%%% OMNeT++ Book -- Typographic Conventions %%%%%%%%%%%%%%%%%%%%%%%%
%
% This file contains the typographic conventions of the book 
%
%%%%%%%%%%%%%%%%%%%%%%%%%%%%%%%%%%%%%%%%%%%%%%%%%%%%%%%%%%%%%%%%%%%%%%%%%%%%%%%%%%%%%%%%%
%
\clearpage
\section*{Typographic Conventions}
%
The following list of typographic conventions shall harmonize the formatting of text and the use of markups throughout the book.
It is going to be added to the preface of the complete book and inserted here for convenience and reference for the authors.

{%
\renewcommand{\arraystretch}{2.0}%
\setlength{\tabcolsep}{1.0mm}%
\begin{table}
\begin{tabular}{@{}lp{5.75cm}p{3.7cm}@{}}
%\begin{tabularx}{1.0\columnwidth}{@{}lXp{3.75cm}@{}}
	%
	\toprule
	%
	\textbf{Characteristic}			& \textbf{Definition / Meaning} 	& \textbf{Examples} \\
	%
	\midrule
	%
	\code{TypeWriter}				& A \code{fixed width font} is used in source code listings and for all source-code-related markups (e.g., to mark function or class names, constants, variables, and likewise). 
									& ``The class \code{TrafficGen} is'' \newline%
									  ``Use \code{copy(\&msg, bool)}'' \\
	%
	\cursive{Cursive}				& \cursive{Cursive markups} are used to emphasize all non-source-code-related terms (e.g., scientific terms, special topics, adjectives). 
									& ``The \cursive{service primitive} is'' \\
	%
	\program{Program}				& Words written in \program{boldface} indicate individual computer programs or product names.
									& ``The \program{opp\_run} program'' \\
	%
	\filename{file.xyz}				& File names use a \textsl{slanted italic} font.
									& ``The \filename{omnetpp.ini} file'' \\
	%
	\keys{KeyName}					& Cursive names in squared brackets represent keys on the keyboard.
									& ``Press \keys{Ctrl} or \keys{Alt}'' \\
	%
	\keys{Key1}{Key2}				& Keys combined with a ``+'' sign are meant to be pressed simultaneously.
									& ``Press \keys{Ctrl}{Shift}{W}'' \\
	%
	\commands{Term}{Term} 			& \commands{Cursive terms} in combination with arrows ($\rightarrow$) mark command sequences in user interface elements, dialogs, pop-up, or pull-down menus.
									& ``\commands{Menu}{File}{Save As...}'' \\
	%
	\texttt{\ldots}					& Three dots inside tables and source code listings indicate left out or abbreviated parts. 
									& \\ 
	%
	\bottomrule
	%
\end{tabular}
\end{table}
}%
%%%%%%%
% EOF %
%%%%%%%