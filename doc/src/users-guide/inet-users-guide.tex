\documentclass{book}
\usepackage{a4wide}

%% possible fonts -- in order of preference
%%\usepackage{palatino}
\usepackage{bookman}
%%\usepackage{charter}
%%\usepackage{newcent}
%%\usepackage{times}
%%\usepackage{avant}
%%\usepackage{helvet}
%%\usepackage{sans}
%%\usepackage{chancery}

\usepackage[T1]{fontenc}
\usepackage{setspace}
\usepackage{ifpdf}
\usepackage{makeidx}
\usepackage{longtable}  %% page wrapping table environment
\usepackage{colortbl}   %% colors for tables
\usepackage{fancyvrb}   %% the "Verbatim" environment
\usepackage{fancyhdr}   %% custom headers and footers
\usepackage{multicol}
\usepackage{listings}   %% source code listings with syntax highlight (lstxxx commands)
\usepackage[tight]{shorttoc}   %% for generating a second table of contents, only containing chapter titles
\usepackage{bytefield}  %% for drawing protocol frames
\usepackage{paralist}   %% for compact lists
\usepackage[nottoc]{tocbibind}  %% makes Bibliography and Index show up in TOC
\settocbibname{References}

\setlength{\textwidth}{160mm}
%\setlength{\oddsidemargin}{12.5mm}
%\setlength{\evensidemargin}{12.5mm}
%\setlength{\topmargin}{0mm}
\setlength{\textheight}{220mm}
%\setlength{\parskip}{1ex}
%\setlength{\parindent}{5ex}

\renewcommand{\bottomfraction}{0.9}
\renewcommand{\topfraction}{0.9}
\renewcommand{\floatpagefraction}{0.9}

%% try to cure overfull hboxes
%% \tolerance=500

%% for navigation in dvi files, only needed by old teTeX versions
%%\usepackage{srcltx}

%% try this for spell checking: cat ess2002.tex | ispell -l -t -a | sort | uniq | more

%%
%% The following snippet changes the horizontal spacing between the number and
%% the title in the table of contents.
%%
%% http://tex.stackexchange.com/questions/33841/how-to-modify-the-space-between-the-numbers-and-text-of-sectioning-titles-in-the
%%
\makeatletter
 \renewcommand*\l@section{\@dottedtocline{1}{2em}{3em}}
 \renewcommand*\l@subsection{\@dottedtocline{2}{5em}{4em}}
\renewcommand*\l@chapter[2]{%
  \ifnum \c@tocdepth >\m@ne
    \addpenalty{-\@highpenalty}%
    \vskip 1.0em \@plus\p@
    \setlength\@tempdima{2em}%
    \begingroup
      \parindent \z@ \rightskip \@pnumwidth
      \parfillskip -\@pnumwidth
      \leavevmode \bfseries
      \advance\leftskip\@tempdima
      \hskip -\leftskip
      #1\nobreak\hfil \nobreak\hb@xt@\@pnumwidth{\hss #2}\par
      \penalty\@highpenalty
    \endgroup
  \fi}
\makeatother

%%
%% OMNeT++ logo, use as {\opp}
%%
\makeatletter
%%\DeclareRobustCommand{\omnetpp}{OM\-NeT\kern-.18em++\@}
\DeclareRobustCommand{\omnetpp}{OMNeT++\@}
\makeatother

\newcommand{\opp}{\omnetpp}

%%
%% PDF Header
%%
% note: \ifpdf now comes from the ifpdf package
%\newif\ifpdf
%\ifx\pdfoutput\undefined
%  \pdffalse
%\else
%  \pdfoutput=1
%  \pdftrue
%\fi
%% PDF-Info
\ifpdf
  \usepackage[pdftex]{graphicx}
  \usepackage[plainpages=false,linktocpage,bookmarksnumbered=true,pdftex]{hyperref}   %% automatic hyperlinking
  \pdfcompresslevel=9
  \pdfinfo{/Author (Andras Varga and others)
    /Title (INET Framework Manual)
    /Subject ()
    /Keywords (INET, INETMANET, OMNeT++, manual)}
\else
  \usepackage{graphicx}
  \usepackage[plainpages=false]{hyperref}   %% automatic hyperlinking
\fi

%%
%% Draft conditional to include unfinished parts
%%
\newif\ifdraft
%\draftfalse %% uncomment for final version
\drafttrue %% uncomment for draft version

%%
%% Generate Index
%%
\makeindex


%%
%% Link colors (hyperref package)
%%
\definecolor{MyDarkBlue}{rgb}{0.16,0.16,0.5}
%% XXX the next line apparently screws up all links except in TOC! they'll be colored nicely, but won't work.
%\hypersetup{
%    colorlinks=true,
%    linkcolor=MyDarkBlue,
%    anchorcolor=MyDarkBlue,
%    citecolor=MyDarkBlue,
%    filecolor=MyDarkBlue,
%    menucolor=MyDarkBlue,
%    runcolor=MyDarkBlue,
%    urlcolor=blue,
%}

%%
%% Heading and Footer
%%
\pagestyle{fancy}
\fancyhf{}
\renewcommand{\footrulewidth}{0.5pt}
\renewcommand{\chaptermark}[1]{\markboth{#1}{}}
\lhead{INET Framework Manual -- \leftmark}
\rfoot{\thepage}

%% this is used for chapter start pages
\fancypagestyle{plain}{
    \rfoot{\thepage}
}

%%
%% Use \begin{graybox}...\end{graybox} for notes
%%
\definecolor{MyGray}{rgb}{0.85,0.85,1.0}
\makeatletter\newenvironment{graybox}%
   {\begin{flushright}\begin{lrbox}{\@tempboxa}\begin{minipage}[r]{0.95\textwidth}}%
   {\end{minipage}\end{lrbox}\colorbox{MyGray}{\usebox{\@tempboxa}}\end{flushright}}%
\makeatother


\newenvironment{note}{\begin{graybox}\textbf{NOTE: }}{\end{graybox}}
\newenvironment{warning}{\begin{graybox}\textbf{WARNING: }}{\end{graybox}}
\newenvironment{caution}{\begin{graybox}\textbf{CAUTION: }}{\end{graybox}}
\newenvironment{rationale}{\begin{graybox}\textbf{Rationale: }}{\end{graybox}}
\newenvironment{important}{\begin{graybox}\textbf{IMPORTANT: }}{\end{graybox}}

%%
%% Set up listings package
%%
\lstloadlanguages{C++,make,perl,tcl,XML,R,Matlab}

%% See listings.pdf,pp20
\lstdefinelanguage{NED} {
    morekeywords={allowunconnected,bool,channel,channelinterface,connections,const,
                  default,double,extends,false,for,gates,if,import,index,inout,input,
                  int,like,module,moduleinterface,network,output,package,parameters,
                  property,simple,sizeof,string,submodules,this,true,types,volatile,
                  xml,xmldoc},
    sensitive=true,
    morecomment=[l]{//},
    morestring=[b]",
}
\lstdefinelanguage{MSG} {
    morekeywords={abstract,bool,char,class,cplusplus,double,enum,extends,false,
                  fields,int,long,message,namespace,noncobject,packet,properties,
                  readonly,short,string,struct,true,unsigned},
    sensitive=true,
    morecomment=[l]{//},
    morestring=[b]",
}
\lstdefinelanguage{inifile} {
    morekeywords={},
    sensitive=true,
    morecomment=[l]{\#},
    morestring=[b]",
}
\lstdefinelanguage{pseudocode} {
    morekeywords={if,then,else,otherwise,whenever,while},
    sensitive=true,
    morecomment=[l]{//},
    morestring=[b]",
    mathescape=true,
}

%% thick ruler on the left; also, designate backtick as LaTeX escape character
%% (e.g. \opp needs to be written as `\opp` inside listing blocks)
\lstset{
    escapechar=`,
    basicstyle=\ttfamily,
    showstringspaces=false,
    frame=leftline,
    framesep=10pt,
    framerule=3pt,
    xleftmargin=15pt
}

\definecolor{NEDRulerColor}{rgb}{0.8,1.0,0.8}  % pale green
\definecolor{MSGRulerColor}{rgb}{0.8,1.0,0.8}  % pale green
\definecolor{CPPRulerColor}{rgb}{0.8,0.8,1.0}  % pale blue
\definecolor{IniRulerColor}{rgb}{0.9,0.9,0.2}  % pale yellow
\definecolor{FileListingRulerColor}{rgb}{0.85,0.85,0.85}  % grey
%\definecolor{CommandLineRulerColor}{rgb}{0.9,0.9,0.2}
\definecolor{PseudoCodeRulerColor}{rgb}{0.0,1.0,1.0}  % cyan

%% See listings.pdf,pp39
\lstnewenvironment{ned}
    {\lstset{language=NED,rulecolor=\color{NEDRulerColor}}}
    {}
\lstnewenvironment{msg}
    {\lstset{language=MSG,rulecolor=\color{MSGRulerColor}}}
    {}
\lstnewenvironment{cpp}
    {\lstset{language=C++,rulecolor=\color{CPPRulerColor}}}
    {}
\lstnewenvironment{inifile}
    {\lstset{language=inifile,rulecolor=\color{IniRulerColor}}}
    {}
\lstnewenvironment{filelisting}
    {\lstset{language={},rulecolor=\color{FileListingRulerColor}}}
    {}
\lstnewenvironment{commandline}
    {\lstset{language={},framesep=11pt,framerule=1pt,xleftmargin=16pt}}
    {}
\lstnewenvironment{pseudocode}
    {\lstset{language=pseudocode,rulecolor=\color{PseudoCodeRulerColor}}}
    {}

%%
%% some customization
%%
\setlength{\parindent}{0pt}
\setlength{\parskip}{1ex}

%%
%% Shortcuts
%%
\newcommand{\appendixchapter}{\chapter} %% html converter needs to know which chapters are appendices

\newcommand{\tbf}{\textbf} %% bold faced text
\newcommand{\ttt}{\texttt} %% type writer font text

\newcommand{\tab}{\hspace*{5mm}} %% tabulator settings

\newcommand{\new}{$^{New!}$}
\newcommand{\changed}{$^{Changed!}$}

%% Colordefinition for table header rows (requires package colortbl)
\newcommand{\tabheadcol}{\rowcolor[gray]{0.8}}

%%
%% Module parameters list
%%
\newenvironment{params}{\begin{itemize}}{\end{itemize}}
\newcommand{\param}[2]{\item \fpar{#1}: #2}

%%
%% Function/Class/Macro/Variable/Program/Parameter/Define names
%%
%% Write the names in type writer font and do an index entry
%% Allows word wrap by automatic hyphenation
%%
%% Usage: \ffunc{take()}
%%    or: \ffunc[take()]{take(obj)}
%% the second form uses the bracketed word for the index entry
%%

%% NED type names
\newcommand{\nedtype}[2][\DefaultOpt]{\def\DefaultOpt{#2}%
  \index{#1}%
  \texttt{\hyphenchar\font=`\-\relax#2}}

%% MSG type names
\newcommand{\msgtype}[2][\DefaultOpt]{\def\DefaultOpt{#2}%
  \index{#1}%
  \texttt{\hyphenchar\font=`\-\relax#2}}

%% Function names
\newcommand{\ffunc}[2][\DefaultOpt]{\def\DefaultOpt{#2}%
  \index{#1}%
  \texttt{\hyphenchar\font=`\-\relax#2}}

%% Class names
\newcommand{\cppclass}[2][\DefaultOpt]{\def\DefaultOpt{#2}%
  \index{#1}%
  \texttt{\hyphenchar\font=`\-\relax#2}}

%% Macro names
\newcommand{\fmac}[2][\DefaultOpt]{\def\DefaultOpt{#2}%
  \index{#1}%
  \texttt{\hyphenchar\font=`\-\relax#2}}

%% Variable names
\newcommand{\fvar}[2][\DefaultOpt]{\def\DefaultOpt{#2}%
  \index{#1}%
  \texttt{\hyphenchar\font=`\-\relax#2}}

%% Program names
\newcommand{\fprog}[2][\DefaultOpt]{\def\DefaultOpt{#2}%
  \index{#1}%
  \texttt{\hyphenchar\font=`\-\relax#2}}

%% Parameter names
\newcommand{\fpar}[2][\DefaultOpt]{\def\DefaultOpt{#2}%
  \index{#1}%
  \texttt{\hyphenchar\font=`\-\relax#2}}

%% Defines
\newcommand{\fdef}[2][\DefaultOpt]{\def\DefaultOpt{#2}%
  \index{#1}%
  \texttt{\hyphenchar\font=`\-\relax#2}}

%% NED/MSG properties
\newcommand{\fprop}[2][\DefaultOpt]{\def\DefaultOpt{#2}%
  \index{#1}%
  \texttt{\hyphenchar\font=`\-\relax#2}}

%% Keywords (NED, MSG)
\newcommand{\fkeyword}[2][\DefaultOpt]{\def\DefaultOpt{#2}%
  \index{#1}%
  \textbf{\texttt{\hyphenchar\font=`\-\relax#2}}}

%% Configuration options
\newcommand{\fconfig}[2][\DefaultOpt]{\def\DefaultOpt{#2}%
  \index{#1}%
  \textbf{\texttt{\hyphenchar\font=`\-\relax#2}}}

%% File names
\newcommand{\ffilename}[2][\DefaultOpt]{\def\DefaultOpt{#2}%
  \index{#1}%
  \texttt{\hyphenchar\font=`\-\relax#2}}

%% Signals
\newcommand{\fsignal}[2][\DefaultOpt]{\def\DefaultOpt{#2}%
  \index{#1}%
  \texttt{\hyphenchar\font=`\-\relax#2}}

\newcommand{\fgate}[1]{\texttt{\hyphenchar\font=`\-\relax#1}}

%% do not number subsubsections
%\setcounter{secnumdepth}{4}

% limit the depth of TOC
\setcounter{tocdepth}{2}

%%
%% Start of document
%%
\begin{document}

%% set the image type preference
\DeclareGraphicsExtensions{.pdf,.png}

\pagestyle{empty}
\pagenumbering{roman}
\include{title}
\cleardoublepage

%%\setcounter{page}{1}
%\newpage
%%\pagenumbering{roman}

%% \shorttableofcontents{Chapters}{0}
%% \cleardoublepage

\tableofcontents
\cleardoublepage

\pagestyle{fancy}
\pagenumbering{arabic}

\chapter{Introduction}
\label{cha:introduction}


\section{What is INET Framework}

INET Framework is an open-source model library for the OMNeT++ simulation
environment. It provides protocols, agents and other models for researchers 
and students working with communication networks. INET is especially useful 
when designing and validating new protocols, or exploring new or exotic scenarios.

INET supports a wide class of communication networks, including wired, 
wireless, mobile, ad hoc and sensor networks.  It contains models for 
the Internet stack (TCP, UDP, IPv4, IPv6, OSPF, BGP, etc.), link layer protocols 
(Ethernet, PPP, IEEE 802.11, various sensor MAC protocols, etc), 
refined support for the wireless physical layer, MANET routing protocols, 
DiffServ, MPLS with LDP and RSVP-TE signalling, several application models,
and many other protocols and components. It also provides support for 
node mobility, advanced visualization, network emulation and more. 

Several other simulation frameworks take INET as a base, and extend it 
into specific directions, such as vehicular networks, overlay/peer-to-peer 
networks, or LTE.

\section{Designed for Experimentation}

INET is built around the concept of modules that communicate by message passing.
Agents and network protocols are represented by components, which can be freely
combined to form hosts, routers, switches, and other networking devices. 
New components can be programmed by the user, and existing components have 
been written so that they are easy to understand and modify.

INET benefits from the infrastructure provided by OMNeT++. Beyond making 
use of the services provided by the OMNeT++ simulation kernel and library
(component model, parameterization, result recording, etc.), this also means
that models may be developed, assembled, parameterized, run, and their 
results evaluted from the comfort of the OMNeT++ Simulation IDE, or 
from the command line.

INET Framework is maintained by the OMNeT++ team for the community, 
utilizing patches and new models contributed by members of the community.

\section{Scope of this Manual}

This manual is written for users who are interested in assembling
simulations using the components provided by the INET Framework.
This manual is accompanied by the INET Reference, which is generated from 
NED and MSG files. A working knowledge of OMNeT++ is assumed.


%%% Local Variables:
%%% mode: latex
%%% TeX-master: "usman"
%%% End:


\cleardoublepage

\chapter{Using the INET Framework}
\label{cha:usage}

\section{Installation}

There are several ways to install the INET Framework:

\begin{itemize}
  \item Let the OMNeT++ IDE download and install it for you. 
      This is the easiest way. Just accept the offer to install INET
      in the dialog that comes up when you first start the IDE, or
      choose \textit{Help > Install Simulation Models} any time later.
  \item From INET Framework web site, \textit{http://inet.omnetpp.org}. 
      The IDE always installs the last stable version compatible with
      your version of OMNeT++. If you need some other version, they
      are available for download from the web site. Installation
      instructions are also provided there.  
  \item From GitHub. If you have experience with \textit{git}, 
      clone the INET Framework project (\ttt{inet\--frame\-work/inet}), 
      check out the revision of your choice, and follow the INSTALL 
      file in the project root.
\end{itemize}
 
 
\section{Installing INET Extensions}

If you plan to make use of INET extensions (e.g. Veins or SimuLTE),
follow the installation instructions provided with them. 

In the absence of specific instructions, the following procedure usually works: 

\begin{itemize}
 \item First, check if the project root contains a file named \ttt{.project}.
 \item If it does, then the project can be imported into the IDE (use \textit{File > Import >
    General > Existing Project} into workspace). make sure that the project is recognized
    as an OMNeT++ project (the \textit{Project Properties} dialog contains a page
    titled \textit{OMNeT++}), and it lists the INET project as dependency 
    (check the \textit{Project References} page in the \textit{Project Properties} dialog).
 \item If there is no \ttt{.project} file, you can create an empty OMNeT++
    project using the \textit{New OMNeT++ Project} wizard in \textit{File > New}, 
    add the INET project as dependency using the \textit{Project References} page 
    in the \textit{Project Properties} dialog, and copy the source files into the project.
\end{itemize}

\section{INET as an OMNeT++-based simulation framework}

The INET Framework builds upon OMNeT++, and uses the same concept: modules
that communicate by message passing. Hosts, routers, switches and other
network devices are represented by OMNeT++ compound modules. These compound
modules are assembled from simple modules that represent protocols,
applications, and other functional units. A network is again an OMNeT++
compound module that contains host, router and other modules. The external
interfaces of modules are described in NED files. NED files describe the
parameters and gates (i.e. ports or connectors) of modules, and also the
submodules and connections (i.e. netlist) of compound modules.

Modules are organized into hierarchical \textit{packages} that directly map to
a folder tree, very much like Java packages. Packages in
INET are organized roughly according to OSI layers; the top packages
include \ttt{inet.applications}, \ttt{inet.transportlayer},
\ttt{inet.networklayer}, \ttt{inet.linklayer}, and \ttt{inet.physicallayer}. 
Other packages are \ttt{inet.routing}, \ttt{inet.mobility}, \ttt{inet.power}, 
\ttt{inet.environment}, and \ttt{inet.node}. These packages correspond to the
\ttt{src/applications/}, \ttt{src/transportlayer/}, etc. directories in the
INET source tree. (The \ttt{src/inet/} directory corresponds to the \ttt{inet} 
package, as defined by the \ttt{src/inet/package.ned} file.) Subdirectories
within the top packages usually correspond to concrete protocols or protocol
families. The implementations of simple modules are C++ classes with the same
name, with the source files placed in the same directory as the NED file.

The \ttt{inet.node} package contains various pre-assembled host, router,
switch, access point, and other modules, for example
\nedtype{StandardHost}, \nedtype{Router} and \nedtype{EtherSwitch} and
\nedtype{AccessPoint}. These compound modules contain some customization
options via parametric submodule types, but they are not meant to be
universal, so it may be necessary to create your own node models for
your particular simulation scenarios.

Network interfaces (Ethernet, IEEE 802.11, etc) are usually compound modules
themselves, and are being composed of a queue, a MAC, and possibly other
simple modules. See \nedtype{EthernetInterface} as an example.

Not all modules implement protocols. There are modules which hold data (for
example \nedtype{Ipv4RoutingTable}), perform autoconfiguration of a network
(\nedtype{Ipv4NetworkConfigurator}), move a mobile node around (for example
\nedtype{ConstSpeedMobility}), and perform housekeeping associated with
radio signals in wireless simulations (\nedtype{RadioMedium}).

Protocol headers and packet formats are described in message definition
files (msg files), which are translated into C++ classes by OMNeT++'s
\textit{opp\_msgc} tool. The generated message classes subclass from OMNeT++'s
\ttt{cPacket} or \ttt{cMessage} classes.

The internal structure of compound modules such as host and router models
can be customized in several ways. The first way is the use of \textit{gate
vectors} and \textit{submodule vectors}. The sizes of vectors may come from
parameters or derived by the number of external connections to the module.
For example, one can have an Ethernet switch model that has as many ports
as needed, i.e. equal to the number of Ethernet devices connected to it.

The second way of customization is \textit{parametric types}, that is, the
type of a submodule (or a channel) may be specified as a string parameter.
For example, the relay unit inside an Ethernet switch has several
alternative implementations, each one being a distinct module type. The
switch model contains a parameter which allows the user to select the
appropriate relay unit implementation.

A third way of customizing modules is \textit{inheritance}: a derived
module may add new parameters, gates, submodules or connections, and may
set inherited unassigned parameters to specific values.


\section{Creating and Running Simulations}

To create a simulation, you would write a NED file that contains the network,
i.e. routers, hosts and other network devices connected together. You can
use a text editor or the IDE's graphical editor to create the network.

Modules in the network contain a lot of unassigned parameters, which need
to be assigned before the simulation can be run.\footnote{The simulator can
interactively ask for parameter values, but this is not very convenient
for repeated runs.} The name of the network to be simulated, parameter values
and other configuration option need to be specified in the \ttt{omnetpp.ini}
file.\footnote{This is the default file name; using other is also possible.}

\ttt{omnetpp.ini} contains parameter assignments as \textit{key=value}
lines, where each key is a wildcard pattern. The simulator matches these
wildcard patterns against full path of the parameter in the module tree
(something like \ttt{Test.host[2].tcp.nagleEnabled}), and value from
the first match will be assigned for the parameter. If no matching line is
found, the default value in the NED file will be used. (If there is no
default value either, the value will be interactively prompted for, or, in
case of a batch run, an error will be raised.)

There are two kinds of wildcards: a single asterisk \ttt{*} matches at most
one component name in the path string, while double asterisk \ttt{**} may
match multiple components. Technically: \ttt{*} never matches a dot or a
square bracket (\ttt{.}, \ttt{[}, \ttt{]}), while \ttt{**} can match any of
them. Patterns are also capable of expressing index ranges
(\ttt{**.host[1..3,5,8].tcp.nagleEnabled}) and ranges of numbers embedded
in names (\ttt{**.switch\{2..3\}.relayUnitType}).

OMNeT++ allows several configurations to be put into the \ttt{omnetpp.ini}
file under \ttt{[Config <name>]} section headers, and the right
configuration can be selected via command-line options when the simulation
is run. Configurations can also build on each other: \ttt{extends=<name>}
lines can be used to set up an inheritance tree among them. This feature
allows minimizing clutter in ini files by letting you factor out common
parts. (Another ways of factoring out common parts are ini file inclusion
and specifying multiple ini files to a simulation.) Settings in the
\ttt{[General]} section apply to all configurations, i.e. \ttt{[General]}
is the root of the section inheritance tree.

Parameter studies can be defined by specifying multiple values for a
parameter, with the \ttt{\$\{10,50,100..500 step 100, 1000\}} syntax;
a repeat count can also be specified.

\ifdraft TODO
how to run;

C++ -> dll (opp\_run) or exe
\fi

\section{Result Collection and Analysis}

how to analyize results

how to configure result collection


\section{Setting up wired network simulations}

For an introduction, in this section we show you how to set up simulations
of wired networks using PPP or Ethernet links with autoconfigured static IP
routing. (If your simulation involves more, such as manually configured
routing tables, dynamic routing, MPLS, or IPv6 or other features and protocols,
you'll find info about them in later chapters.)

Such a network can be assembled using the predefined \nedtype{StandardHost}
and \nedtype{Router} modules. For automatic IP address assignment and
static IP routing we can use the \nedtype{Ipv4NetworkConfigurator} utility
module.

TODO:  ethg, pppg;  automatically expand (++)

todo which modules are needed into it, what they do, etc.

how to add apps, etc


\section{Setting up wireless network simulations}

In this section we show you how to set up wireless network simulations using
IEEE 802.11 links with autoconfigured static IP routing. Such a network can be
assembled using the predefined \nedtype{WirelessHost} and \nedtype{AccessPoint}
modules. \nedtype{RadioMedium} is also required for wireless simulations. It
keeps track of which nodes are within interference distance of other nodes.

TODO

%%% Local Variables:
%%% mode: latex
%%% TeX-master: "usman"
%%% End:


\cleardoublepage

\chapter{Node Architecture}
\label{cha:node-architecture}

\section{Addresses}

The INET Framework uses a number of C++ classes to represent various
addresses in the network. These classes support initialization and
assignment from binary and string representation of the address, and
accessing the address in both forms. (Storage is in binary form). They also
support the "unspecified" special value (and the \ffunc{isUnspecified()}
method) that corresponds to the all-zeros address.

\begin{itemize}
  \item \cppclass{MacAddress} represents a 48-bit IEEE 802 MAC address. The
    textual notation it understands and produces is hex string.

  \item \cppclass{Ipv4Address} represents a 32-bit IPv4 address. It can parse
    and produce textual representations in the "dotted decimal" syntax.

  \item \cppclass{Ipv6Address} represents a 128-bit IPv6 address. It can parse
    and produce address strings in the canonical (RFC 3513) syntax.

  \item \cppclass{L3Address} is conceptually a union of a \cppclass{Ipv4Address}
    and \cppclass{Ipv6Address}: an instance stores either an IPv4 address or an
    IPv6 address. \cppclass{L3Address} is mainly used in the transport layer and above
    to abstract away network addresses. It can be assigned from both \cppclass{Ipv4Address}
    and \cppclass{Ipv6Address}, and can also parse string representations of both.
    The \ffunc{getType()}, \ffunc{toIPv4()} and \ffunc{toIPv6()} methods can be used
    to access the value.
\end{itemize}

\ifdraft TODO
TODO: Resolving addresses with L3AddressResolver
\fi


\ifdraft TODO
\section{The Notification Board}

The \nedtype{NotificationBoard} module allows publish-subscribe
communication among modules within a host. Using
\nedtype{NotificationBoard}, modules can notify each other about
events such as routing table changes, interface status changes
(up/down), interface configuration changes, wireless handovers, changes in
the state of the wireless channel, mobile node position changes, etc.
\nedtype{NotificationBoard} acts as a intermediary between the module where
the events occur and modules which are interested in learning about
those events.

\nedtype{NotificationBoard} has exactly one instance within a host or
router model, and it is accessed via direct C++ method calls (not message
exchange). Modules can \textit{subscribe} to categories of changes
(e.g. ``routing table changed'' or ``radio channel became empty''). When
such a change occurs, the corresponding module (e.g. the
\nedtype{Ipv4RoutingTable} or the physical layer module) will let
\nedtype{NotificationBoard} know, and it will disseminate this information
to all interested modules.

\sloppypar Notification events are grouped into \textit{categories}.
Examples of categories are: \ttt{NF\_HOSTPOSITION\_UPDATED},
\ttt{NF\_RADIOSTATE\_CHANGED}, \ttt{NF\_PP\_TX\_BEGIN},
\ttt{NF\_PP\_TX\_END}, \ttt{NF\_IPv4\_ROUTE\_ADDED},
\ttt{NF\_BEACON\_LOST}, \ttt{NF\_NODE\_FAILURE}, \ttt{NF\_NODE\_RECOVERY},
etc. Each category is identified by an integer (right now it's assigned in
the source code via an enum, in the future we'll convert to dynamic
category registration).

To trigger a notification, the client must obtain a pointer to the
\nedtype{NotificationBoard} of the given host or router (explained later),
and call its \ffunc{fireChangeNotification()} method. The notification will
be delivered to all subscribed clients immediately, inside the
\ffunc{fireChangeNotification()} call.

Clients that wish to receive notifications should implement (subclass from)
the \cppclass{INotifiable} interface, obtain a pointer to the
\nedtype{NotificationBoard}, and subscribe to the categories they are
interested in by calling the \ffunc{subscribe()} method of the
\nedtype{NotificationBoard}. Notifications will be delivered to the
\ffunc{receiveChangeNotification()} method of the client (redefined from
\cppclass{INotifiable}).

In cases when the category itself (an \ttt{int}) does not carry enough
information about the notification event, one can pass additional
information in a data class. There is no restriction on what the data class
may contain, except that it has to be subclassed from \cppclass{cObject},
and of course producers and consumers of notifications should agree on its
contents. If no extra info is needed, one can pass a \ttt{NULL} pointer in
the \ffunc{fireChangeNotification()} method.

A module which implements \cppclass{INotifiable} looks like this:

\begin{cpp}
class Foo : public cSimpleModule, public INotifiable {
    ...
    virtual void receiveChangeNotification(int category, const cObject *details) {..}
    ...
};
\end{cpp}

Note: \cppclass{cObject} was called \cppclass{cPolymorphic} in earlier versions
of OMNeT++. You may occasionally still see the latter name in source code; it
is an alias (typedef) to \cppclass{cObject}.

Obtaining a pointer to the \nedtype{NotificationBoard} module of that host/router:

\begin{cpp}
NotificationBoard *nb; // this is best made a module class member
nb = NotificationBoardAccess().get();  // best done in initialize()
\end{cpp}

TODO how to fire a notification
\fi


\section{The Interface Table}

The \nedtype{InterfaceTable} module holds one of the key data structures in
the INET Framework: information about the network interfaces in the host.
The interface table module does not send or receive messages; other modules
access it using standard C++ member function calls.

\ifdraft TODO
\subsection{Accessing the Interface Table}

If a module wants to work with the interface table, first it needs to obtain a
pointer to it. This can be done with the help of the
\cppclass{InterfaceTableAccess} utility class:

\begin{cpp}
IInterfaceTable *ift = InterfaceTableAccess().get();
\end{cpp}

\cppclass{InterfaceTableAccess} requires the interface table module to be a
direct child of the host and be called \ttt{"interfaceTable"} in order to
be able to find it. The \ffunc{get()} method never returns \ttt{NULL}: if
it cannot find the interface table module or cannot cast it to the
appropriate C++ type (\cppclass{IInterfaceTable}), it throws an exception
and stop the simulation with an error message.

For completeness, \cppclass{InterfaceTableAccess} also has a
\ffunc{getIfExists()} method which can be used if the code does not require
the presence of the interface table. This method returns \ttt{NULL} if the
interface table cannot be found.

Note that the returned C++ type is \cppclass{IInterfaceTable}; the initial
"\ttt{I}" stands for "interface". \cppclass{IInterfaceTable} is an abstract
class interface that \cppclass{InterfaceTable} implements. Using the abstract
class interface allows one to transparently replace the interface table with
another implementation, without the need for any change or even
recompilation of the INET Framework.
\fi

\subsection{Interface Entries}

Interfaces in the interface table are represented with the
\cppclass{InterfaceEntry} class. \cppclass{IInterfaceTable} provides member
functions for adding, removing, enumerating and looking up interfaces.

Interfaces have unique names and interface IDs; either can be used to look up
an interface (IDs are naturally more efficient). Interface IDs are invariant to
the addition and removal of other interfaces.

Data stored by an interface entry include:

\begin{itemize}
  \item \textit{name} and \textit{interface ID} (as described above)
  \item \textit{MTU}: Maximum Transmission Unit, e.g. 1500 on Ethernet
  \item several flags:
    \begin{itemize}
      \item \textit{down}: current state (up or down)
      \item \textit{broadcast}: whether the interface supports broadcast
      \item \textit{multicast} whether the interface supports multicast
      \item \textit{pointToPoint}: whether the interface is point-to-point link
      \item \textit{loopback}: whether the interface is a loopback interface
    \end{itemize}
  \item \textit{datarate} in bit/s
  \item \textit{link-layer address} (for now, only IEEE 802 MAC addresses are supported)
  \item \textit{network-layer gate index}: which gate of the network layer within the host the NIC is connected to
  \item \textit{host gate IDs}: the IDs of the input and output gate of the host the NIC is connected to
\end{itemize}

\tbf{Extensibility}. You have probably noticed that the above list does not
contain data such as the IPv4 or IPv6 address of the interface. Such
information is not part of \cppclass{InterfaceEntry} because we do not want
\nedtype{InterfaceTable} to depend on either the IPv4 or the IPv6 protocol
implementation; we want both to be optional, and we want
\nedtype{InterfaceTable} to be able to support possibly other network
protocols as well.

Thus, extra data items are added to \cppclass{InterfaceEntry} via
extension. Two kinds of extensions are envisioned: extension by the link
layer (i.e. the NIC), and extension by the network layer protocol:

\begin{itemize}

\item \tbf{NICs} can extend interface entries via C++ class inheritance, that is, by
simply subclassing \cppclass{InterfaceEntry} and adding extra data and
functions. This is possible because NICs create and register entries in
\nedtype{InterfaceTable}, so in their code one can just write
\ttt{new MyExtendedInterfaceEntry()} instead of \ttt{new InterfaceEntry()}.

\item \textbf{Network layer protocols} cannot add data via subclassing, so
composition has to be used. \cppclass{InterfaceEntry} contains pointers to
network-layer specific data structures. For example, there are pointers to
IPv4 specific data, and IPv6 specific data. These objects can be accessed with
the following \cppclass{InterfaceEntry} member functions: \ffunc{ipv4Data()},
\ffunc{ipv6Data()}, and \ffunc{getGenericNetworkProtocolData()}.
They return pointers of the types \cppclass{Ipv4InterfaceData},
\cppclass{Ipv6InterfaceData}, and \cppclass{GenericNetworkProtocolInterfaceData},
respectively. For illustration, \cppclass{Ipv4InterfaceData} is installed
onto the interface entries by the \nedtype{Ipv4RoutingTable} module, and it
contains data such as the IP address of the interface, the netmask, link
metric for routing, and IP multicast addresses associated with the
interface. A protocol data pointer will be \ttt{NULL} if the corresponding
network protocol is not used in the simulation; for example, in IPv4
simulations only \ffunc{ipv4Data()} will return a non-\ttt{NULL} value.


\end{itemize}


\subsection{Interface Registration}

Interfaces are registered dynamically in the initialization phase by modules
that represent network interface cards (NICs). The INET Framework makes use
of the multi-stage initialization feature of OMNeT++, and interface registration takes
place in the first stage (i.e. stage \ttt{INITSTAGE\_LINK\_LAYER}).

Example code that performs interface registration:

\begin{cpp}
void PPP::initialize(int stage)
{
    if (stage == INITSTAGE_LINK_LAYER) {
        ...
        interfaceEntry = registerInterface(datarate);
    ...
}

InterfaceEntry *PPP::registerInterface(double datarate)
{
    InterfaceEntry *e = new InterfaceEntry(this);

    // interface name: NIC module's name without special characters ([])
    e->setName(OPP_Global::stripnonalnum(getParentModule()->getFullName()).c_str());

    // data rate
    e->setDatarate(datarate);

    // generate a link-layer address to be used as interface token for IPv6
    InterfaceToken token(0, simulation.getUniqueNumber(), 64);
    e->setInterfaceToken(token);

    // set MTU from module parameter of similar name
    e->setMtu(par("mtu"));

    // capabilities
    e->setMulticast(true);
    e->setPointToPoint(true);

    // add
    IInterfaceTable *ift = findModuleFromPar<IInterfaceTable>(par("interfaceTableModule"), this);
    ift->addInterface(e);

    return e;
}
\end{cpp}


\ifdraft TODO
\subsection{Interface Change Notifications}

\nedtype{InterfaceTable} has a change notification mechanism built in, with
the granularity of interface entries.

Clients that wish to be notified when something changes in
\nedtype{InterfaceTable} can subscribe to the following notification
categories in the host's \nedtype{NotificationBoard}:

\begin{itemize}
  \item \tbf{\ttt{NF\_INTERFACE\_CREATED}}: an interface entry has been
    created and added to the interface table
  \item \tbf{\ttt{NF\_INTERFACE\_DELETED}}: an interface entry is going
    to be removed from the interface table. This is a pre-delete
    notification so that clients have access to interface data that are
    possibly needed to react to the change
% TODO hmmm -- rename the constant? also fire a post-change notification??
  \item \tbf{\ttt{NF\_INTERFACE\_CONFIG\_CHANGED}}: a configuration setting
    in an interface entry has changed (e.g. MTU or IP address)
  \item \tbf{\ttt{NF\_INTERFACE\_STATE\_CHANGED}}: a state variable in an
    interface entry has changed (e.g. the up/down flag)
\end{itemize}

In all those notifications, the data field is a pointer to the
corresponding \cppclass{InterfaceEntry} object. This is even true for
\ttt{NF\_INTERFACE\_DELETED} (which is actually a pre-delete notification).
\fi

\ifdraft TODO
\section{Initialization Stages}

Node architecture makes it necessary to use multi-stage initialization.
What happens in each stage is this:

In stage 0, interfaces register themselves in \nedtype{InterfaceTable} modules

In stage 1, routing files are read.

In stage 2, network configurators (e.g. \nedtype{FlatNetworkConfiguration})
assign addresses and set up routing tables.

In stage 3, TODO...

In stage 4, TODO...

...

The multi-stage initialization process itself is described in the OMNeT++ Manual.
\fi

\section{Communication between protocol layers}

In the INET Framework, when an upper-layer protocol wants to send a data
packet over a lower-layer protocol, the upper-layer module just sends the
message object representing the packet to the lower-layer module, which
will in turn encapsulate it and send it. The reverse process takes place
when a lower layer protocol receives a packet and sends it up after
decapsulation.

It is often necessary to convey extra information with the packet. For
example, when an application-layer module wants to send data over TCP, some
connection identifier needs to be specified for TCP. When TCP sends a
segment over IP, IP will need a destination address and possibly other
parameters like TTL. When IP sends a datagram to an Ethernet interface for
transmission, a destination MAC address must be specified. This extra
information is attached to the message object to as \textit{control info}.

Control info are small value objects, which are attached to packets
(message objects) with its \ttt{setControlInfo()} member function.
Control info only holds auxiliary information for the next protocol layer,
and is not supposed to be sent over the network to other hosts and routers.


\ifdraft TODO
\section{Publish-Subscribe Communication within Nodes}

The \nedtype{NotificationBoard} module makes it possible for several modules to
communicate in a publish-subscribe manner. For example, the radio module
(\nedtype{Ieee80211Radio}) fires a \textit{"radio state changed"} notification when
the state of the radio channel changes (from TRANSMIT to IDLE, for example),
and it will be delivered to other modules that have previously subscribed
to that notification category. The notification mechanism uses C++ functions
calls, no message sending is involved.

The notification board submodule within the host (router) must be called
\ttt{notificationBoard} for other modules to find it.
\fi

\ifdraft TODO
\section{Network interfaces}

todo...
\fi

\ifdraft TODO
\section{The wireless infrastructure}

todo...
\fi


\section{NED Conventions}

\subsection{The @node Property}

By convention, compound modules that implement network devices (hosts,
routers, switches, access points, base stations, etc.) are marked with the
\ttt{@node} NED property. As node models may themselves be hierarchical, the
\ttt{@node} property is used by protocol implementations and other simple
modules to determine which ancestor compound module represents the physical
network node they live in.

\subsection{The @labels Module Property}

The \ttt{@labels} property can be added to modules and gates, and it
allows the OMNeT++ graphical editor to provide better editing experience.
First we look at \ttt{@labels} as a module property.

\ttt{@labels(node)} has been added to all NED module types that may occur on
network level. When editing a network, the graphical editor will NED types
with \ttt{@labels(node)} to the top of the component palette, allowing the
user to find them easier.

Other labels can also be specified in the \ttt{@labels(...)} property. This
has the effect that if one module with a particular label has already been
added to the compound module being edited, other module types with the same
label are also brought to the top of the palette. For example,
\nedtype{EtherSwitch} is annotated with \ttt{@labels(node,ethernet-node)}.
When you drop an \nedtype{EtherSwitch} into a compound module, that will
bring \nedtype{EtherHost} (which is also tagged with the
\ttt{ethernet-node} label) to the top of the palette, making it easier to
find.

\begin{ned}
module EtherSwitch
{
    parameters:
        @node();
        @labels(node,ethernet-node);
        @display("i=device/switch");
    ...
}
\end{ned}

Module types that are already present in the compound module also appear in
the top part of the palette. The reason is that if you already added a
\nedtype{StandardHost}, for example, then you are likely to add more of the
same kind. Gate labels (see next section) also affect palette order: modules
which can be connected to modules already added to the compound module
will also be listed at the top of the palette. The final ordering is the
result of a scoring algorithm.


\subsection{The @labels Gate Property}

Gates can also be labelled with \ttt{@labels()}; the purpose is to make it easier
to connect modules in the editor. If you connect two modules in the editor,
the gate selection menu will list gate pairs that have a label in common.

\ifdraft TODO
screenshot
\fi

For example, when connecting hosts and routers, the editor will offer connecting
Ethernet gates with Ethernet gates, and PPP gates with PPP gates. This is the
result of gate labelling like this:

\begin{ned}
module StandardHost
{
    ...
    gates:
        inout pppg[] @labels(PPPFrame-conn);
        inout ethg[] @labels(EtherFrame-conn);
    ...
}
\end{ned}

Guidelines for choosing gate label names: For gates of modules that
implement protocols, use the C++ class name of the packet or acompanying
control info (see later) associated with the gate, whichever applies;
append \ttt{/up} or \ttt{/down} to the name of the control info class. For
gates of network nodes, use the class names of packets (frames) that travel
on the corresponding link, with the \ttt{-conn} suffix. The suffix prevents
protocol-level modules to be promoted in the graphical editor palette when
a network is edited.

Examples:

\begin{ned}
simple TCP like ITCP
{
    ...
    gates:
        input appIn[] @labels(TCPCommand/down);
        output appOut[] @labels(TCPCommand/up);
        input ipIn @labels(TCPSegment,IPv4ControlInfo/up,IPControlInfo/up);
        output ipOut @labels(TCPSegment,IPv4ControlInfo/down,IPv6ControlInfo/up);
}
\end{ned}


\begin{ned}
simple PPP
{
    ...
    gates:
        input netwIn;
        output netwOut;
        inout phys @labels(PPPFrame);
}
\end{ned}

%%% Local Variables:
%%% mode: latex
%%% TeX-master: "usman"
%%% End:


\cleardoublepage

\chapter{Point-to-Point Links}
\label{cha:ppp}


\section{Overview}

The INET Framework contains an implementation of the Point-to-Point Protocol
as described in RFC1661 with the following limitations:

\begin{itemize}
\item There are no LCP messages for link configuration,
link termination and link maintenance.
The link can be configured by setting module parameters.
\item PFC and ACFC are not supported, the PPP frame
always contains the 1-byte Address and Control fields
and a 2-byte Protocol field.
\item PPP authentication is not supported
\item Link quality monitoring protocols are not supported.
\item There are no NCP messages, the network layer protocols are
configured by other means.
\end{itemize}

The modules of the PPP model can be found in the \nedtype{inet.linklayer.ppp}
package:

\begin{description}
\item[\nedtype{Ppp}] This simple module performs encapsulation
of network datagrams into PPP frames and decapsulation of
the incoming PPP frames. It can be connected to the network
layer directly or can be configured to get the outgoing messages
from an output queue. The module collects statistics about
the transmitted and dropped packages.

\item[\nedtype{PppInterface}] is a compound module complementing
the \nedtype{Ppp} module with an output queue. It implements
the \nedtype{IWiredInterface} interface. Input and output hooks can be configured
for further processing of the network messages.

\end{description}

\section{PPP frames}

According to RFC1662 the PPP frames contain the following fields:

\begin{bytefield}[bitheight=2\baselineskip]{40}
\bitbox{8}{Flag \\ 01111110} &
\bitbox{8}{Address \\ 11111111} &
\bitbox{8}{Control \\ 00000011} \\
\bitbox{8}{Protocol \\ 8/16 bits} &
\bitbox{10}{Information \\ \textasteriskcentered } &
\bitbox{8}{Padding \\ \textasteriskcentered } \\
\bitbox{8}{FCS \\ 16/32 bits}
\bitbox{8}{Flag \\ 01111110 } &
\bitbox[ltb]{14}{Inter-frame Fill \\ or next Address}
\end{bytefield}

The corresponding message type in the INET framework is \msgtype{PppFrame}.
It contains the Information field as an encapsulated \cppclass{cMessage}
object. The Flag, Address and Control fields are omitted
from \msgtype{PppFrame} because they are constants. The FCS field is
omitted because no CRC computed during the simulation, the bit error
attribute of the \cppclass{cMessage} used instead. The Protocol
field is omitted because the protocol is determined from the class
of the encapsulated message.

The length of the PPP frame is equal to the length of the encapsulated
datagram plus 7 bytes. This computation assumes that
\begin{itemize}
\item there is no inter-octet time fill, so only one Flag sequence
needed per frame
\item padding is not applied
\item PFC and ACFC compression is not applied
\item FCS is 16 bit
\item no escaping was applied
\end{itemize}

\section{PPP module}

The PPP module receives packets from the upper layer in the \fvar{netwIn}
gate, encapsulates them into \msgtype{PppFrame}s, and send it to the
physical layer through the \fvar{phys} gate. The \msgtype{PppFrame}s
received from the \fvar{phys} gate are decapsulated and sent to the upper
layer immediately through the \fvar{netwOut} gate.

Incoming datagrams are waiting in a queue if the line is currently busy.
In routers, PPP relies on an external queue module (implementing
\nedtype{IOutputQueue}) to model finite buffer, implement QoS and/or RED,
and requests packets from this external queue one-by-one. The name
of this queue is given as the \fpar{queueModule} parameter.

In hosts, no such queue is used, so \nedtype{Ppp} contains an internal
queue named txQueue to queue up packets wainting for transmission.
Conceptually txQueue is of inifinite size, but for better diagnostics
one can specify a hard limit in the \fpar{txQueueLimit} parameter -- if
this is exceeded, the simulation stops with an error.

The module can be used in simulations where the nodes are connected and
disconnected dinamically. If the channel between the PPP modules is down,
the messages received from the upper layer are dropped (including the messages
waiting in the queue). When the connection is restored it will
poll the queue and transmits the messages again.

The PPP module registers itself in the interface table of the node.
The \fvar{mtu} of the entry can be specified by the
\fpar{mtu} module parameter. The module checks the state of the physical link
and updates the entry in the interface table.
% FIXME: The module does not notice if the datarate of the channel changed!
%        It should update the interface entry.

\ifdraft TODO
The node containing the PPP module must also contain a
\nedtype{NofiticationBoard} component. Notifications are sent when
transmission of a new PPP frame started (\verb!NF_PP_TX_BEGIN!), finished
(\verb!NF_PP_TX_END!) or when a PPP frame received (\verb!NF_PP_RX_END!).
\fi

\ifdraft TODO
The PPP component is the source of the following signals:
\begin{itemize}
\item \tbf{txState} state of the link (0=idle,1=busy)
\item \tbf{txPkBytes} number of bytes transmitted
\item \tbf{rxPkBytesOk} number of bytes received successfully
\item \tbf{droppedPkBytesBitError} number of bytes received in erronous frames
\item \tbf{droppedPkBytesIfaceDown} number of bytes dropped because the link is down
\item \tbf{rcvdPkBytesFromHL} number of bytes received from the the upper layer
\item \tbf{passedUpPkBytes} number of bytes sent to the the upper layer
\end{itemize}

These signals are recorded as statistics (sum, count and vector), so
they can be analyzed after the simulation.
\fi

When the simulation is executed with the graphical user interface
the module displays useful statistics. If the link is operating,
the datarate and number of received, sent and dropped messages
show in the tooltip. When the link is broken, the number of dropped
messages is displayed. The state of the module is indicated by the color
of the module icon and the connection (yellow=transmitting).

\section{PPPInterface module}

The \nedtype{PppInterface} is a compound module implementing the \nedtype{IWiredInterface}
interface. It contains a \nedtype{Ppp} module and a passive queue for the messages
received from the network layer.

The queue type is specified by the \fpar{queueType} parameter.
It can be set to \nedtype{NoQueue} or to a module type implementing
the \nedtype{IOutputQueue} interface. There are implementations
with QoS and RED support.

In typical use of the \nedtype{Ppp} module it is augmented with other nodes
that monitor the traffic or simulate package loss and duplication.
The \nedtype{PppInterface} module abstract that usage by adding
\nedtype{IHook} components to the network input and output of the
\nedtype{Ppp} component. Any number of hook can be added by
specifying the \fpar{numOutputHooks} and \fpar{numInputHooks}
parameters and the types of the \fvar{outputHook} and \fvar{inputHook}
components. The hooks are chained in their numeric order.


%%% Local Variables:
%%% mode: latex
%%% TeX-master: "usman"
%%% End:



\cleardoublepage

% last synchronized to 'dbc28949bf4332ac86d84b95705fbea9af4f84f7'
\chapter{The Ethernet Model}
\label{cha:ethernet}

% TODO: comment numWirelessPorts in MacRelayUnitPP
% TODO: comment origByteLength in EtherFrame
% FIXME: wrong header length in EtherFrame.msg

\section{Overview}

Variations: 10Mb/s ethernet, fast ethernet, Gigabit Ethernet, Fast Gigabit Ethernet, full duplex

The Ethernet model contains a MAC model (\nedtype{EtherMac}), LLC model (\nedtype{EtherLlc}) as well
as a bus (\nedtype{EtherBus}, for modelling coaxial cable) and a hub (\nedtype{EtherHub}) model.
A switch model (\nedtype{EtherSwitch}) is also provided.

\begin{itemize}
  \item \nedtype{EtherHost} is a sample node with an Ethernet NIC;
  \item \nedtype{EtherSwitch}, \nedtype{EtherBus}, \nedtype{EtherHub} model switching hub, repeating hub and
        the old coxial cable;
  \item basic components of the model: \nedtype{EtherMac}, \nedtype{EtherLlc}/\nedtype{EtherEncap} module types,
        \nedtype{MacRelayUnit} (\nedtype{MACRelayUnitNP} and \nedtype{MACRelayUnitPP}), \nedtype{EtherFrame} message type,
        \cppclass{MacAddress} class
\end{itemize}


\section{Physical layer}

Stations on an Ethernet networks are connected by coaxial,
twisted pair or fibre cables. (Coaxial only has historical importance,
but is supported by INET anyway.) There are several cable types specified
in the standard.

In the INET framework, the cables are represented by connections.
The connections used in Ethernet LANs must be derived from
\nedtype{DatarateConnection} and should have their \fpar{delay} and
\fpar{datarate} parameters set.
The delay parameter can be used to model the distance between the
nodes. The datarate parameter can have four values:

\begin{itemize}
  \item \ttt{10Mbps} classic Ethernet
  \item \ttt{100Mbps} Fast Ethernet
  \item \ttt{1Gbps} Gigabit Ethernet
  \item \ttt{10Gbps} Fast Gigabit Ethernet
\end{itemize}


\subsection{EtherHub}

Ethernet hubs are a simple broadcast devices. Messages arriving on a port
are regenerated and broadcast to every other port.

The connections connected to the hub must have the same data rate.
Cable lengths should be reflected in the delays of the connections.

Messages are not interpreted by the \nedtype{EtherType} hub model in any way,
thus the hub model is not specific to Ethernet. Messages may
represent anything, from the beginning of a frame transmission to
end (or abortion) of transmission.

% TODO: model delay in hubs: class I device 140 bit time, class II device 92 bit time (for fast ethernet)

\subsection{EtherBus}

The \nedtype{EtherBus} component can model a common coaxial cable
found in early Ethernet LANs. The nodes are attached via taps at specific
positions on the cable. When a node sends a signal, it will propagate
along the cable in both directions at the given propagation speed.

The gates of the \nedtype{EtherBus} represent taps. The positions
of the taps are given by the \fpar{positions} parameter as a
space separated list of distances in metres. If there are more
gates then positions given, the last distance is repeated.
The bus component send the incoming message in one direction and
a copy of the message to the other direction (except at the ends).
The propagation delays are computed from the distances of the taps
and the \fpar{propagationSpeed} parameter.

Messages are not interpreted by the bus model in any way, thus the bus
model is not specific to Ethernet. Messages may represent anything, 
from the beginning of a frame transmission to end (or abortion) of transmission.

% FIXME #356 NED comment is wrong: data rate must not be zero!
% FIXME #354 default propagation speed is wrong (should be 2e8mps)
%            btw there is a hard coded propagation speed in EtherMACBase.cc


\section{Ethernet Interfaces}

\subsection{EthernetInterface}

The \nedtype{EthernetInterface} compound module implements the \nedtype{IWiredInterface}
interface. Complements \nedtype{EtherMac} and \nedtype{EtherEncap} with an output queue
for QoS and RED support. It also has configurable input/output filters as \nedtype{IHook}
components similarly to the \nedtype{PppInterface} module.

% TODO there is no IWiredNic with EtherLLC


\subsection{Ethernet MAC Layer}

The Ethernet MAC (Media Access Control) layer transmits the Ethernet frames on
the physical media. This is a sublayer within the data link layer. Because
encapsulation/decapsulation is not always needed (e.g. switches does not do
encapsulation/decapsulation), it is implemented in a separate modules
(\nedtype{EtherEncap} and \nedtype{EtherLlc}) that are part of the LLC layer.

\subsection{Implemented Standards}

The Ethernet model operates according to the following standards:

\begin{itemize}
  \item Ethernet: IEEE 802.3-1998
  \item Fast Ethernet: IEEE 802.3u-1995
  \item Full-Duplex Ethernet with Flow Control: IEEE 802.3x-1997
  \item Gigabit Ethernet: IEEE 802.3z-1998
\end{itemize}

Nowadays almost all Ethernet networks operate using full-duplex
point-to-point connections between hosts and switches. This means
that there are no collisions, and the behaviour of the MAC component
is much simpler than in classic Ethernet that used coaxial cables and
hubs. The INET framework contains two MAC modules for Ethernet:
the \nedtype{EtherMacFullDuplex} is simpler to understand and easier to extend,
because it supports only full-duplex connections. The \nedtype{EtherMac}
module implements the full MAC functionality including CSMA/CD, it
can operate both half-duplex and full-duplex mode.

\subsection*{Packets and frames}

The environment of the MAC modules is described by the \nedtype{IEtherMac}
module interface. Each MAC modules has gates to connect to the physical
layer (\ttt{phys\$i} and \ttt{phys\$o}) and to connect to the upper layer
(LLC module is hosts, relay units in switches): \ttt{upperLayerIn} and
\ttt{upperLayerOut}.

When a frame is received from the higher layers, it must be an
\msgtype{EtherFrame}, and with all protocol fields filled out
(including the destination MAC address). The source address, if left empty,
will be filled in with the configured \fpar{address} of the MAC.
% TODO document auto MAC address


Packets received from the network are \msgtype{EtherTraffic} objects.
They are messages representing inter-frame gaps (\msgtype{EtherPadding}),
jam signals (\msgtype{EtherJam}), control frames (\msgtype{EtherPauseFrame})
or data frames (all derived from \msgtype{EtherFrame}). Data frames
are passed up to the higher layers without modification.
In \fpar{promiscuous} mode, the MAC passes up all received frames;
otherwise, only the frames with matching MAC addresses and
the broadcast frames are passed up.

Also, the module properly responds to PAUSE frames, but never sends them
by itself -- however, it transmits PAUSE frames received from upper layers.
See section~\ref{subsec:pause_handling} for more info.

\subsection*{Queueing}

When the transmission line is busy, messages received from the upper layer
needs to be queued.

In routers, MAC relies on an external queue module (see \nedtype{OutputQueue}),
and requests packets from this external queue one-by-one. The name of the
external queue must be given as the \fpar{queueModule} parameer.
There are implementations of \nedtype{OutputQueue} to model finite buffer,
QoS and/or RED.

In hosts, no such queue is used, so MAC contains an internal
queue named \fvar{txQueue} to queue up packets waiting for transmission.
Conceptually, \fvar{txQueue} is of infinite size, but for better diagnostics
one can specify a hard limit in the \fpar{txQueueLimit} parameter -- if this is
exceeded, the simulation stops with an error.

\subsection*{PAUSE handling}
\label{subsec:pause_handling}

The 802.3x standard supports PAUSE frames as a means of flow
control. The frame contains a timer value, expressed as a multiple
of 512 bit-times, that specifies how long the transmitter should
remain quiet. If the receiver becomes uncongested before the
transmitter's pause timer expires, the receiver may elect to send
another PAUSE frame to the transmitter with a timer value of zero,
allowing the transmitter to resume immediately.

\nedtype{EtherMac} will properly respond to PAUSE frames it receives
(\msgtype{EtherPauseFrame} class),
however it will never send a PAUSE frame by itself. (For one thing,
it doesn't have an input buffer that can overflow.)

\nedtype{EtherMac}, however, transmits PAUSE frames received by higher layers,
and \nedtype{EtherLlc} can be instructed by a command to send a PAUSE frame to MAC.

% FIXME PAUSE frames should only be sent on full-duplex ethernet.
%       If a switch uses half-duplex mode to connect to hosts, it can ask sending hosts
%       to slow down their sending rates:
%       - force collisions with incoming frames
%       - make it appear as if the channel is busy
% FIXME PAUSE frames should have 0x8808 in the etherType field

\subsection*{Error handling}

If the MAC is not connected to the network ("cable unplugged"), it will
start up in "disabled" mode. A disabled MAC simply discards any messages
it receives. It is currently not supported to dynamically connect/disconnect
a MAC.

CRC checks are modeled by the \fvar{bitError} flag of the packets. Erronous
packets are dropped by the MAC.


%\subsection*{Auto-Negotiation}
% Ethernet Auto-Negotiation not supported



\subsection{EtherMacFullDuplex}

From the two MAC implementation \nedtype{EtherMacFullDuplex} is the simpler one,
it operates only in full-duplex mode (its \fpar{duplexEnabled} parameter fixed to
\ttt{true} in its NED definition). This module does not need to implement
CSMA/CD, so there is no collision detection, retransmission with exponential backoff,
carrier extension and frame bursting. Flow control works as described in section
\ref{subsec:pause_handling}.

% FIXME remove frameBursting from NED def, or fix it to false
%       currently setting it to 'true' has no effect

In the \nedtype{EtherMacFullDuplex} module,
packets arrived at the \ttt{phys\$i} gate are handled when their last bit received.

Outgoing packets are transmitted according to the following state diagram:

\begin{center}
\includegraphics{figures/EtherMACFullDuplex_txstates}
\end{center}

The \nedtype{EtherMacFullDuplex} module records two scalars in addition to the
ones mentioned earlier:
\begin{itemize}
\item \ttt{rx channel idle (\%)}: reception channel idle time
        as a percentage of the total simulation time
\item \ttt{rx channel utilization (\%)}: total reception
        time as a percentage of the total simulation time
\end{itemize}

\subsection{EtherMac}

Ethernet MAC layer implementing CSMA/CD. It supports both half-duplex and full-duplex operations;
in full-duplex mode it behaves as \nedtype{EtherMacFullDuplex}. In half-duplex  mode
it detects collisions, sends jam messages and retransmit frames upon collisions using
the exponential backoff algorithm. In Gigabit Ethernet networks it supports carrier
extension and frame bursting. Carrier extension can be turned off by setting the
\fpar{carrierExtension} parameter to \ttt{false}.

Unlike \nedtype{EtherMacFullDuplex}, this MAC module processes the incoming packets when their
first bit is received. The end of the reception is calculated by the MAC and
detected by scheduling a self message.

When frames collide the transmission is aborted -- in this case the transmitting
station transmits a jam signal. Jam signals are represented
by a \msgtype{EtherJam} message. The jam message contains the tree identifier
of the frame whose transmission is aborted. When the \nedtype{EtherMac} receives a jam
signal, it knows that the corresponding transmission ended in jamming and have
been aborted. Thus when it receives as many jams as collided frames, it can
be sure that the channel is free again. (Receiving a jam message marks the
beginning of the jam signal, so actually has to wait for the duration of the jamming.)

The operation of the MAC module can be schematized by the following state chart:

\begin{center}
\includegraphics{figures/EtherMAC_txstates}
\end{center}

The module generates these extra signals:
\begin{itemize}
\item \fsignal{collision} when collision starts (received a frame,
         while transmitting or receiving another one; or start to transmit while receiving a frame),
         the constant value 1
\item \fsignal{backoff} when jamming period ended and before waiting according to the
         exponential backoff algorith, the constant value 1
\end{itemize}

These scalar statistics are generated about the state of the line:
\begin{itemize}
  \item \ttt{rx channel idle (\%)} reception channel idle time (full duplex) or channel
         idle time (half-duplex), as a percentage of the total simulation time
  \item \ttt{rx channel utilization (\%)} total successful reception time (full-duplex) or total
         successful reception/transmission time (half duplex), as a percentage
         of the total simulation time
  \item \ttt{rx channel collision (\%)} total unsuccessful reception time, as a percentage
         of the total simulation time
  \item \ttt{collisions} total number collisions (same as count of \fsignal{collisionSignal})
  \item \ttt{backoffs} total number of backoffs (same as count of \fsignal{backoffSignal})
\end{itemize}

\subsection{EtherEncap}

The \nedtype{EtherEncap} module generates \msgtype{EthernetIIFrame} messages.

EtherFrameII

\subsection{EtherLlc}

TODO what it does


% document error conditions (causing error() calls in the code)

% FIXME handleRestransmission() comment is not true: // no beginSendFrames(), because end of jam signal sending will trigger it automatically
%       in case of inner queue, the queued msg is not transmitted
% FIXME should not enter PAUSE state when !duplexMode


\section{Switches}

Ethernet switches play an important role in modern Ethernet LANs. Unlike
passive hubs and repeaters, that work in the physical layer, the switches
operate in the data link layer and routes data frames between the connected
subnets.

While a hub repeats the data frames on each connected line, possibly causing
collisions, switches help to segment the network to small collision domains.
In modern Gigabit LANs each node is connected to the switch direclty
by full-duplex lines, so no collisions are possible. In this case the
CSMA/CD is not needed and the channel utilization can be high.

\subsection{MAC relay units}

INET framework ethernet switches are built from \nedtype{IMacRelayUnit}
components. Each relay unit has N input and output gates for sending/receiving
Ethernet frames. They should be connected to \nedtype{IEtherMac} modules.

Internally the relay unit holds a table for the destination address -> output
port mapping. When it receives a data frame it updates the table with the
source address->input port. The table can also be pre-loaded from a text file
while initializing the relay unit. The file name given as the \fpar{addressTableFile}
parameter. Each line of the file contains a hexadecimal MAC address and a decimal port
number separated by tabs. Comment lines beginning with '\#' are also allowed:

\begin{verbatim}
01 ff ff ff ff    0
00-ff-ff-ee-d1    1
0A:AA:BC:DE:FF    2
\end{verbatim}

% FIXME #352 addressTableSize is not checked in readAddressTable -> if overflown
%            then later check updateTableWithAddress has no effect
% FIXME format is wrong in the comment of readAddressTable()

The size of the lookup table is restricted by the \fpar{addressTableSize} parameter.
When the table is full, the oldest address is deleted. Entries are also deleted
if their age exceeds the duration given as the \fpar{agingTime} parameter.

If the destination address is not found in the table, the frame is broadcasted.
The frame is not sent to the same subnet it was received from, because the
target already received the original frame. The only exception if the frame
arrived through a radio channel, in this case the target can be out of range.
The port range 0..\fpar{numWirelessPorts}-1 are reserved for wireless connections.

The \nedtype{IMacRelayUnit} module is not a concrete implementation,
it just defines gates and parameters an \nedtype{IMacRelayUnit} should have.
Concrete inplementations add
capacity and performance aspects to the model (number of frames processed
per second, amount of memory available in the switch, etc.)
C++ implementations can subclass from the class \cppclass{MACRelayUnitBase}.

There are two versions of \nedtype{IMacRelayUnit}:

\begin{description}
  \item[\nedtype{MACRelayUnitNP}] models one or more CPUs with shared memory,
    working from a single shared queue.
  \item[\nedtype{MACRelayUnitPP}] models one CPU assigned to each incoming port,
    working with shared memory but separate queues.
\end{description}

In both models input messages are queued. CPUs poll messages from the queue
and process them in \fpar{processingTime}. If the memory usage exceeds
\fpar{bufferSize}, the frame will be dropped.

A simple scheme for sending PAUSE frames is built in (although
users will probably change it). When the buffer level goes
above a high watermark, PAUSE frames are sent on all ports.
The watermark and the pause time is configurable; use zero
values to disable the PAUSE feature.

% FIXME valid values for pauseTime: 0..0xFFFF
% FIXME ETHER_PAUSE_COMMAND_BYTES should be 4 in Ethernet.h (2bytes opcode + 2bytes pauseTime)
% FIXME PAUSE frame should not be sent on all ports probably
% TODO add lowWatermark, send PauseFrame(pauseUnits=0) to resume sending

The relay units collects the following statistics:

\begin{description}
\item[usedBufferBytes] memory usage as function of time
\item[processedBytes] count and length of processed frames
\item[droppedBytes] count and length of frames dropped caused by out of memory
\end{description}

% FIXME MACRelayUnitNP: no signals are generated, how does @statistic work in the ned file?

\subsection{EtherSwitch}

Model of an Ethernet switch containing a relay unit and multiple MAC units.

The duplexChannel attributes of the MACs must be set according to the
medium connected to the port; if collisions are possible (it's a bus or hub)
it must be set to false, otherwise it can be set to true.
By default it uses half duples MAC with CSMA/CD.

TODO STP, RSTP



%%% Local Variables:
%%% mode: latex
%%% TeX-master: "usman"
%%% End:

\cleardoublepage

\chapter{The Physical Environment}
\label{cha:environment}

\section{Overview}

Wireless networks are heavily affected by the physical environment, and the
requirements for today's ubiquitous wireless communication devices are
increasingly demanding. Cellular networks serve densely populated urban
areas, wireless LANs need to be able to cover large buildings with several
offices, low-power wireless sensors must tolerate noisy industrial
environments, batteries need to remain operational under various external
conditions, and so on.

The propagation of radio signals, the movement of communicating agents,
battery exhaustion, etc., depend on the surrounding physical environment.
For example, signals can be absorbed by objects, can pass through objects,
can be refracted by surfaces, can be reflected from surfaces, or battery
nominal capacity might depend on external temperature. These effects cannot
be ignored in high-fidelity simulations.

In order to help the modeling process, the model of the physical
environment is separated from the rest of the simulation models. The main
goal of the physical environment model is to describe buildings, walls,
vegetation, terrain, weather, and other physical objects and conditions
that might have effects on radio signal propagation, movement, batteries,
etc. This separation makes the model reusable by all other simulation
models that depend on these circumstances.

The following sections provide a brief overview of the physical environment
model.

\section{The Physical Environment Model}

The physical environment is represented in an INET simulation by a
\nedtype{PhysicalEnvironment} module. This module normally has one instance
in the network, and acts as a database that other parts of the simulation
can query at runtime.

\section{Global Physical Properties}

The physical environment model stores the following global
properties:

\begin{itemize}
  \item \textit{space limits}: global bounds for the 3-dimensional space
  \item \textit{temperature}: global parameter for temperature-dependent models
\end{itemize}

Space limits are useful for limiting the propagation and reflection of
radio signals, to constrain movement of communicating agents, and for
detecting incorrectly positioned physical objects.

Temperature can be useful for modeling batteries, as it affects the
maximum capacity, internal resistance, self-discharge and other properties
of real-life electrochemical energy storage devices.

\section{Physical Objects}

The most important aspect of the physical environment is the objects which
are present in it. For example, simulating an indoor Wifi scenario may need
to model walls, floors, ceilings, doors, windows, furniture, and similar
objects, because they all affect signal propagation.

Objects are located in space, and have shapes and materials. The INET
physical layer infrastructure supports basic shapes and homogeneous
materials, which simplifies description and still allows for a reasonable
approximation of reality. Physical objects in INET have the following
properties:

\begin{itemize}
  \item \textit{shape}: describes the 3-dimensional shape of the object,
    independent of its position and orientation
  \item \textit{position}: determines where the object is located in the 3-dimensional space
  \item \textit{orientation}: determines how the object is rotated relative to its
    default orientation
  \item \textit{material}: describes material-specific physical properties
  \item \textit{graphical properties}: provides parameters for better visualization
\end{itemize}

Physical objects in INET are stationary, they cannot change their position
or orientation over time.

Since the shape of the physical objects might be quite diverse, the model
is designed to be extensible with new shapes. Concave shapes are not yet
supported, such shapes can be represented by splitting them up into smaller,
convex parts. The current implementation provides the following shapes:

\begin{itemize}
  \item \textit{sphere}: specified by a radius
  \item \textit{cuboid}: specified by a length, a width, and a height
  \item \textit{prism}: specified by a 2-dimensional polygon base and a height
  \item \textit{polyhedron}: specified by the convex hull of a set of
    3-dimensional vertices
\end{itemize}

\section{Visualization}

The \nedtype{PhysicalEnvironment} module is capable is visualizing the objects on the
user interface. Rendering makes use of the following graphical object properties:

\begin{itemize}
  \item \textit{line width}: affects surface outline
  \item \textit{line color}: affects surface outline
  \item \textit{fill color}: affects surface fill
  \item \textit{opacity}: affects surface outline and fill
  \item \textit{tags}: allows filtering objects on the graphical user interface
\end{itemize}

The projection of 3D objects to the 2D canvas can be parameterized with an
arbitrary view angle. The default view angle is the Z axis (i.e. top view).
The view angle can also be changed during runtime, by changing the
appropriate module parameter.

The projection mechanism can be accessed by other models (e.g. mobility
models) for their own visualizations.

\section{Specifying Physical Objects}

Physical objects are defined for the \nedtype{PhysicalEnvironment} module
in an XML document. The document format allows one to define physical objects
together with their properties, and one can also define shapes and materials
that are shared (i.e. referenced) by several objects.

The following example shows some shapes, materials and objects defined in XML:

\begin{verbatim}
<environment>
  <!-- shapes and materials -->
  <shape id="1" type="sphere" radius="10"/>
  <shape id="2" type="cuboid" size="20 30 40"/>
  <shape id="3" type="prism" height="100"
         points="0 0 100 0 100 100 0 100"/>
  <shape id="4" type="polyhedron"
         points="0 0 0 100 0 0 100 100 0 0 100 0 ..."/>
  <material id="1" resistivity="100"
         relativePermittivity="4.5" relativePermeability="1"/>

  <!-- an object that uses a previously defined shape and material -->
  <object position="min 100 200 0" orientation="45 -30 0"
          shape="1" material="1"
          line-color="0 0 0" fill-color="112 128 144" opacity="0.5"/>

  <!-- an object defined with an in-line shape -->
  <object position="min 100 200 0" orientation="45 -30 0"
          shape="cuboid 20 30 40" material="concrete"
          line-color="0 0 0" fill-color="112 128 144" tags="Building"/>
</environment>
\end{verbatim}

For more details, please refer to the documentation of the
\nedtype{PhysicalEnvironment} module.

\section{Data Structure}

In order to model the physical environment in detail, a scenario might contain
several thousands or even more physical objects. Simulation models might
need to query these objects quite often. For example, when the physical layer
computes obstacle loss for a transmission, it needs to find the obstructing
physical objects for each receiver. This requires computing the intersection
between physical objects and the path traveled by the radio signal.

To speed up the computation of intersections, INET stores physical objects
in a highly efficient data structure, which currently can be one of the
following:

\begin{itemize}
  \item \nedtype{GridObjectCache}: organizes objects into a 3D spatial grid with
    a configurable constant cell size, where cells contain the objects that
    intersect with them
  \item \nedtype{BvhObjectCache}: organizes objects into a 3D tree, where
    leaves contain a configurable number of closely positioned objects.
    (This data structure is similar to quadtree and octree, but is designed for
    storing finite-sized objects.)
\end{itemize}

The physical environment model uses \nedtype{GridObjectCache} by default.

%%% Local Variables:
%%% mode: latex
%%% TeX-master: "usman"
%%% End:


\cleardoublepage

\include{ch-power}
\cleardoublepage

\chapter{The Physical Layer}
\label{cha:physicallayer}

\section{Overview}

Today's electric devices use more and more wireless communication methods such
as Wifi, Bluetooth, NFC, UMTS, and LTE. Despite the diversity of these devices
there are many similarities in the modeling of their physical layer components.
The models often have similar signal representations and signal processing steps,
and they also share the physical medium model where communication takes place.

In general, the physical layer simulation is a very time consuming task. The
simulation of signal propagation, signal fading, signal interference, and signal
decoding in detail may often result in unacceptable performance. Finding the
right abstractions, the right level of detail, and the right trade-offs between
accuracy and performance is difficult and very important.

To summarize, the physical layer is designed with the following goals in mind:
\begin{itemize}
  \item customizability
  \item extensibility
  \item scalable level of detail
  \item ability to exploit parallel hardware
\end{itemize}

The following sections provide a brief overview of the physical layer model. For
more details on the available modules, their parameterization and the actual
implementations please refer to the documentation in the corresponding NED and
C++ source files.

\subsection{Customizability}

Real world communication devices often provide a wide variety of configuration
options to allow adapting to the physical conditions where they are required
to operate. For example, a Wifi router administration interface often provides
parameters to configure the transmission power, bitrate, preamble type, carrier
frequency, RTS threshold, beacon interval, etc. Mostly these parameters have
default values assigned, so the user doesn't have to set them separately, but
may override them as needed.

Similarly to real world devices the physical layer models also provide a wide
variety of parameters to control their behavior. The most common NED parameters
are various physical quantities with physical units such as transmission power
\ttt{[W]}, reception sensitivity \ttt{[W]}, carrier frequency \ttt{[Hz]},
communication range \ttt{[m]}, propagation speed \ttt{[m/s]}, SNIR reception
threshold \ttt{[dB]}, bitrate \ttt{[b/s]}. Occasionally models support new
parameters or new combinations, which don't exist in real world hardware, to
allow further experimentation.

Another important and commonly used parameter kind selects among alternative
implementations of a particular interface by providing its name. Different
implementations are often separate modules, which come with their own set of
parameters to avoid the confusion of mixing their unrelated parameters. Some
modules may be split into more submodules. This further deepens the module
hierarchy, but allows better extensibility.

\subsection{Extensibility}

Similarly to designing other simulation models, modeling the physical layer is
not at all an unambiguous task. For example, the research literature contains a
number of different path loss models for signal propagation, there are different
bit error models for a particular protocol standard, representing the signal in
the analog domain can also be done in several different ways, and so on.

In order to support this diversity the physical layer is designed to be
extensible with alternative implementations at various parts of the model. This
is realized by separately defining C++ and NED interfaces between modules, and
also by providing parameters in their parent modules to easily select among the
available implementations.

New models can be added by implementing the required interfaces from scratch, or
by deriving from already existing implementations and overriding functionality.
This architecture allows the user to create new models with less effort, and to
focus on the real differences, while the rest of the physical layer remains the
same.

\subsection{Scalable Level of Detail}

There are many possible ways to model various aspects of the physical layer.
The most important difference lies in the trade-off between performance versus
accuracy. In order to support the different trade-offs the physical layer is
designed to be scalable with respect to the simulated level of detail. In other
words, it's scalable from high-performance less accurate simulations to high
fidelity slower simulations.

The physical layer model is scalable along the following axes:

\begin{itemize}
  \item simulation model
  \item software architecture
  \item data representation
  \item number of messages
\end{itemize}

The simulation model might vary from simple statistical models to accurate
emulation. The simplest models ignore the actual bits of the transmission. For
example, the extremely simple unit disc radio even ignores the signal power. The
most accurate models use precise signal representation for all four domains:
bit domain, symbol domain, sample domain, and analog domain representations.
They also emulate most functions of real hardware in detail: forward error
correction, interleaving, scrambling, modulation, spreading, pulse shaping, and
so on.

The software architecture might vary from flat to layered. A flat architecture
is efficient but not modular. Functionality can only be affected through simple
parameters and not by providing alternative implementations. Whereas a layered
architecture is more flexible at the cost of more complex data structures, more
data conversions, more resource management, and thus slower processing. On the
other hand, it provides more customization opportunities to replace parts with
alternative implementations and to do research easier in the area.

The data representation might vary from scalar to multidimensional values. In
the analog domain of the physical layer data quite often changes over time,
frequency, space, or any combination thereof. The most obvious example is the
analog signal power, but there are others such as signal phase or the signal to
noise ratio.

The number of messages per transmission added to the future event queue might
vary from one to the number of radios. One message might be sufficient, for
example, if the transmission is intended to a single destination, and other
receivers are either not affected, or the effect is negligible. On the other
hand, it might be necessary to process all transmissions by all receivers in
order to have the desired effect on the higher layers. For example, if a MAC
model is configured to promiscuous mode, it needs to receive all transmissions.

\subsection{Exploiting Parallel Hardware}

The physical processes simulated by the physical layer are inherently parallel.
The computation of the transmission arrival space-time coordinates, the analog
signal representation of transmissions and receptions, the interfering
receptions and noises, the signal to noise ratio, the decoded bits, the bit
errors, and the physical layer indications all provide a good parallelization
opportunity, because they dominate the physical layer performance and are
independent for each receiver. Therefore the physical layer is designed to be
able to utilize parallel hardware, multi-core CPUs, vector instructions and the
highly parallel GPU.

The idea is to have a central component in the software architecture where
parallel computation can happen. This central component is the medium model
that knows about all radios, transmissions, interferences, and receptions
anyway. It uses optimistic parallel computation in multiple background threads
while the main simulation thread continues normal execution. When a new
transmission enters the channel the already computed and affected results are
invalidated or updated, and the affected ongoing optimistic parallel
computations are canceled.

\section{The Radio Model}

The radio model describes the physical device that is capable of transmitting
and receiving signals on the medium. It contains an antenna model, a transmitter
model, a receiver model, and an energy consumer model. The antenna model is
shared between the transmitter model and the receiver model. The separation of
the transmitter model and the receiver model allows asymmetric configurations.
The energy consumer model is optional and it's only used when the simulation of
energy consumption is necessary.

The radio model has an operational mode that is called the radio mode. The radio
mode is externally controlled usually by the MAC model. In transceiver mode, the
radio can simultaneously transmit and receive a signal. Changing the radio mode
may optionally take a non-zero amount of time. The supported radio modes are the
following:

\begin{itemize}
  \item \ttt{off}: communication isn't possible, energy consumption is zero
  \item \ttt{sleep}: communication isn't possible, energy consumption is minimal
  \item \ttt{receiver}: only reception is possible, energy consumption is low
  \item \ttt{transmitter}: only transmission is possible, energy consumption is
high
  \item \ttt{transceiver}: reception and transmission is simultaneously
possible, energy consumption is high
  \item \ttt{switching}: communication isn't possible, energy consumption is
minimal
\end{itemize}

In addition to the radio mode, the transmitter and the receiver models have
separate states which describe what they are doing. Changes to these states are
automatically published by the radio. The signaled transmitter states are the
following:

\begin{itemize}
  \item \ttt{undefined}: isn't operating
  \item \ttt{idle}: there's no transmission in progress
  \item \ttt{transmitting}: transmission is in progress
\end{itemize}

The signaled receiver states are the following:

\begin{itemize}
  \item \ttt{undefined}: isn't operating
  \item \ttt{idle}: there's no reception in progress
  \item \ttt{busy}: received signal is not interpretable
  \item \ttt{synchronizing}: synchronization is in progress
  \item \ttt{receiving}: reception is in progress
\end{itemize}

When a radio wants to transmit a signal on the medium it sends direct messages
to all affected radios with the help of the central medium module. The messages
contain a shared data structure which describes the transmission the way it
entered the medium. The messages arrive at the moment when start of the
transmission arrive at the receiver. The receiver radios also handle the
incoming messages with the help of the central medium module. This kind of
centralization allows the medium to do shared computations in a more efficient
way and it also makes parallel computation possible.

As stated above the radio module utilizes multiple submodules to further split
its task. This design decision makes it more extensible and customizable. The
following sections describe the parts of the radio model.

\subsection{Antenna Models}

The antenna model describes the effects of the physical device which converts
electric signals into radio waves, and vice versa. This model captures the
antenna characteristics that heavily affect the quality of the communication
channel. For example, various antenna shapes, antenna size and geometry, antenna
arrays, and antenna orientation causes different directional or frequency
selectivity.

The antenna model provides a position and an orientation using a mobility model
that defaults to the mobility of the node. The main purpose of this model is to
compute the antenna gain based on the specific antenna characteristics and the
direction of the signal. The signal direction is computed by the medium from the
position and the orientation of the transmitter and the receiver. The following
list provides some examples:

\begin{itemize}
  \item \nedtype{IsotropicAntenna}: antenna gain is exactly 1 in any direction
  \item \nedtype{ConstantGainAntenna}: antenna gain is a constant determined by
a parameter
  \item \nedtype{DipoleAntenna}: antenna gain depends on the direction according
to the dipole antenna characteristics
  \item \nedtype{InterpolatingAntenna}: antenna gain is computed by linear
interpolation according to a table indexed by the direction angles
\end{itemize}

The antenna models are in the \ttt{src/physicallayer/antenna/} directory.

\subsection{Transmitter Models}

The transmitter model describes the physical process which converts packets into
electric signals. In other words, this model converts a MAC packet into a signal
that is transmitted on the medium. The conversion process and the representation
of the signal depends on the level of detail and the physical characteristics
of the implemented protocol.

In the flat model the transmitter model skips the symbol domain and the sample
domain representations, and it directly creates the analog domain representation.
The bit domain representation is reduced to the bit length of the packet and the
actual bits are ignored.

In the layered model the conversion process involves various processing steps
such as packet serialization, forward error correction encoding, scrambling,
interleaving, and modulation. This transmitter model requires much more
computation, but it produces accurate bit domain, symbol domain, and sample
domain representations.

The various protocol specific transmitter models are in the corresponding
directories.

\subsection{Receiver Models}

The receiver model describes the physical process which converts electric
signals into packets. In other words, this model converts a reception, along
with an interference computed by the medium model, into a MAC packet and a
reception indication. It also determines the following for each transmission: 

\begin{itemize}
  \item \ttt{is the reception possible or not}: based on the signal
characteristics such as reception power, carrier frequency, bandwidth, preamble
mode, modulation scheme
  \item \ttt{if the reception is possible, is reception attempted or not}: based
on the ongoing reception and the support of signal capturing
  \item \ttt{if the reception is attempted, is reception successful or not}:
based on the error model and the simulated part of the signal decoding
\end{itemize}

In the flat model the receiver model skips the sample domain, the symbol domain,
and the bit domain representations, and it directly creates the packet domain
representation by copying the packet from the transmission. It uses the error
model to decide if the reception is successful or not.

In the layered model the conversion process involves various processing steps
such as demodulation, descrambling, deinterleaving, forward error correction
decoding, and deserialization. This reception model requires much more
computation, but it produces accurate sample domain, symbol domain, and bit
domain representations.

The various protocol specific receiver models are in the corresponding 
directories.

\subsection{Error Models}

Determining the reception errors is a crucial part of the reception process.
There are often several different statistical error models in the literature
even for a particular physical layer. In order to support this diversity the
error model is a separate replaceable component of the receiver. 

The error model describes how the signal to noise ratio affects the amount of
errors at the receiver. The main purpose of this model is to determine whether
if the received packet has errors or not. It also computes various physical
layer indications for higher layers such as packet error rate, bit error rate,
and symbol error rate. For the layered reception model it needs to compute the
erroneous bits, symbols, or samples depending on the lowest simulated physical
domain where the real decoding starts. The error model is optional, if omitted
all receptions are considered successful.

The error models are in the \ttt{src/physicallayer/errormodel/} directory and
also in the corresponding protocol specific directories.

\subsection{Energy Consumer Models}

A substantial part of the energy consumption of communication devices comes from
transmitting and receiving signals. The energy consumer model describes how the
radio consumes energy depending on its activity. This model is optional, if
omitted energy consumption is ignored. The following list provides some examples:

\begin{itemize}
  \item \nedtype{StateBasedEnergyConsumer}: the constant power consumption is
determined by valid combinations of the radio mode, the transmitter state and
the receiver state
\end{itemize}

The energy consumer models are in the \ttt{src/physicallayer/energyconsumer/} directory.

\ifdraft
TODO: layered
\subsection{Layered Radio Model}

This module further splits the transmitter and receiver models to allow bit
precise communication modeling.

TODO: layered

The following sections describe the parts of the layered radio model.

\subsubsection{Encoding and Decoding}

This module describes how the packet domain signal representation is converted
into the bit domain, and vice versa.

TODO: layered

\subsubsection{Modulation and Demodulation}

This module describes how the bit domain signal representation is converted into
the symbol domain, and vice versa.

TODO: layered

\subsubsection{Pulse Shaping and Pulse Filtering}

This module describes how the symbol domain signal representation is converted
into the sample domain, and vice versa.

TODO: layered


\subsubsection{Digital Analog and Analog Digital Conversion}

This module describes how the sample domain signal representation is converted
into the analog domain, and vice versa.

TODO: layered
\fi

\section{The Medium Model}

The medium model describes the shared physical medium where communication takes
place. It keeps track of radios, noise sources, ongoing transmissions,
background noise, and other ongoing noises. The medium computes when, where and
how transmissions and noises arrive at receivers. It also efficiently provides
the set of interfering transmissions and noises for the receivers. It doesn't
send or handle messages on its own, it rather acts as a mediator between radios.

The medium model provides a couple of parameters to optimize for performance.
For example, the filter parameters control how the model determines the set of
affected radios when a new transmission enters the medium. The medium model
maintains the following global limits among the registered radios to support
various optimizations:

\begin{itemize}
  \item maximum speed
  \item maximum antenna gain
  \item maximum transmission power
  \item minimum interference power
  \item minimum reception power
  \item minimum interference time
  \item maximum transmission duration
  \item maximum detection range
  \item maximum interference range
  \item maximum communication range
  \item minimum and maximum mobility constraints
\end{itemize}

The medium module utilizes multiple submodules to further split its task. This
design makes it extensible and customizable. The following sections describe
the parts of the medium.

\subsection{Propagation Models}

Whenever the radio transmits a signal, it starts to propagate through space. The
transmitter might move during the transmission interval, and the receiver might
also move during both the propagation time and the reception interval. The
effect of these movements becomes more and more important as the propagation
speed and the speed of the radios gets closer to each other. This is especially
true for acoustic communication because the propagation speed of the signal is
much smaller. In general, it's difficult to accurately compute the signal
arrival, but it's usually not really necessary, some kind of approximation
suffices.

The propagation model is a separate component of the medium model to make it
extensible with alternative implementations. This model describes how signals
propagate through space over time. The main purpose of this model is to compute
the arrival space-time coordinates for transmissions. The following list
provides some examples:

\begin{itemize}
  \item \nedtype{ConstantTimePropagation}: propagation time is independent of
the traveled distance and it's determined by a constant parameter
  \item \nedtype{ConstantSpeedPropagation}: propagation time is proportional to
the traveled distance determined by a constant propagation speed parameter
\ifdraft
TODO: parallel
  \item \nedtype{ConstantSpeedGPUPropagation}: propagation time is computed in
parallel on the GPU for all receivers
\fi
\end{itemize}

The propagation models are in the \ttt{src/physicallayer/propagation/} directory.

\subsection{Path Loss Models}

As the signal propagates through space its power density decreases. Path loss
might be due to the combination of many effects, such as free-space loss,
refraction, diffraction, reflection, and absorption. There are several different
models in the literature, which differ in their parameterization and application
area.

The path loss model is a separate component of the medium model to make it
extensible with alternative implementations. This model describes the reduction
of power as the signal propagates through space. It computes the power loss
factor based on the traveled distance, the signal frequency and the propagation
speed. It may also provide the opposite, that is the traveled distance based on
the power loss factor, the signal frequency and the propagation speed. The
latter computation is useful for determining the maximum communication range
based on the transmission power and the reception sensitvity. The following list
provides some examples:

\begin{itemize}
  \item \nedtype{FreeSpacePathLoss}
  \item \nedtype{LogNormalShadowing}
  \item \nedtype{TwoRayGroundReflection}
  \item \nedtype{BreakpointPathLoss}
  \item \nedtype{NakagamiFading}
  \item \nedtype{RayleighFading}
  \item \nedtype{RicianFading}
  \item \nedtype{SuiPathLoss}
  \item \nedtype{UwbIrStochasticPathLoss}
  \item etc.
\end{itemize}

The path loss models are in the \ttt{src/physicallayer/pathloss/} directory.

\subsection{Obstacle Loss Models}

When the signal propagates through space it also passes through physical objects
present in that space. As the signal passes through, its power decreases when it
reflects from the surfaces of physical objects, and also when it's absorbed by
their material. There are various ways to model this effect, which differ in the
trade-off between accuracy and performance.

The obstacle loss model is a separate component of the medium model to make it
extensible with alternative implementations. This model describes the reduction
of power as the signal passes through physical objects. The main purpose of this
model is to compute the power loss factor based on the traveled straight path,
the signal frequency and the physical properties of the obstructing physical
objects. The obstacle model utilizes the physical environment model to query the
obstructing physical objects. The following list provides some examples:

\begin{itemize}
  \item \nedtype{TracingObstacleLoss}: the power loss is based on computing the
accurate dielectric and reflection loss along the straight path considering the
shape, the position, the orientation, and the material of obstructing physical
objects
\end{itemize}

The obstacle loss models are in the \ttt{src/physicallayer/obstacleloss/}
directory.

\ifdraft
TODO: multipath
\subsection{Multipath Models}

The signal reflects from the surfaces, it refracts as it passes through the
surfaces, it diffracts as it passes around the objects. 

The multipath model describes the alternative paths that the signal travels
before it either reaches the receiver or sufficiently fades.
\fi

\subsection{Background Noise Models}

The thermal noise, the cosmic background noise, and other random fluctuations of
the electromagnetic field affect the quality of the communication channel. This
kind of noise doesn't come from a particular source and it doesn't propagate
through space.

The background noise model is a separate component of the medium model to make
it extensible with alternative implementations. This model describes how the
background noise changes over space and time. The main purpose of this model is
to compute the analog representation of the noise signal for a given space-time
interval.

\begin{itemize}
  \item \nedtype{IsotropicBackgroundNoise}: the background noise is independent
of the time and the position, and its power is determined by a constant parameter 
\end{itemize}

The background noise models are in the \ttt{src/physicallayer/backgroundnoise/}
directory.

\subsection{Neighbor Cache Models}

Transceivers are considered neighbors if successful communication is possible
between them. In wired communication systems it's mostly quite obvious which
transceivers are neighbors, because they are connected by wires. In contrast,
in wireless communication systems determining which radios are neighbors isn't
obvious at all. 

The neighbor cache model is a separate component of the medium model to make it
extensible with alternative implementations. This model provides an efficient
way of keeping track of the neighbor relationship between radios. The main
purpose of this model is to compute the affected set of receivers on the medium
for a given transmission. The model also has to follow the movement of radios,
but it might provide conservative approximations for queries. The following list
provides some examples:

\begin{itemize}
  \item \nedtype{NeighborListNeighborCache}: maintains a separate periodically
updated  neighbor list for each radio
  \item \nedtype{GridNeighborCache}: organizes radios in a 3 dimensional grid
with constant cell size and updates periodically
  \item \nedtype{QuadTreeNeighborCache}: organizes radios in a 2 dimensional
quad tree (ignoring the Z axis) with constant node size and updates periodically
\end{itemize}

The neighbor cache models are in the \ttt{src/physicallayer/neighborcache/}
directory.

\section{Signal Representation}

The data structures that represent the transmitted and the received signals
might contain many different data depending on the simulated level of detail. In
addition, the reception data structure might contain various physical layer
indications, which are computed during the reception process. The following list
provides some examples:

\begin{itemize}
  \item \ttt{packet domain}: actual packet, packet error rate, packet error bit,
etc.
  \item \ttt{bit domain}: various bit lengths, bitrates, actual bits, forward
error correction code, interleaving scheme, scrambling scheme, bit error rate,
number of bit errors, actual erroneous bits, etc.
  \item \ttt{symbol domain}: number of symbols, symbol rate, actual symbols,
modulation scheme, symbol error rate, number of symbol errors, actual erroneous
symbols, etc.
  \item \ttt{sample domain}: number of samples, sampling rate, actual samples,
etc.
  \item \ttt{analog domain}: space-time coordinates, antenna orientations,
communication range, interference range, detection range, carrier frequency,
subcarrier frequencies, bandwidths, scalar or dimensional power, receive signal
strength indication, signal to noise and interference ratio, etc.
\end{itemize}

In simple case the packet domain specifies the MAC packet only, and the bit
domain specifies the bit length and the bitrate. The symbol domain specifies the
used modulation, and the sample domain is simply ignored. The most important
part is the analog domain representation, because it's indispensable to be able
to compute some kind of signal to noise and interference ratio. The following
figure shows four different kinds of analog domain representations, but other
representations are also possible.

\begin{figure}[h!]
\centering
\includegraphics[width=\textwidth]{figures/phyanalog}
\caption{Various analog signal representations}
\end{figure}

The first representation is called range-based, and it's used by the unit disc
radio. The advantage of this data structure is that it's compact, predictable,
and provides high performance. The disadvantage is that it's very inaccurate in
terms of modeling reality. Nevertheless, this representation might be sufficient
for developing a new routing protocol if accurate simulation of packet loss is
not important.

The second data structure represents a narrowband signal with a scalar signal
power, a carrier frequency, and a bandwidth. The advantage of this
representation is that it allows to compute a real signal to noise ratio, which
in turn can be used by the error model to compute bit and packet error rates.
This representation is most of the time sufficient for the simulation of IEEE
802.11 networks.

The third data structure describes a signal power that changes over time. In
this case the signal power is represented with a one-dimensional time dependent
value that precisely follows the transmitted pulses. This representation is used
by the IEEE 802.15.4a UWB radio.

The last representation uses a multi-dimensional value to describe the signal
power that changes over both time and frequency. The IEEE 802.11b model might
use this representation to simulate crosstalk, where one channel interferes with
another. In order to make it work the frequency spectrum of the signal has to
follow the real spectrum more precisely at both ends of the band.

The flat signal representation uses a single object to simulatenously describe
all domains of the transmission or the reception. In contrast, the layered
signal representation uses one object to describe every domain seperately. The
advantage of the latter is that it's extensible with alternative implementations
for each domain. The disadvantage is that it needs more allocation and resource
management.

\section{Signal Processing}

The following figure shows the process of how a MAC packet gets from the
transmitter radio through the medium to the receiver radio. The figure focues on
how data flows between the processing components of the physical layer. The blue
boxes represent the data structures, and the red boxes represent the processing
components.

\begin{figure}[h!]
\centering
\includegraphics[width=\textwidth]{figures/phydataflow}
\caption{Signal processing data flow}
\end{figure}

The transmission process starts in the transmitter radio when it receives a MAC
packet from the higher layer. The radio must be in transmitter or transceiver
mode before receiving a MAC packet, otherwise it throws an exception. At first
the transmitter model creates a data structure that describes the transmitted
signal based on the received MAC packet and the attached transmission request.
The resulting data structure is immutable, it's not going to be changed in any
later processing step.

Thereafter the propagation model computes the arrival space-time coordinates for
all receivers. In the next step the medium model determines the set of affected
receivers. Which radio constitutes affected depends on a number of factors such
as the maximum communication range of the transmitter, the radio mode of the
receiver, the listening mode of the receiver, or potentially the MAC address of
the receiver. Using the result the medium model sends a separate message with
the shared transmission data structure to all affected receivers. There's no
need to send a message to all radios on the channel, because the computation
of interfering signals is independent of this step.

Thereafter the attenuation model computes the reception for the receiver using
the original transmission and the arrival data structure. It applies the path
loss model, the obstacle loss model and the multipath model to the transmission.
The resulting data structure is also immutable, it's not going to be changed in
any later processing step.

Thereafter the medium model computes the interference for the reception by
collecting all interfering receptions and noises. Another signal is considered
interfering if it owerlaps both in time and frequency domains with respect to 
the minimum interference parameters. The background noise model also computes a 
noise signal that is added to the interference.

The reception process starts in the receiver radio when it receives a message
from the transmitter radio. The radio must be in receiver or transceiver mode
before the message arrives, otherwise it ignores the message. At first the
receiver model determines is whether the reception is actually attempted or not.
This decision depends on the reception power, whether there's another ongoing
reception process, and capturing is enabled or not.

Thereafter the receiver model computes the signal to noise and interference
ratio from the reception and the interference. Using the result, the bitrate,
and the modulation scheme the error model computes the necessary error rates.
Alternatively the error model might compute the erroneous bits, or symbols by
altering the corresponding data of the original transmission. 

Thereafter the receiver determines the received MAC packet by either simply
reusing the original, or actually decoding from the lowest represented domain
in the reception. Finally, it attaches the physical layer indication to the MAC
packet, and sends it up to the higher layer.

The following sections describe the data structures that are created during
signal processing.

\subsubsection{Transmission Request}

This data structure contains parameters that control how the transmitter
produces the transmission. For example, it might override the default
transmission power, ot the default bitrate of the transmitter. It is attached as
a control info object to the MAC packet sent down from the MAC module to the
radio.

\subsubsection{Transmission}

This data structure describes the transmission of a signal. It specifies the
start/end time, start/end antenna position, start/end antenna orientation of the
transmitter. In other words, it describes when, where and how the signal
started/ended to interact with the medium. The transmitter model creates one
transmission instance per MAC packet.

\subsubsection{Arrival}

This data structure decscirbes the space and time coordinates of a transmission
arriving at a particular receiver. It specifies the start/end time, start/end
antenna position, start/end antenna orientation of the receiver. The propagation
model creates one arrival instance per transmission per receiver.

\subsubsection{Listening}

This data structure describes the way the receiver radio is listening on the
medium. The physical layer ignores certain transmissions either during computing
the interference or even the complete reception of such transmissions. For
example, a narrowband listening specifies a carrier frequency and a bandwidth. 

\subsubsection{Reception}

This data structure describes the reception of a signal by a particular receiver.
It specifies at least the start/end time, start/end antenna position, start/end
antenna orientation of the receiver. The attenuation model creates one reception
instance per transmission per receiver.

\subsubsection{Noise}

This data structure describes a meaningless signal or a meaningless composition
of multiple signals. All models contain at least the start/end time, and
start/end position.

\subsubsection{Interference}

This data structure describes the interfering signals and noises that affect a
particular reception. It also specifies the total noise that is the composition
of all interference.

\subsubsection{SNIR}

This data structure describes the signal to noise and interference ratio of a
particular reception. It also specifies the minimum signal to noise and
interference ratio.

\subsubsection{Reception Decision}

This data structure describes whether if the reception of a signal is possible
or not, is attempted or not, and is successful or not.

\subsubsection{Reception Indication}

This data structure describes the physical layer indications such as RSSI, SNIR,
PER, BER, SER. These physical properties are optional and may be omitted if the
receiver is configured to do so or if it doesn't support providing the data. The
reception indication is attached as a control info object to the MAC packet sent
up from the radio to the MAC module. 

\section{Visualization}

In order to help understanding the communication in the network the physical
layer supports visualizing its state. The following list shows what can be
displayed:

\begin{itemize}
  \item ongoing transmissions
  \item recent successful receptions
  \item recent obstacle intersections and surface normal vectors
\end{itemize}

The ongoing transmissions can be displayed with 3 dimensional spheres or with 2
dimensional rings laying in the XY plane. As the signal propagates through space
the figure grows with it to show where the beginning of the signal is. The inner
circle of the ring figure shows as the end of the signal propagates through
space. 

The recent successful receptions are displayed as straight lines between the
original positions of the transmission and the reception. The recent obstacle
intersections are also displayed as straight lines from the start of the
intersection to the end of it.

\ifdraft
TODO: scalar vs dimensional
TODO: flat vs layered
TODO: Generic, IEEE 802.11, IEEE 802.15.4
TODO: acoustic underwater example
TODO: wireless vs. wired medium
\section{Use Cases}
\fi


%%% Local Variables:
%%% mode: latex
%%% TeX-master: "usman"
%%% End:


\cleardoublepage

\chapter{The 802.11 Model}
\label{cha:80211}


\section{Overview}

This chapter provides an overview of the IEEE 802.11 model for the INET Framework.

An IEEE 802.11 interface (NIC) comes in several flavours, differring
in their role (ad-hoc station, infrastructure mode station, or
access point) and their level of detail:

\begin{enumerate}
 \item \nedtype{Ieee80211Interface}: a generic (configurable) NIC
 \item \nedtype{Ieee80211NicAdhoc}: for ad-hoc mode
 \item \nedtype{Ieee80211NicAP}, \nedtype{Ieee80211NicAPSimplified}: for use in an access point
 \item \nedtype{Ieee80211NicSTA}, \nedtype{Ieee80211NicSTASimplified}: for use in an
   infrastructure-mode station
\end{enumerate}

NICs consist of four layers, which are the following (in top-down order):

\begin{enumerate}
  \item agent
  \item management
  \item MAC
  \item physical layer (radio)
\end{enumerate}

\textit{The physical layer} modules (\nedtype{Ieee80211Radio}) deal with modelling
transmission and reception of frames. They model the characteristics of
the radio channel, and determine if a frame was received correctly
(that is, it did not suffer bit errors due to low signal power or
interference in the radio channel). Frames received correctly are passed
up to the MAC. The implementation of these modules is based on the
Mobility Framework.

\textit{The MAC layer} (\nedtype{Ieee80211Mac}) performs transmission of frames according
to the CSMA/CA protocol. It receives data and management frames from
the upper layers, and transmits them.

\textit{The management layer} performs encapsulation and decapsulation of data packets
for the MAC, and exchanges management frames via the MAC with its peer
management entities in other STAs and APs. Beacon, Probe Request/Response,
Authentication, Association Request/Response etc frames are generated
and interpreted by management entities, and transmitted/received via
the MAC layer. During scanning, it is the management entity that periodically
switches channels, and collects information from received beacons and
probe responses.

The management layer has several implementations which differ in their role
(STA/AP/ad-hoc) and level of detail: \nedtype{Ieee80211MgmtAdhoc},
\nedtype{Ieee80211MgmtAp}, \nedtype{Ieee80211MgmtApSimplified}, \nedtype{Ieee80211MgmtSta},
\nedtype{Ieee80211MgmtStaSimplified}. The ..Simplified ones differ from the others
in that they do not model the scan-authenticate-associate process,
so they cannot be used in experiments involving handover.

\textit{The agent} is what instructs the management layer to perform
scanning, authentication and association. The management layer itself
just carries out these commands by performing the scanning, authentication
and association procedures, and reports back the results to the agent.

The agent layer is currenly only present in the \nedtype{Ieee80211NicSTA} NIC module,
as an \nedtype{Ieee80211AgentSta} module. The managament entities in other NIC
variants do not have as much freedom as to need an agent to control them.

By modifying or replacing the agent, one can alter the dynamic behaviour
of STAs in the network, for example implement different handover strategies.

\subsection{Limitations}

See the documentation of \nedtype{Ieee80211Mac} for features unsupported by this
model.

\ifdraft TODO
 further details about the implementation: what is modelled and what is
 not (beacons, auth, ...), communication between modules, frame formats,
 ...
\fi



%%% Local Variables:
%%% mode: latex
%%% TeX-master: "usman"
%%% End:


\cleardoublepage

\chapter{MAC Protocols for Wireless Sensor Networks}
\label{cha:sensor-macs}

\section{Overview}

The INET Framework contains the implementation of several MAC protocols
for wireless sensor networks (WSNs), including B-MAC, L-MAC and X-MAC.

To create a wireless node with a specific MAC protocol, use a node type 
that has a wireless interface, and set the interface type to the 
appropriate type. For example, \nedtype{WirelessHost} is a node type 
which is preconfigured to have one wireless interface, \ttt{wlan[0]}.
\ttt{wlan[0]} is of parametric type, so if you build the network from
\nedtype{WirelessHost} nodes, you can configure all of them to use
e.g. B-MAC with the following line in the ini file:

\begin{verbatim}
**.wlan[0].typename = "BMacInterface"
\end{verbatim}


\section{B-MAC}
\label{sec:bmac}

B-MAC (Berkeley MAC) is a carrier sense media access protocol for 
wireless sensor networks that provides a flexible interface to obtain
ultra low power operation, effective collision avoidance, and 
high channel utilization. To achieve low power operation, 
B-MAC employs an adaptive preamble sampling scheme to reduce duty cycle 
and minimize idle listening. B-MAC is designed for low traffic, 
low power communication, and is one of the most widely used 
protocols (e.g. it is part of TinyOS).

The \nedtype{BMac} module type implements the B-MAC protocol.

\nedtype{BMacInterface} is a \nedtype{WirelessInterface} with the MAC type
set to \nedtype{BMac}.


\section{L-MAC}
\label{sec:lmac}

L-MAC (Lightweight MAC) is an energy-efficient medium acces protocol designed 
for wireless sensor networks. Although the protocol uses TDMA to give nodes 
in the WSN the opportunity to communicate collision-free, the network is
self-organizing in terms of time slot assignment and synchronization. 
The protocol reduces the number of transceiver state switches and hence
the energy wasted in preamble transmissions.

The \nedtype{LMac} module type implements the L-MAC protocol, based on the
paper ``A lightweight medium access protocol (LMAC) for wireless sensor networks''
by van Hoesel and P. Havinga.

\nedtype{LMacInterface} is a \nedtype{WirelessInterface} with the MAC type
set to \nedtype{LMac}.


\section{X-MAC}
\label{sec:xmac}

X-MAC is a low-power MAC protocol for wireless sensor networks (WSNs).
In contrast to B-MAC which employs an extended preamble and preamble sampling,
X-MAC uses a shortened preamble that reduces latency at each hop and 
improves energy consumption while retaining the advantages 
of low power listening, namely low power communication, simplicity 
and a decoupling of transmitter and receiver sleep schedules.
 
The \nedtype{XMac} module type implements the X-MAC protocol, based on
the paper ``X-MAC: A Short Preamble MAC Protocol for Duty-Cycled 
Wireless Sensor Networks'' by Michael Buettner, Gary V. Yee, Eric Anderson
and Richard Han.

\nedtype{XMacInterface} is a \nedtype{WirelessInterface} with the MAC type
set to \nedtype{XMac}.


-----------

TODO other MACs: Acking, CSMA/CA, Shortcut, Loopback, TUN -- where to document?


%%% Local Variables:
%%% mode: latex
%%% TeX-master: "usman"
%%% End:


\cleardoublepage

% last updated to inet commit 'e10ce09097652b72f41fcbde5dc81a2cc2537d32'

\chapter{Node Mobility}
\label{cha:mobility}


\section{Overview}

In order to accurately evaluate a protocol for an ad-hoc network,
it is important to use a realistic model for the motion of mobile
hosts. Signal strengths, radio interference and channel occupancy
depends on the distances between nodes. The choice of the mobility
model can significantly influence the results of a simulation
(e.g. data packet delivery ratio, end-to-end delay, average hop count)
as shown in \cite{Camp02asurvey}.

There are two methods for incorporating mobility into simulations:
using traces and synthetic models. Traces contains recorded motion
of the mobile hosts, as observed in real life system. Synthetic models
use mathematical models for describing the behaviour of the mobile hosts.

There are mobility models that represent mobile nodes whose movements
are independent of each other (entity models) and mobility models
that represent mobile nodes whose movements are dependent on each other
(group models). Some of the most frequently used entity models are the
Random Walk Mobility Model, Random Waypoint Mobility Model, Random
Direction Mobility Model, Gauss-Markov Mobility Model, City Section
Mobility Model. The group models include the Column Mobility Model,
Nomadic Community Mobility Model, Pursue Mobility Model,
Reference Point Group Mobility Model.

The INET framework has components for the following trace files:

\begin{itemize}
\item \tbf{Bonn Motion} native file format of the
    \href{http://net.cs.uni-bonn.de/wg/cs/applications/bonnmotion/}{BonnMotion}
    scenario generation tool.
\item \tbf{Ns2} trace file generated by the CMU's scenario generator that used in Ns2.
\item \tbf{ANSim} XML trace file of the ANSim (Ad-Hoc Network Simulation) tool.
\end{itemize}

It is easy to integrate new entity mobility models into the INET framework,
but group mobility is not supported yet. Therefore all the models
shipped with INET are implementations of entitiy models:

\begin{itemize}
\item \tbf{Deterministic Motions} for fixed position nodes and nodes
      moving on a linear, circular, rectangular paths.
\item \tbf{Random Waypoint} model includes pause times between changes
      in destination and speed.
\item \tbf{Gauss-Markov} model uses one tuning parameter to vary the degree
      of randomness in mobility pattern.
\item \tbf{Mass Mobility} models a mass point with inertia and momentum.
\item \tbf{Chiang Mobility} uses a probabilistic transition matrix to change
      the state of motion of the node.
\end{itemize}

\section{Mobility in INET}

In INET mobile nodes have to contain a module implementing the
\nedtype{IMobility} marker interface. This module stores the current
coordinates of the node and is responsible for updating the position
periodically and emitting the \ttt{mobilityStateChanged} signal
when the position changed.

Many mobility models allow the user to define a cubic volume that the node 
can not leave. The volume is configured by setting the \fpar{constraintAreaX}, 
\fpar{constraintAreaY}, \fpar{constraintAreaZ},
\fpar{constraintAreaWidth}, \fpar{constraintAreaHeight} and
\fpar{constraintAreaDepth} parameters.

If the \fpar{initFromDisplayString} parameter, the initial position is taken from
the display string. Otherwise the position can be given as the \fpar{initialX},
\fpar{initialY} and \fpar{initialZ} parameters. If neither of these parameters
are given, a random initial position is choosen within the contraint area.

When the node reaches the boundary of the constraint area, the mobility
component has to prevent the node to exit. Many mobility models offer the 
following policies:

\begin{itemize}
  \item reflect of the wall
  \item reappear at the opposite edge (torus area)
  \item placed at a randomly chosen position of the area
  \item stop the simulation with an error
\end{itemize}

The p[0] and p[1] fields of the display string of the node is
also updated, so if the simulation run is animated, the node is
actually moving on the screen. The current position of the node
can be obtained from the display string.

The radio simulations has a \nedtype{ChannelControl} module that takes case of
establishing communication channels between nodes that are within
communication distance and tearing down the connections when they
move out of range. The \nedtype{ChannelControl} module uses to
\ttt{mobilityStateChanged} signal to determine when the connection
status needs to be updated.

There are two possibilities to implement a new mobility model. The simpler but
limited one is to use \nedtype{TurtleMobility} as the mobility component and to
write a script similar to the turtle graphics of LOGO. The second is to
implement a simple module in C++. In this case the C++ class of the mobility
module should be derived from \cppclass{IMobility} and its NED type should
implement the \nedtype{IMobility} interface.

%\begin{figure}
%\begin{center}
%\includegraphics{figures/mobility_classes}
%\end{center}
%\end{figure}

% FIXME The Z coordinate is often initialized randomly.
%       It would be better (backward compatibility)
%       to initialize them to 0. (contsraintAreaDepth=0 not handled everywhere)


\section{Implemented models}

\subsection{Deterministic movements}

\begin{description}

\item[StationaryMobility] This mobility module does nothing;
it can be used for stationary nodes.

\item[StaticGridMobility] Places all nodes in a rectangular grid.

% TODO it always creates an N x N grid, generalize

\item[LinearMobility] This is a linear mobility model with speed,
angle and acceleration parameters. Angle only changes when the mobile
node hits a wall: then it reflects off the wall at the same angle.

z coordinate is constant
movement is always parallel with X-Y plane

% TODO interpret 'angle' as asimuth and introduce inclination angle (default is 0)
%      to describe movement along arbitrary line segment
% FIXME why different speed and lastSpeed

\item[CircleMobility] Moves the node around a circle parallel to the X-Y plane
with constant speed.
The node bounces from the bounds of the constraint area.
The circle is given by the \fpar{cx}, \fpar{cy} and \fpar{r} parameters,
The initial position determined by the \fpar{startAngle} parameter.
The position of the node is refreshed in \fpar{updateInterval} steps.

\item[RectangleMobility] Moves the node around the constraint area.
configuration: speed, startPos, updateInterval
% should be derived from LineSegmentsMobilityBase?

\item[TractorMobility] Moves a tractor through a field with a certain
amount of rows. The following figure illustrates the movement of the
tractor when the \fpar{rowCount} parameter is 2. The trajectory follows
the segments in $1,2,3,4,5,6,7,8,1,2,3\ldots$ order. The area is configured
by the \fpar{x1}, \fpar{y1}, \fpar{x2}, \fpar{y2} parameters.

% TODO use constraint area instead of new x1,y1,x2,y2 parameters as in RectangleMobility

\begin{center}
\setlength{\unitlength}{0.5mm}
\begin{picture}(80,80)
\put(40,72){$1$} \put(10,70){\vector(1,0){30}} \put(10,70){\line(1,0){60}}
\put(72,55){$2$} \put(70,70){\vector(0,-1){15}} \put(70,70){\line(0,-1){30}}
\put(40,42){$3$} \put(70,40){\vector(-1,0){30}} \put(70,40){\line(-1,0){60}}
\put(5,25){$4$} \put(10,40){\vector(0,-1){15}} \put(10,40){\line(0,-1){30}}
\put(40,12){$5$} \put(10,10){\vector(1,0){30}} \put(10,10){\line(1,0){60}}
\put(72,25){$6$} \put(70,10){\vector(0,1){15}} \put(70,10){\line(0,1){30}}
\put(40, 33){$7$}
\put(5,55){$8$} \put(10,40){\vector(0,1){15}} \put(10,40){\line(0,1){30}}
\put(0,72){$(x_1,y_1)$} \put(65,2){$(x_2,y_2)$}
\end{picture}
\end{center}

\end{description}

\subsection{Random movements}

\begin{description}

\item[RandomWPMobility]

In the Random Waypoint mobility model the nodes move in line segments. For each
line segment, a random destination position (distributed uniformly over the
playground) and a random speed is chosen. You can define a speed as a variate
from which a new value will be drawn for each line segment; it is customary to
specify it as \ttt{uniform(minSpeed, maxSpeed)}. When the node reaches the
target position, it waits for the time \fpar{waitTime} which can also be defined as a
variate. After this time the the algorithm calculates a new random position, etc.

\item[GaussMarkovMobility] The Gauss-Markov model contains a tuning
parameter, that control the randomness in the movement of the node.
Let the magnitude and direction of speed of the node at the $n$th time step be
$s_n$ and $d_n$. The next speed and direction is computed as

$$ s_{n+1} = \alpha s_n + (1 - \alpha) \bar{s} +
             \sqrt{(1-\alpha^2)} s_{x_n} $$

$$ d_{n+1} = \alpha s_n + (1 - \alpha) \bar{d} +
             \sqrt{(1-\alpha^2)} d_{x_n} $$

where $\bar{s}$ and $\bar{d}$ are constants representing the mean value
of speed and direction as $n \to \infty$; and $s_{x_n}$ and $d_{x_n}$
are random variables with Gaussian distribution.

Totally random walk (Brownian motion) is obtained by setting $\alpha=0$,
while $\alpha=1$ results a linear motion.

To ensure that the node does not remain at the boundary of the constraint
area for a long time, the mean value of the direction ($\bar{d}$) modified
as the node enters the margin area. For example at the right edge of the
area it is set to 180 degrees, so the new direction is away from the edge.

% FIXME the GaussMarkovMobility module has only one variance parameter.
%       it should have separate speed and direction parameters

\item[MassMobility]

This is a random mobility model for a mobile host with
a mass. It is the one used in \cite{Perkins99optimizedsmooth}.

\begin{quote}
"An MH moves within the room according to the following pattern. It moves
along a straight line for a certain period of time before it makes a turn.
This moving period is a random number, normally distributed with average of
5 seconds and standard deviation of 0.1 second. When it makes a turn, the
new direction (angle) in which it will move is a normally distributed
random number with average equal to the previous direction and standard
deviation of 30 degrees. Its speed is also a normally distributed random
number, with a controlled average, ranging from 0.1 to 0.45 (unit/sec), and
standard deviation of 0.01 (unit/sec). A new such random number is picked
as its speed when it makes a turn. This pattern of mobility is intended to
model node movement during which the nodes have momentum, and thus do not
start, stop, or turn abruptly. When it hits a wall, it reflects off the
wall at the same angle; in our simulated world, there is little other
choice."
\end{quote}

This implementation can be parameterized a bit more, via the
\fpar{changeInterval}, \fpar{changeAngleBy} and \fpar{changeSpeedBy} parameters.
The parameters described above correspond to the following settings:

\begin{inifile}
changeInterval = normal(5, 0.1)
changeAngleBy = normal(0, 30)
speed = normal(avgSpeed, 0.01)
\end{inifile}

\item[ChiangMobility] Chiang's random walk movement model
(\cite{Chiang98wirelessnetwork}).

In this model, the state of the mobile node in each direction (x and y) can be:

\begin{itemize}
  \item 0: the node stays in its current position
  \item 1: the node moves forward
  \item 2: the node moves backward
\end{itemize}

The $(i,j)$ element of the state transition matrix determines the
probability that the state changes from $i$ to $j$:

$$ \left(
\begin{array}{ccc}
  0 & 0.5 & 0.5 \\
  0.3 & 0.7 & 0 \\
  0.3 & 0 & 0.7
\end{array}
\right) $$

The \nedtype{ChiangMobility} module supports the following parameters:
\begin{itemize}
  \item \fpar{updateInterval} position update interval
  \item \fpar{stateTransitionInterval} state update interval
  \item \fpar{speed}: the speed of the node
\end{itemize}


% FIXME last line of setTargetPosition() contains a sign error, should be
%              targetPosition = lastPosition + lastSpeed * stateTransitionUpdateInterval;
% FIXME when reflecting at the boundary, state variables should be reflected too

\item[ConstSpeedMobility]

\nedtype{ConstSpeedMobility} does not use one of the standard mobility
approaches. The user can define a velocity for each Host and an update interval. If
the velocity is greater than zero (i.e. the Host is not stationary) the
\nedtype{ConstSpeedMobility} module calculates a random target position for the
Host. Depending to the update interval and the velocity it calculates the number of
steps to reach the destination and the step-size. Every update interval
\nedtype{ConstSpeedMobility} calculates the new position on its way to the
target position and updates the display. Once the target position is reached
\nedtype{ConstSpeedMobility} calculates a new target position.

This component has been taken over from Mobility Framework 1.0a5.

% FIXME this is a special case of RandomWPMobility, remove

\end{description}

\subsection{Replaying trace files}

\begin{description}

\item[BonnMotionMobility] Uses the native file format of
\href{http://www.cs.uni-bonn.de/IV/BonnMotion/}{BonnMotion}.

The file is a plain text file, where every line describes the motion
of one host. A line consists of one or more (t, x, y) triplets of real
numbers, like:

\begin{verbatim}
t1 x1 y1 t2 x2 y2 t3 x3 y3 t4 x4 y4 ...
\end{verbatim}

The meaning is that the given node gets to $(xk,yk)$ at $tk$. There's no
separate notation for wait, so x and y coordinates will be repeated there.

\item[Ns2Mobility] Nodes are moving according to the trace files used
in NS2.
The trace file has this format:

\begin{verbatim}
# '#' starts a comment, ends at the end of line
$node_(<id>) set X_ <x> # sets x coordinate of the node identified by <id>
$node_(<id>) set Y_ <y> # sets y coordinate of the node identified by <id>
$node_(<id>) set Z_ <z> # sets z coordinate (ignored)
$ns at $time "$node_(<id>) setdest <x> <y> <speed>" # at $time start moving
towards <x>,<y> with <speed>
\end{verbatim}

The \nedtype{Ns2MotionMobility} module has the following parameters:

\begin{itemize}
  \item \fpar{traceFile} the Ns2 trace file
  \item \fpar{nodeId} node identifier in the trace file; -1 gets substituted by
  parent module's index
  \item \fpar{scrollX},\fpar{scrollY} user specified translation of the
  coordinates
\end{itemize}

% TODO cleaning the code (e.g. duplicated bounds check in setTargetPosition())
% TODO implement cached file access as in BonnMotionMobility

\item[ANSimMobility] reads trace files of the \href{http://www.ansim.info}{ANSim} Tool.

The nodes are moving along linear segments described by an XML trace file
conforming to this DTD:

\begin{verbatim}
<!ELEMENT mobility (position_change*)>
<!ELEMENT position_change (node_id, start_time, end_time, destination)>
<!ELEMENT node_id (#PCDATA)>
<!ELEMENT start_time (#PCDATA)>
<!ELEMENT end_time (#PCDATA)>
<!ELEMENT destination (xpos, ypos)>
<!ELEMENT xpos (#PCDATA)>
<!ELEMENT ypos (#PCDATA)>
\end{verbatim}

Parameters of the module:

\begin{itemize}
  \item \fpar{ansimTrace} the trace file
  \item \fpar{nodeId} the \verb!node_id! of this node, -1 gets substituted to
  parent module's index
\end{itemize}

\begin{note}
The \nedtype{AnsimMobility} module process only the \ttt{position\_{}change}
elements and it ignores the \ttt{start\_{}time} attribute. It starts the move
on the next segment immediately.
\end{note}


\end{description}




\section{Mobility scripts}

The \nedtype{TurtleMobility} module can be parametrized by a script file
containing LOGO-style movement commands in XML format.

The module has these parameters:

\begin{itemize}
\item \ttt{updateInterval} time interval to update the hosts position
\item \ttt{constraintAreaX}, \ttt{constraintAreaY}, \ttt{constraintAreaWidth},
      \ttt{constraintAreaHeight}: constraint area that the node can not leave
\item \ttt{turtleScript} XML file describing the movements
\end{itemize}

The content of the XML file should conform to the following DTD (can be
found as \ffilename{TurtleMobility.dtd} in the source tree):

\begin{verbatim}
<!ELEMENT movements (movement)*>

<!ELEMENT movement (repeat|set|forward|turn|wait|moveto|moveby)*>
<!ATTLIST movement id NMTOKEN #IMPLIED>

<!ELEMENT repeat (repeat|set|forward|turn|wait|moveto|moveby)*>
<!ATTLIST repeat n CDATA #IMPLIED>

<!ELEMENT set EMPTY>
<!ATTLIST set x     CDATA #IMPLIED
              y     CDATA #IMPLIED
              speed CDATA #IMPLIED
              angle CDATA #IMPLIED
              borderPolicy (reflect|wrap|placerandomly|error) #IMPLIED>

<!ELEMENT forward EMPTY>
<!ATTLIST forward d CDATA #IMPLIED
                  t CDATA #IMPLIED>

<!ELEMENT turn EMPTY>
<!ATTLIST turn angle CDATA #REQUIRED>

<!ELEMENT wait EMPTY>
<!ATTLIST wait t CDATA #REQUIRED>

<!ELEMENT moveto EMPTY>
<!ATTLIST moveto x CDATA #IMPLIED
                 y CDATA #IMPLIED
                 t CDATA #IMPLIED>

<!ELEMENT moveby EMPTY>
<!ATTLIST moveby x CDATA #IMPLIED
                 y CDATA #IMPLIED
                 t CDATA #IMPLIED>
\end{verbatim}

The file contains \ttt{movement} elements, each describing a trajectory.
The \ttt{id} attribute of the \ttt{movement} element can be used to
refer the movement from the ini file using the syntax:

\begin{inifile}
**.mobility.turtleScript = xmldoc("turtle.xml", "movements//movement[@id='1']")
\end{inifile}

The motion of the node is composed of uniform linear segments.
The state of motion is described by the following variables:
\begin{itemize}
\item \ttt{position}: $(x,y)$ coordinate of the current location of the node
\item \ttt{speed}, \ttt{angle}: magnitude and direction of the node's velocity
\item \ttt{targetPos}: target position of the current line segment. If given
                       the \ttt{speed} and \ttt{angle} is not used
\item \ttt{targetTime} the end time of the current linear motion
\item \ttt{borderPolicy}: one of
    \begin{itemize}
      \item \ttt{reflect} the node reflects at the boundary,
      \item \ttt{wrap} the node appears at the other side of the area,
      \item \ttt{placerandomly} the node placed at a random position of the area,
      \item \ttt{error} signals an error when the node reaches the boundary
    \end{itemize}
\end{itemize}

The \ttt{movement} elements may contain the the following commands:

\begin{itemize}
\item \ttt{repeat(n)} repeats its content n times, or indefinetly if the n attribute
              is omitted.
\item \ttt{set(x,y,speed,angle,borderPolicy)} modifies the state of the node.
\item \ttt{forward(d,t)} moves the node for $t$ time or to the $d$ distance
with the current speed. If both $d$ and $t$ is given, then the current
speed is ignored.
\item \ttt{turn(angle)} increase the angle of the node by $angle$ degrees.
\item \ttt{moveto(x,y,t)} moves to point $(x,y)$ in the given time. If
$t$ is not specified, it is computed from the current speed.
\item \ttt{moveby(x,y,t)} moves by offset $(x,y)$ in the given time. If
$t$ is not specified, it is computed from the current speed.
\item \ttt{wait(t)} waits for the specified amount of time.
\end{itemize}

Attribute values must be given without physical units, distances are assumed
to be given as meters, time intervals in seconds and speeds in meter per seconds.
Attibutes can contain expressions that are evaluated each time the
command is executed. The limits of the constraint area can be
referenced as \verb!$MINX!, \verb!$MAXX!, \verb!$MINY!, and \verb!$MAXY!.
Random number distibutions generate a new random number when evaluated,
so the script can describe random as well as deterministic scenarios.

To illustrate the usage of the module, we show how some mobility
models can be implemented as scripts:

\begin{itemize}
\item RectangleMobility:

\begin{verbatim}
    <movement>
        <set x="$MINX" y="$MINY" angle="0" speed="10"/>
        <repeat>
            <repeat n="2">
                <forward d="$MAXX-$MINX"/>
                <turn angle="90"/>
                <forward d="$MAXY-$MINY"/>
                <turn angle="90"/>
            </repeat>
        </repeat>
    </movement>
\end{verbatim}

\item Random Waypoint:

\begin{verbatim}
    <movement>
        <repeat>
            <set speed="uniform(20,60)"/>
            <moveto x="uniform($MINX,$MAXX)" y="uniform($MINY,$MAXY)"/>
            <wait t="uniform(5,10)">
        </repeat>
    </movement>
\end{verbatim}

\item MassMobility:

\begin{verbatim}
    <movement>
        <repeat>
            <set speed="uniform(10,20)"/>
            <turn angle="uniform(-30,30)"/>
            <forward t="uniform(0.1,1)"/>
        </repeat>
    </movement>
\end{verbatim}

\end{itemize}


%%% Local Variables:
%%% mode: latex
%%% TeX-master: "usman"
%%% End:



\cleardoublepage

\chapter{The IPv4 Protocol Family}
\label{cha:ipv4}


\section{Overview}

The IP protocol is the workhorse protocol of the TCP/IP protocol suite.
All UDP, TCP, ICMP packets are encapsulated into IP datagrams and
transported by the IP layer.
While higher layer protocols transfer data among two communication end-point,
the IP layer provides an hop-by-hop, unreliable and connectionless delivery
service. IP does not maintain any state information about the individual
datagrams, each datagram handled independently.

The nodes that are connected to the Internet can be either a host or a router.
The hosts can send and recieve IP datagrams, and their operating system
implements the full TCP/IP stack including the transport layer. On the
other hand, routers have more than one interface cards and perform packet
routing between the connected networks. Routers does not need the
transport layer, they work on the IP level only. The division
between routers and hosts is not strict, because if a host
have several interfaces, they can usually be configured to operate
as a router too.

Each node on the Internet has a unique IP address. IP datagrams contain
the IP address of the destination. The task of the routers is to find
out the IP address of the next hop on the local network, and forward
the packet to it. Sometimes the datagram is larger, than the maximum
datagram that can be sent on the link (e.g. Ethernet has an 1500 bytes limit.).
In this case the datagram is split into fragments and each fragment is
transmitted independently. The destination host must collect all fragments,
and assemble the datagram, before sending up the data to the transport
layer.

\subsection{INET modules}

The INET framework contains several modules to build the
IPv4 network layer of hosts and routers:
\begin{itemize}
  \item \nedtype{Ipv4} is the main module that implements RFC791. This
        module performs IP encapsulation/decapsulation, fragmentation
        and assembly, and routing of IP datagrams.
  \item The \nedtype{Ipv4RoutingTable} is a helper module that manages the routing
        table of the node. It is queried by the \nedtype{Ipv4} module
        for best routes, and updated by the routing daemons implementing
        RIP, OSPF, Manet, etc. protocols.
  \item The \nedtype{Icmp} module can be used to generate ICMP error packets. It also
        supports ICMP echo applications.
  \item The \nedtype{Arp} module performs the dynamic translation of IP addresses
        to MAC addresses.
  \item The \nedtype{Igmpv2} module to generate and process multicast group
        membership reports.
\end{itemize}

These modules are assembled into a complete network layer module
called \nedtype{Ipv4NetworkLayer}. This module has
dedicated gates for TCP, UDP, SCTP, RSVP, OSPF, Manet, and Ping
higher layer protocols. It can be connected to several network
interface cards: Ethernet, PPP, Wlan, or external interfaces.
The \nedtype{Ipv4NetworkLayer} module is used to build IPv4 hosts
(\nedtype{StandardHost}) and routers (\nedtype{Router}).

The implementation of these modules are based on the following RFCs:
\begin{itemize}
  \item RFC791: Internet Protocol
  \item RFC792: Internet Control Message Protocol
  \item RFC826: Address Resolution Protocol
  \item RFC1122: Requirements for Internet Hosts - Communication Layers
  \item RFC2236: Internet Group Management Protocol, Version 2
\end{itemize}

The subsequent sections describe the IPv4 modules in detail.

\section{The IPv4 Module}

The \nedtype{Ipv4} module implements the IPv4 protocol.

For connecting the upper layer protocols the \nedtype{Ipv4} module
has \emph{transportIn[]} and \emph{transportOut[]} gate vectors.

The IP packets are sent to the \nedtype{Arp} module through the
\emph{queueOut} gate. The incoming IP packets are received
directly from the network interface cards through the
\emph{queueIn[]} gates. Each interface card knows its own
network layer gate index.


\subsection{IP packets}

IP datagrams start with a variable length IP header.
The minimum length of the header is 20 bytes, and
it can contain at most 40 bytes for options, so
the maximum length of the IP header is 60 bytes.

\begin{center}
\begin{bytefield}{32}
\bitheader{0,3,4,7,8,15,16,18,19,23,24,31} \\
\bitbox{4}{Version} &
\bitbox{4}{IHL} &
\bitbox{8}{\small Type of Service} &
\bitbox{16}{Total Length} \\
\bitbox{16}{Identification} &
\bitbox{3}{Flags} &
\bitbox{13}{Fragment Offset} \\
\bitbox{8}{Time to Live} &
\bitbox{8}{Protocol} &
\bitbox{16}{Header Checksum} \\
\bitbox{32}{Source Address} \\
\bitbox{32}{Destination Address} \\
\bitbox{24}{Options} &
\bitbox{8}{Padding} \\
\end{bytefield}
\end{center}

The \ttt{Version} field is 4 for IPv4. The 4-bit \ttt{IHL} field is the
number of 32-bit words in the header. It is needed because the header
may contain optional fields, so its length may vary. The minimum IP header
length is 20, the maximum length is 60. The header is always padded to
multiple of 4 bytes. The \ttt{Type of Service} field designed to store
priority and preference values of the IP packet, so applications can
request low delay, high throughput, and maximium reliability from the
routing algorithms. In reality these fields are rarely set by applications,
and the routers mostly ignore them. The \ttt{Total Length} field is the
length of the whole datagram in bytes. The \ttt{Identification} field
is used for identifying the datagram sent by a host. It is usually generated
by incrementing a counter for each outgoing datagram. When the datagram
gets fragmented by a router, its \ttt{Identification} field is kept unchanged
to the other end can collect them. In datagram fragments the \ttt{Fragment Offset}
is the address of the fragment in the payload of the original datagram. It is
measured in 8-byte units, so fragment lengths must be a multiple of 8.
Each fragment except the last one, has its \ttt{MF} (more fragments) bit set
in the \ttt{Flags} field. The other used flag in \ttt{Flags} is the \ttt{DF}
(don't fragment) bit which forbids the fragmentation of the datagram.
The \ttt{Time to Live} field is decremented by each router in the path,
and the datagram is dropped if it reached 0. Its purpose is to prevent
endless cycles if the routing tables are not properly configured, but
can be used for limiting hop count range of the datagram (e.g. for local
broadcasts, but the \fprog{traceroute} program uses this field too).
The \ttt{Protocol} field is for demultiplexing the payload of the IP
datagram to higher level protocols. Each transport protocol has a registered
protocol identifier. The \ttt{Header Checksum} field is the 16-bit one's
complement sum of the header fields considered as a sequence of 16-bit numbers.
The \ttt{Source Address} and \ttt{Destination Address} are the IPv4 addresses
of the source and destination respectively.

The \ttt{Options} field contains 0 or more IP options. It is always padded
with zeros to a 32-bit boundary. An option is either a single-byte option
code or an option code + option length followed by the actual values for
the option. Thus IP implementations can skip unknown options.

An IP datagram is represented by the \msgtype{IPv4Datagram} message class.
It contains variables corresponding the fields of the IP header, except:
\begin{itemize}
  \item \fvar{Header Checksum} omitted, modeled by error bit of packets
  \item \fvar{Options} only the following options are permitted and the
                       datagram can contain at most one option:
        \begin{itemize}
          \item Loose Source Routing
          \item Strict Source Routing
          \item Timestamp
          \item Record Route
        \end{itemize}
\end{itemize}

The \fvar{Type of Service} field is called \ttt{diffServCodePoint} in
\nedtype{IPv4Datagram}.

Before sending the \msgtype{IPv4Datagram} through the network, the \nedtype{Ipv4}
module attaches a \cppclass{IPv4RoutingDecision} control info.
The control info contains the IP address of the next hop, and the
identifier of the interface it should be sent. The ARP module translate
the IP address to the hardware address on the local net of the specified
interface and forwards the datagram to the interface card.


\subsection{Parameters}

The \nedtype{Ipv4} module has the following parameters:
\begin{itemize}
  \item \fpar{procDelay} processing time of each incoming datagram.
  \item \fpar{timeToLive} default TTL of unicast datagrams.
  \item \fpar{multicastTimeToLive} default TTL of multicast datagrams.
  \item \fpar{protocolMapping} string value containing the \ttt{protocol id}
        $\rightarrow$ \ttt{gate index} mapping, e.g. \ttt{``6:0,17:1,1:2,2:3,46:4''}.
  \item \fpar{fragmentTimeout} the maximum duration until fragments are kept
          in the fragment buffer.
  \item \fpar{forceBroadcast} if \fkeyword{true}, then link-local broadcast
          datagrams are sent out through each interface, if the higher
          layer did not specify the outgoing interface.
\end{itemize}

% compile time options: WITH\_MANET, NEWFRAGMENT

\subsection{Statistics}

The \nedtype{Ipv4} module does not write any statistics into files,
but it has some statistical information that can be watched during
the simulation in the gui environment.
\begin{itemize}
  \item \ttt{numForwarded}: number of forwarded datagrams, i.e. sent to one of the
        interfaces (not broadcast), counted before fragmentation.
  \item \ttt{numLocalDeliver}: number of datagrams locally delivered.
        (Each fragment counted separately.)
  \item \ttt{numMulticast}: number of routed multicast datagrams.
  \item \ttt{numDropped} number of dropped packets.
        Either because there is no any interface, the interface is not specified and
        no \fpar{forceBroadcast}, or received from the network but IP forwarding disabled.
  \item \ttt{numUnroutable}: number of unroutable datagrams, i.e. there is no
        route to the destination. (But if outgoing interface is specified it is routed!)
\end{itemize}

In the graphical interface the bubble of the \nedtype{Ipv4} module
also displays these counters.


\section{The IPv4RoutingTable module}

The \nedtype{Ipv4RoutingTable} module represents the routing table.
IP hosts and routers contain one instance of this class. It has
methods to manage the routing table and the interface table,
so one can achieve functionality similar to the \fprog{route} and
\fprog{ifconfig} commands.

This is a simple module without gates, it requires function calls to it
(message handling does nothing). Methods are provided for reading and
updating the interface table and the route table, as well as for unicast
and multicast routing.

\subsection*{Parameters}

The \nedtype{Ipv4RoutingTable} module has the following parameters:

\begin{itemize}
  \item \fpar{routerId}: for routers, the router id using IPv4 address dotted notation;
        specify ``auto'' to select the highest interface address; should be left empty ``''
        for hosts
  \item \fpar{IPForward}: turns IP forwarding on/off (It is always \fkeyword{true}
                          in a \nedtype{Router} and is \fkeyword{false} by default
                          in a \nedtype{StandardHost}.)
  \item \fpar{forwardMulticast}: turns multicast IP forwarding on/off. Default is \fkeyword{false},
  should be set to \fkeyword{true} in multicast routers.
  \item \fpar{routingFile}: name of routing file that configures IP addresses and routes of the node
  containing this routing table. Its format is described in section \ref{subsec:routing_files}.
\end{itemize}

\begin{warning}
The \fpar{routingFile} parameter is obsolete. The prefered method for network configuration
is to use \nedtype{Ipv4NetworkConfigurator}. The old config files should be replaced
with the XML configuration of \nedtype{Ipv4NetworkConfigurator}. Section \ref{subsec:ipv4configurator}
describes the format of the new configuration files.
\end{warning}

% FIXME (#467) IPv4RoutingTable::invalidateCache() should clear localBroadcastAddresses.
% IPv4RoutingTable::findBestMatchingRoute() should search in this order:
%          1. host routes (exact match)
%          2. network routes (longest match)
%          3. default routes (round robin)
%   It is ok, if host routes has 255.255.255.255 netmask, and default has 0.0.0.0 netmask.
% FIXME IPv4RoutingTable::findBestMatchingRoute() if(...MANET...) branch always set bestRoute to NULL,
%       because if there were exact match, it would have been choosen in the previous loop.

\section{The ICMP module}

The Internet Control Message Protocol (ICMP) is the error reporting and
diagnostic mechanism of the Internet.
It uses the services of IP, so it is a transport layer protocol, but unlike
TCP or UDP it is not used to transfer user data. It can not be separated
from the IP, because the routing errors are reported by ICMP.

The \nedtype{Icmp} module can be used to send error messages and ping
request. It can also respond to incoming ICMP messages.

Each ICMP message is encapsulated within an IP datagram, so its delivery
is unreliable.

\begin{center}
\begin{bytefield}{32}
\bitheader{0,7,8,15,31} \\
\bitbox{8}{Type} &
\bitbox{8}{Code} &
\bitbox{16}{Checksum} \\
\bitbox{32}{Rest of header} \\
\wordbox{2}{Internet Header + 8 bytes of Original Datagram}
\end{bytefield}
\end{center}

The corresponding message class (\msgtype{ICMPMessage}) contains only
the Type and Code fields. The message encapsulates the IP packet that
triggered the error, or the data of the ping request/reply.

% FIXME type=PARAMETER_PROBLEM, code=0: missing Pointer field from ICMPMessage
%            REDIRECT: Gateway Internet Address
%            ECHO_REQUEST, ECHO_REPLY: Identifier, Sequence Number
%            TIMESTAMP_REQUEST, TIMESTAMP_REPLY: Identifier, Sequence Number, Originate Timestamp, Receive Timestamp, Transmit Timestamp

% FIXME wrong type codes for ICMP_DESTINATION_UNREACHABLE (3), ICMP_ECHO_REQUEST (8), ICMP_ECHO_REPLY (0), ICMP_TIMESTAMP_REQUEST (13), ICMP_TIMESTAMP_REPLY (14)

% FIXME ICMP header serializer handles only ICMP_ECHO_REQUEST, ICMP_ECHO_REPLY, ICMP_DESTINATION_UNREACHABLE, ICMP_TIME_EXCEEDED
%       ICMP header deserializer handles only ICMP_ECHO_REQUEST, ICMP_ECHO_REPLY

% FIXME ICMP error should not be send if the original datagram
%         1. is an ICMP error
%         2. was sent to a broadcast or multicast address
%         3. datagram was sent with a link-layer broadcast
%         4. a fragment other than the first
%         5. a datagram whose source address is 0.0.0.0, 127.*.*.*, broadcast or multicast address
%      currently only the 1. and half of 2. checked


% \section{The IGMP module}

\section{The ARP module}

The \nedtype{Arp} module implements the Address Resolution Protocol (RFC826).
The ARP protocol is designed to translate a local protocol address
to a hardware address. Altough the ARP protocol can be used with
several network protocol and hardware addressing schemes, in practice
they are almost always IPv4 and 802.3 addresses. The INET implementation
of the ARP protocol (the \nedtype{Arp} module) supports only
IP address $\rightarrow$ MAC address translation.

If a node wants to send an IP packet to a node whose MAC address is unknown,
it broadcasts an ARP frame on the Ethernet network.
In the request its publish its own IP and
MAC addresses, so each node in the local subnet can update their mapping.
The node whose MAC address was requested will respond with an ARP frame
containing its own MAC address directly to the node that sent the
request. When the original node receives the ARP response, it updates
its ARP cache and sends the delayed IP packet using the learned MAC address.

The frame format of the ARP request and reponse is shown in Figure \ref{fig:ARP_frame}.
In our case the HTYPE (hardware type), PTYPE (protocol type), HLEN (hardware address length)
and PLEN (protocol address length) are constants: HTYPE=Ethernet (1), PTYPE=IPv4 (2048), HLEN=6,
PLEN=4. The OPER (operation) field is 1 for an ARP request and 2 for an ARP response.
The SHA field contains the 48-bit hardware address of the sender, SPA field is
the 32-bit IP address of the sender; THA and TPA are the addresses of the target.
The message class corresponding to the ARP frame is \msgtype{ArpPacket}.
In this class only the OPER, SHA, SPA, THA and TPA fields are stored.
The length of an \msgtype{ArpPacket} is 28 bytes.

\begin{figure}[h]
\begin{center}
\label{fig:ARP_frame}
\begin{bytefield}{16}
\bitheader{0,7,8,15} \\
\bitbox{16}{HTYPE} \\
\bitbox{16}{PTYPE} \\
\bitbox{8}{HLEN} &
\bitbox{8}{PLEN} \\
\bitbox{16}{OPER} \\
\wordbox{3}{SHA} \\
\wordbox{2}{SPA} \\
\wordbox{3}{THA} \\
\wordbox{2}{TPA} \\
\end{bytefield}
\caption{ARP frame}
\end{center}
\end{figure}

The \nedtype{Arp} module receives IP datagrams and ARP responses from \nedtype{Ipv4}
on the \ttt{ipIn} gate and transmits IP datagrams and ARP requests on the \ttt{nicOut[]} gates
towards the network interface cards. ARP broadcasts the requests on the local network,
so the NIC's entry in the \nedtype{InterfaceTable} should have \ffunc{isBroadcast()} flag
set in order to participate in the address resolution.

The incoming IP packet should have an attached \cppclass{IPv4RoutingDecision} control
info containing the IP address of the next hop. The next hop can be either an
IPv4 broadcast/multicast or a unicast address. The corresponding MAC addresses
can be computed for broadcast and multicast addresses (RFC 1122, 6.4); unicast
addresses are mapped using the ARP procotol.

If the hardware address is found
in the ARP cache, then the packet is transmitted to the addressed interface immediately.
Otherwise the packet is queued and an address resolution takes place.
The \nedtype{Arp} module creates an \msgtype{ArpPacket} object, sets the sender
MAC and IP address to its own address, sets the destination IP address
to the address of the target of the IP datagram, leave the destination MAC address
blank and broadcasts the packet on each network interface with broadcast capability.
Before sending the ARP packet, it retransmission a timer. If the timer expires,
it will retransmit the ARP request, until the maximum retry count is reached.
If there is no response to the ARP request, then the address resolution fails,
and the IP packet is dropped from the queue. Otherwise the MAC address of the
destination is learned and the IP packet can be transmitted on the corresponding
interface.

When an ARP packet is received on the \ttt{ipIn} gate, and the sender's IP
is already in the ARP cache, it is updated with the information in the ARP frame.
Then it is checked that the destination IP of the packet matches with our
address. In this case a new entry is created with the sender addresses in the
ARP cache, and if the packet is a request a response is created and sent directly
to the originator. If proxy ARP is enabled, the request can be responded
with our MAC address if we can route IP packets to the destination.

Usually each \nedtype{Arp} module maintains a local ARP cache.
However it is possible to use a global cache. The global cache is filled
in with entries of the IP and MAC addresses of the known interfaces
when the ARP modules are initiated (at simulation time 0).
\nedtype{Arp} modules that are using the global ARP cache
never initiate an address resolution; if an IP address not
found in the global cache, the simulation stops with an error.
However they will respond to ARP request, so the simulation can
be configured so that some \nedtype{Arp}s use local, while others
the global cache.

When an entry is inserted or updated in the local ARP cache,
the simulation time saved in the entry. The mapping in the
entry is not used after the configured \fpar{cacheTimeout}
elapsed. This parameter does not affect the entries of
the global cache however.

The module parameters of \nedtype{Arp} are:

\begin{itemize}
  \item \fpar{retryTimeout}: number of seconds ARP waits between retries to resolve an IPv4 address (default is 1s)
  \item \fpar{retryCount}: number of times ARP will attempt to resolve an IPv4 address (default is 3)
  \item \fpar{cacheTimeout}: number of seconds unused entries in the cache will time out (default is 120s)
  \item \fpar{proxyARP}: enables proxy ARP mode (default is \fkeyword{true})
  \item \fpar{globalARP}: use global ARP cache (default is \fkeyword{false})
\end{itemize}

The \nedtype{Arp} module emits four signals:

\begin{itemize}
  \item \ttt{sentReq}: emits 1 each time an ARP request is sent
  \item \ttt{sentReplies}: emits 1 each time an ARP response is sent
  \item \ttt{initiatedResolution}: emits 1 each time an ARP resolution is initiated
  \item \ttt{failedResolution}: emits 1 each time an ARP resolution is failed
\end{itemize}

These signals are recorded as vectors and their counts as scalars.

% TODO watches, animation effects

\section{The IGMP module}

The IGMP module is responsible for distributing the information of
multicast group memberships from hosts to routers. When an interface
of a host joins to a multicast group, it will send an IGMP report
on that interface to routers. It can also send reports when the
interface leaves the multicast group, so it does not want to
receive those multicast datagrams. The IGMP module of multicast
routers processes these IGMP reports: it updates the list of
groups, that has members on the link of the incoming message.

The \nedtype{IIgmp} module interface defines the connections
of IGMP modules.
IGMP reports are transmitted by IP, so the module contains
gates to be connected to the IP module (\ttt{ipIn/ipOut}). The IP
module delivers packets with protocol number 2 to the IGMP module.
However some multicast routing protocols (like DVMRP) also exchange
routing information by sending IGMP messages, so they should be
connected to the \ttt{routerIn/routerOut} gates of the IGMP module.
The IGMP module delivers the IGMP messages not processed by itself
to the connected routing module.

The \nedtype{Igmpv2} module implements version 2 of the IGMP protocol
(RFC 2236). Next we describe its behaviour in host and routers in details.
Note that multicast routers behaves as hosts too, i.e. they are sending
reports to other routers when joining or leaving a multicast group.

\subsection{Host behaviour}

When an interface joins to a multicast group, the host
will send a Membership Report immediately to the group address.
This report is repeated after \fpar{unsolicetedReportInterval} to
cover the possibility of the first report being lost.

When a host's interface leaves a multicast group, and it was
the last host that sent a Membership Report for that group,
it will send a Leave Group message to the all-routers multicast
group (224.0.0.2).

This module also responds to IGMP Queries. When the host
receives a Group-Specific Query on an interface that belongs
to that group, then it will set a timer to a random value
between 0 and Max Response Time of the Query. If the timer
expires before the host observe a Membership Report sent
by other hosts, then the host sends an IGMPv2 Membership Report.
When the host receives a General Query on an interface,
a timer is initialized and a report is sent for each group
membership of the interface.

\subsection{Router behaviour}

Multicast routers maintains a list for each interface containing
the multicast groups that have listeners on that interface.
This list is updated when IGMP Membership Reports and Leave Group
messages arrive, or when a timer expires since the last Query.

When multiple routers are connected to the same link, the one with
the smallest IP address will be the Querier. When other routers
observe that they are Non-Queriers (by receiving an IGMP Query
with a lower source address), they stop sending IGMP Queries
until \fpar{otherQuerierPresentInterval} elapsed since the last
received query.

Routers periodically (\fpar{queryInterval}) send a General Query
on each attached network for which this router is a Querier.
On startup the router sends \fpar{startupQueryCount} queries
separated by \fpar{startupQueryInterval}. A General Query
has unspecified Group Address field, a Max Response Time
field set to \fpar{queryResponseInterval}, and is sent to the
all-systems multicast address (224.0.0.1).

When a router receives a Membership Report, it will add the
reported group to the list of multicast group memberships.
At the same time it will set a timer for the membership
to \fpar{groupMembershipInterval}. Repeated reports restart
the timer. If the timer expires, the router assumes
that the group has no local members, and multicast traffic
is no more forwarded to that interface.

When a Querier receives a Leave Group message for a group,
it sends a Group-Specific Query to the group being left.
It repeats the Query \fpar{lastMemberQueryCount} times in
separated by \fpar{lastMemberQueryInterval} until a Membership
Report is received. If no Report received, then the router
assumes that the group has no local members.

% FIXME IGMPv2 not compatible with IGMPv1 hosts and routers

\subsection{Disabling IGMP}

The IPv4 \nedtype{Ipv4NetworkLayer} contains an instance of the IGMP
module. IGMP can be turned off by setting the \fpar{enabled}
parameter to false. When disabled, then no IGMP message
is generated, and incoming IGMP messages are ignored.

\subsection{Parameters}

The following parameters has effects in both hosts and routers:

\begin{itemize}
  \item \fpar{enabled} if \fkeyword{false} then the IGMP module is silent. Default is \fkeyword{true}.
\end{itemize}

These parameters are only used in hosts:

\begin{itemize}
  \item \fpar{unsolicitedReportInterval} the time between repetitions of a
   host's initial report of membership in a group. Default is 10s.
\end{itemize}

Router timeouts are configured by these parameters:

\begin{itemize}
  \item \fpar{robustnessVariable} the IGMP is robust to \fpar{robustnessVariable}-1
   packet losses. Default is 2.
  \item \fpar{queryInterval} the interval between General Queries sent by a Querier.
   Default is 125s.
  \item \fpar{queryResponseInterval} the Max Response Time inserted into General Queries
  \item \fpar{groupMembershipInterval} the amount of time that must pass before
   a multicast router decides there are no more members of a group on a network.
   Fixed to \fpar{robustnessVariable} * \fpar{queryInterval} + \fpar{queryResponseInterval}.
  \item \fpar{otherQuerierPresentInterval} the length of time that must
   pass before a multicast router decides that there is no longer
   another multicast router which should be the querier.
   Fixed to \fpar{robustnessVariable} * \fpar{queryInterval} + \fpar{queryResponseInterval} / 2.
  \item \fpar{startupQueryInterval} the interval between General Queries
   sent by a Querier on startup. Default is \fpar{queryInterval} / 4.
  \item \fpar{startupQueryCount} the number of Queries sent out on startup,
   separated by the \fpar{startupQueryInterval}. Default is \fpar{robustnessVariable}.
  \item \fpar{lastMemberQueryInterval} the Max Response Time inserted into
   Group-Specific Queries sent in response to Leave Group messages, and
   is also the amount of time between Group-Specific Query messages.
   Default is 1s.
  \item \fpar{lastMemberQueryCount} the number of Group-Specific Queries
   sent before the router assumes there are no local members.
   Default is \fpar{robustnessVariable}.
\end{itemize}

\section{The IPv4NetworkLayer module}

The \nedtype{Ipv4NetworkLayer} module packs the \nedtype{IP}, \nedtype{Icmp},
\nedtype{Arp}, and \nedtype{IGMP} modules into one compound module.
The compound module defines gates for connecting UDP, TCP, SCTP, RSVP and
OSPF transport protocols. The \ttt{pingIn} and \ttt{pingOut} gates of the
\nedtype{Icmp} module are also available, while its \ttt{errorOut} gate
is connected to an inner \nedtype{ErrorHandling} component that writes
the ICMP errors to the log.

The component can be used in hosts and routers to support IPv4.

\section{The NetworkInfo module}

The \nedtype{NetworkInfo} module can be used to dump detailed information
about the network layer. This module does not send or received messages,
it is invoked by the \nedtype{ScenarioManager} instead. For example
the following \nedtype{ScenarioManager} script dump the routing table
of the \ttt{LSR2} module at simulation time $t=2$ into \ffilename{LSR2\_002.txt}:
\begin{filelisting}
<scenario>
  <at t="2">
    <routing module="NetworkInfo" target="LSR2" file="LSR2_002.txt"/>
  </at>
</scenario>
\end{filelisting}

The module currently support only the \ttt{routing} command which dumps
the routing table. The command has four parameters given as XML attributes:
\begin{itemize}
  \item \ttt{target} the name of the node that owns the routing table to be dumped
  \item \ttt{filename} the name of the file the output is directed to
  \item \ttt{mode} if set to ``a'', the output is appended to the file,
                   otherwise the target is truncated if the file existed
  \item \ttt{compat} if set to ``linux'', then the output is generated
                     in the format of the \ttt{route -n} command of Linux.
                     The output is sorted only if \ttt{compat} is
                     \fkeyword{true}.
\end{itemize}

%%% Local Variables:
%%% mode: latex
%%% TeX-master: "usman"
%%% End:


\cleardoublepage

% based on 'integration' branch: 446a1265c28822456bc0230d362bbabdee7778ba (2012-03-19)

\chapter{Configuring IPv4 networks}
\label{cha:ipv4-config}

\section{Overview}

TODO: this describes static autoconfiguration of IPv4 networks

An IPv4 network is composed of several nodes like hosts, routers,
switches, hubs, Ethernet buses, or wireless access points.
The nodes having a IPv4 network layer (hosts and routers) should be
configured at the beginning of the simulation. The configuration
assigns IP addresses to the nodes, and fills their routing tables.
If multicast forwarding is simulated, then the multicast routing
tables also must be filled in.

% TODO define nodes, IP nodes, routers, multicast routers

The configuration can be manual (each address and route is fully specified
by the user), or automatic (addresses and routes are generated by
a configurator module at startup).

Before version 1.99.4 INET offered \nedtype{Ipv4FlatNetworkConfigurator}
for automatic and routing files for manual configuration.
Both had serious limitations, so a new configurator has been added
in version 1.99.4: \nedtype{Ipv4NetworkConfigurator}. This configurator
supports both fully manual and fully automatic configuration. It
can also be used with partially specified manual configurations,
the configurator fills in the gaps automatically.

The next section describes the usage of \nedtype{Ipv4NetworkConfigurator},
\nedtype{Ipv4FlatNetworkConfigurator} and old routing files are described
in the following sections.

\subsection{Ipv4NetworkConfigurator}
\label{subsec:ipv4configurator}

The \nedtype{Ipv4NetworkConfigurator} assigns IP addresses and sets up
static routing for an IPv4 network.

It assigns per-interface IP addresses, strives to take subnets into account,
and can also optimize the generated routing tables by merging routing entries.

Hierarchical routing can be set up by using only a fraction of configuration
entries compared to the number of nodes. The configurator also does
routing table optimization that significantly decreases the size of routing
tables in large networks.

The configuration is performed in stage 2 of the initialization. At this
point interface modules (e.g. PPP) has already registered their interface
in the interface table. If an interface is named \ttt{ppp[0]}, then the
corresponding interface entry is named \ttt{ppp0}. This name can be used
in the config file to refer to the interface.

The configurator goes through the following steps:
\begin{enumerate}
  \item  Builds a graph representing the network topology. The graph
     will have a vertex for every module that has a @node property (this
     includes hosts, routers, and L2 devices like switches, access points,
     Ethernet hubs, etc.) It also assigns weights to vertices and edges that
     will be used by the shortest path algorithm when setting up routes.
     Weights will be infinite for IP nodes that have IP forwarding disabled
     (to prevent routes from transiting them), and zero for all other nodes
     (routers and and L2 devices). Edge weights are chosen to be inversely
     proportional to the bitrate of the link, so that the configurator
     prefers connections with higher bandwidth. For internal purposes,
     the configurator also builds a table of all "links" (the link data
     structure consists of the set of network interfaces that are
     on the same point-to-point link or LAN)

  \item  Assigns IP addresses to all interfaces of all nodes. The
     assignment process takes into consideration the addresses and netmasks
     already present on the interfaces (possibly set in earlier initialize
     stages), and the configuration provided in the XML format (described
     below). The configuration can specify "templates" for the address
     and netmask, with parts that are fixed and parts that can be chosen
     by the configurator (e.g. "10.0.x.x"). In the most general case,
     the configurator is allowed to choose any address and netmask for all
     interfaces (which results in automatic address assignment). In the most
     constrained case, the configurator is forced to use the requested addresses
     and netmasks for all interfaces (which translates to manual address assignment).
     There are many possible configuration options between these two extremums. The
     configurator assigns addresses in a way that maximizes the number of
     nodes per subnet. Once it figures out the nodes that belong to a single
     subnet it, will optimize for allocating the longest possible netmask.
     The configurator might fail to assign netmasks and addresses according
     to the given configuration parameters; if that happens, the assignment
     process stops and an error is signalled.

  \item  Adds the manual routes that are specified in the configuration.

  \item  Adds static routes to all routing tables in the network. The
     configurator uses Dijkstra's weighted shortest path algorithm to find
     the desired routes between all possible node pairs. The resulting
     routing tables will have one entry for all destination interfaces in the
     network. The configurator can be safely instructed to add default routes
     where applicable, significantly reducing the size of the host routing
     tables. It can also add subnet routes instead of interface routes further
     reducing the size of routing tables. Turning on this option requires
     careful design to avoid having IP addresses from the same subnet on
     different links. CAVEAT: Using manual routes and static route generation
     together may have unwanted side effects, because route generation ignores
     manual routes.

  \item  Then it optimizes the routing tables for size. This optimization allows
     configuring larger networks with smaller memory footprint and makes the
     routing table lookup faster. The resulting routing table might be
     different in that it will route packets that the original routing table
     did not. Nevertheless the following invariant holds: any packet routed
     by the original routing table (has matching route) will still be routed
     the same way by the optimized routing table.

  \item  Finally it dumps the requested results of the configuration. It can
     dump network topology, assigned IP addresses, routing tables and its
     own configuration format.
\end{enumerate}

The module can dump the result of the configuration in the XML format
which it can read. This is useful to save the result of a time consuming
configuration (large network with optimized routes), and use it as
the config file of subsequent runs.

\subsubsection*{Network topology graph}

The network topology graph is constructed from the nodes
of the network. The node is a module having a @node property
(this includes hosts, routers, and L2 devices like switches,
 access points, Ethernet hubs, etc.). An IP node is a node
that contains an \nedtype{InterfaceTable} and a \nedtype{Ipv4RoutingTable}.
A router is an IP node that has multiple network interfaces,
and IP forwarding is enabled in its routing table module.
In multicast routers the \fpar{forwardMulticast} parameter
is also set to \fkeyword{true}.

A link is a set of interfaces that can send datagrams to each other
without intervening routers. Each interface belongs to exactly
one link. For example two interface connected
by a point-to-point connection forms a link. Ethernet interfaces
connected via buses, hubs or switches.
The configurator identifies links by discovering
the connections between the IP nodes, buses, hubs, and switches.

Wireless links are identified by the \fpar{ssid} or \fpar{accessPointAddress}
parameter of the 802.11 management module. Wireless interfaces
whose node does not contain a management module are supposed
to be on the same wireless link. Wireless links can also be
configured in the configuration file of \nedtype{Ipv4NetworkConfigurator}:
\begin{verbatim}
<config>
  <wireless hosts="area1.*" interfaces="wlan*">
</config>
\end{verbatim}
puts wlan interfaces of the specified hosts into the same wireless link.

If a link contains only one router, it is marked as the gateway
of the link. Each datagram whose destination is outside the link
must go through the gateway.

\subsubsection*{Address assignment}

Addresses can be set up manually by giving the address and netmask for
each IP node. If some part of the address or netmask is unspecified,
then the configurator can fill them automatically. Unspecified fields
are given as an ``x'' character in the dotted notation of the address.
For example, if the address is specified as 192.168.1.1 and the
netmask is 255.255.255.0, then the node address will be 192.168.1.1
and its subnet is 192.168.1.0. If it is given as 192.168.x.x and
255.255.x.x, then the configurator chooses a subnet address in the range
of 192.168.0.0 - 192.168.255.252, and an IP address within the chosen
subnet. (The maximum subnet mask is 255.255.255.252 allows 2 nodes in the subnet.)

The following configuration generates network addresses below the 10.0.0.0
address for each link, and assign unique IP addresses to each host:

\begin{verbatim}
<config>
  <interface hosts="*" address="10.x.x.x" netmask="255.x.x.x"/>
</config>
\end{verbatim}

The configurator tries to put nodes on the same link into the same subnet,
so its enough to configure the address of only one node on each link.

The following example configures a hierarchical network in a way that keeps
routing tables small.
\begin{verbatim}
<config>
  <interface hosts="area11.lan1.*" address="10.11.1.x" netmask="255.255.255.x"/>
  <interface hosts="area11.lan2.*" address="10.11.2.x" netmask="255.255.255.x"/>
  <interface hosts="area12.lan1.*" address="10.12.1.x" netmask="255.255.255.x"/>
  <interface hosts="area12.lan2.*" address="10.12.2.x" netmask="255.255.255.x"/>
  <interface hosts="area*.router*" address="10.x.x.x" netmask="x.x.x.x"/>
  <interface hosts="*" address="10.x.x.x" netmask="255.x.x.0"/>
</config>
\end{verbatim}

The XML configuration must contain exactly one \verb!<config>! element. Under the
root element there can be multiple of the following elements:

The interface element provides configuration parameters for one or more
interfaces in the network. The selector attributes limit the scope where
the interface element has effects. The parameter attributes limit the
range of assignable addresses and netmasks.
The \verb!<interface>! element may contain the following attributes:
\begin{compactitem}
    \item \ttt{@hosts}
      Optional selector attribute that specifies a list of host name patterns.
      Only interfaces in the specified hosts are affected. The pattern might
      be a full path starting from the network, or a module name anywhere in
      the hierarchy, and other patterns similar to ini file keys. The default
      value is "*" that matches all hosts.
      e.g. "subnet.client*" or "host* router[0..3]" or "area*.*.host[0]"

    \item \ttt{@names}
      Optional selector attribute that specifies a list of interface name
      patterns. Only interfaces with the specified names are affected. The
      default value is "*" that matches all interfaces.
      e.g. "eth* ppp0" or "*"

    \item \ttt{@towards}
      Optional selector attribute that specifies a list of host name patterns.
      Only interfaces connected towards the specified hosts are affected. The
      specified name will be matched against the names of hosts that are on
      the same LAN with the one that is being configured. This works even if
      there's a switch between the configured host and the one specified here.
      For wired networks it might be easier to specify this parameter instead
      of specifying the interface names. The default value is "*".
      e.g. "ap" or "server" or "client*"

    \item \ttt{@among}
      Optional selector attribute that specifies a list of host name patterns.
      Only interfaces in the specified hosts connected towards the specified
      hosts are affected.
      The 'among="X Y Z"' is same as 'hosts="X Y Z" towards="X Y Z"'.

    \item \ttt{@address}
      Optional parameter attribute that limits the range of assignable
      addresses. Wildcards are allowed with using 'x' as part of the address
      in place of a byte. Unspecified parts will be filled automatically by
      the configurator. The default value "" means that the address will not
      be configured. Unconfigured interfaces still have allocated addresses
      in their subnets allowing them to become configured later very easily.
      e.g. "192.168.1.1" or "10.0.x.x"

    \item \ttt{@netmask}
      Optional parameter attribute that limits the range of assignable
      netmasks. Wildcards are allowed with using 'x' as part of the netmask
      in place of a byte. Unspecified parts will be filled automatically be
      the configurator. The default value "" means that any netmask can be
      configured.
      e.g. "255.255.255.0" or "255.255.x.x" or "255.255.x.0"

    \item \ttt{@mtu}                number
      Optional parameter attribute to set the MTU parameter in the interface.
      When unspecified the interface parameter is left unchanged.

    \item \ttt{@metric}                number
      Optional parameter attribute to set the Metric parameter in the interface.
      When unspecified the interface parameter is left unchanged.
\end{compactitem}

Wireless interfaces can similarly be configured by adding
\verb!<wireless>! elements to the configuration. Each \verb!<wireless>!
element with a different id defines a separate subnet.
\begin{compactitem}
    \item \ttt{@id} (optional)
      identifies wireless network, unique value used if missed

    \item \ttt{@hosts}
      Optional selector attribute that specifies a list of host name patterns.
      Only interfaces in the specified hosts are affected. The default value
      is "*" that matches all hosts.

    \item \ttt{@interfaces}
      Optional selector attribute that specifies a list of interface name
      patterns. Only interfaces with the specified names are affected. The
      default value is "*" that matches all interfaces.
\end{compactitem}


\subsubsection{Multicast groups}

Multicast groups can be configured by adding \verb!<multicast-group>!
elements to the configuration file. Interfaces belongs to a multicast
group will join to the group automatically.

For example
\begin{verbatim}
<config>
  <multicast-group hosts="router*" interfaces="eth*" address="224.0.0.5"/>
</config>
\end{verbatim}
adds all Ethernet interfaces of nodes whose name starts with ``router''
to the 224.0.0.5 multicast group.

The \verb!<multicast-group>! element has the following attributes:
\begin{compactitem}
    \item \ttt{@hosts}
      Optional selector attribute that specifies a list of host name patterns.
      Only interfaces in the specified hosts are affected. The default value
      is "*" that matches all hosts.

    \item \ttt{@interfaces}
      Optional selector attribute that specifies a list of interface name
      patterns. Only interfaces with the specified names are affected. The
      default value is "*" that matches all interfaces.

    \item \ttt{@towards}
      Optional selector attribute that specifies a list of host name patterns.
      Only interfaces connected towards the specified hosts are affected.
      The default value is "*".

    \item \ttt{@among}
      Optional selector attribute that specifies a list of host name patterns.
      Only interfaces in the specified hosts connected towards the specified
      hosts are affected.
      The 'among="X Y Z"' is same as 'hosts="X Y Z" towards="X Y Z"'.

    \item \ttt{@address}
      Mandatory parameter attribute that specifies a list of multicast group
      addresses to be assigned. Values must be selected from the valid range
      of multicast addresses.
      e.g. "224.0.0.1 224.0.1.33"
\end{compactitem}


\subsubsection*{Manual route configuration}

The \nedtype{Ipv4NetworkConfigurator} module allows the user
to fully specify the routing tables of IP nodes at the beginning
of the simulation.

The \verb!<route>! elements of the configuration add a route to the
routing tables of selected nodes. The element has the following attributes:
\begin{compactitem}
    \item \ttt{@hosts}
      Optional selector attribute that specifies a list of host name patterns.
      Only routing tables in the specified hosts are affected. The default
      value "" means all hosts will be affected.
      e.g. "host* router[0..3]"

    \item \ttt{@destination}
      Optional parameter attribute that specifies the destination address in
      the route (L3AddressResolver syntax). The default value is "*".
      e.g. "192.168.1.1" or "subnet.client[3]" or "subnet.server(ipv4)" or "*"

    \item \ttt{@netmask}
      Optional parameter attribute that specifies the netmask in the route.
      The default value is "*".
      e.g. "255.255.255.0" or "/29" or "*"

    \item \ttt{@gateway}
      Optional parameter attribute that specifies the gateway (next-hop)
      address in the route (L3AddressResolver syntax). When unspecified
      the interface parameter must be specified. The default value is "*".
      e.g. "192.168.1.254" or "subnet.router" or "*"

    \item \ttt{@interface}
      Optional parameter attribute that specifies the output interface name
      in the route. When unspecified the gateway parameter must be specified.
      This parameter has no default value.
      e.g. "eth0"

    \item \ttt{@metric}
      Optional parameter attribute that specifies the metric in the route.
      The default value is 0.
\end{compactitem}

Multicast routing tables can similarly be configured by adding
\verb!<multicast-route>! elements to the configuration.
\begin{compactitem}
    \item \ttt{@hosts}
      Optional selector attribute that specifies a list of host name patterns.
      Only routing tables in the specified hosts are affected.
      e.g. "host* router[0..3]"

    \item \ttt{@source}
      Optional parameter attribute that specifies the address of the source
      network. The default value is "*" that matches all sources.

    \item \ttt{@netmask}
      Optional parameter attribute that specifies the netmask of the source
      network. The default value is "*" that matches all sources.

    \item \ttt{@groups}
      Optional List of IPv4 multicast addresses specifying the groups this entry
      applies to. The default value is "*" that matches all multicast groups.
      e.g. "225.0.0.1 225.0.1.2".

    \item \ttt{@metric}
      Optional parameter attribute that specifies the metric in the route.

    \item \ttt{@parent}
      Optional parameter attribute that specifies the name of the interface
      the multicast datagrams are expected to arrive. When a datagram arrives
      on the parent interface, it will be forwarded towards the child interfaces;
      otherwise it will be dropped. The default value is the interface on the
      shortest path towards the source of the datagram.

    \item \ttt{@children}
      Mandatory parameter attribute that specifies a list of interface name
      patterns:
      \begin{compactitem}
        \item a name pattern (e.g. "ppp*") matches the name of the interface
        \item a 'towards' pattern (starting with ">", e.g. ">router*") matches the interface
         by naming one of the neighbour nodes on its link.
      \end{compactitem}
      Incoming multicast datagrams are forwarded to each child interface except the
      one they arrived in.
\end{compactitem}

The following example adds an entry to the multicast routing table of \ttt{router1},
that intsructs the routing algorithm to forward multicast datagrams whose source
is in the 10.0.1.0 network and whose destinatation address is 225.0.0.1 to
send on the \ttt{eth1} and \ttt{eth2} interfaces assuming it arrived on the
\ttt{eth0} interface:

\begin{verbatim}
<multicast-route hosts="router1" source="10.0.1.0" netmask="255.255.255.0"
                 groups="225.0.0.1" metric="10"
                 parent="eth0" children="eth1 eth2"/>
\end{verbatim}

\subsubsection*{Automatic route configuration}

If the \fpar{addStaticRoutes} parameter is true, then
the configurator add static routes to all routing tables.

The configurator uses Dijkstra's weighted shortest path algorithm to find
the desired routes between all possible node pairs. The resulting
routing tables will have one entry for all destination interfaces in the
network.

%     Weights will be infinite for IP nodes that have IP forwarding disabled
%     (to prevent routes from transiting them), and zero for all other nodes
%     (routers and and L2 devices). Edge weights are chosen to be inversely
%     proportional to the bitrate of the link, so that the configurator
%     prefers connections with higher bandwidth. For internal purposes,

The configurator can be safely instructed to add default routes
where applicable, significantly reducing the size of the host routing
tables. It can also add subnet routes instead of interface routes further
reducing the size of routing tables. Turning on this option requires
careful design to avoid having IP addresses from the same subnet on
different links.


\begin{caution}
Using manual routes and static route generation
together may have unwanted side effects, because route generation ignores
manual routes. Therefore if the configuration file contains
manual routes, then the \fpar{addStaticRoutes} parameter should be set
to \fkeyword{false}.
\end{caution}

\subsubsection*{Route optimization}

If the \fpar{optimizeRoutes} parameter is \fkeyword{true} then the
configurator tries to optimize the routing table for size.
This optimization allows configuring larger networks with smaller
memory footprint and makes the routing table lookup faster.

The optimization is performed by merging routes whose gateway and
outgoing interface is the same by finding a common prefix that
matches only those routes. The resulting routing table might be
different in that it will route packets that the original routing table
did not. Nevertheless the following invariant holds: any packet routed
by the original routing table (has matching route) will still be routed
the same way by the optimized routing table.

\subsubsection*{Parameters}

This list summarize the parameters of the \nedtype{IPv4NetorkConfigurator}:

\begin{params}
  \param{config}
   {XML configuration parameters for IP address assignment and adding manual routes.}
  \param{assignAddresses}
   {assign IP addresses to all interfaces in the network}
  \param{assignDisjunctSubnetAddresses}
   {avoid using the same address prefix and
    netmask on different links when assigning IP addresses to interfaces}
  \param{addStaticRoutes}
   {add static routes to the routing tables of all nodes
    to route to all destination interfaces (only where applicable; turn off when
    config file contains manual routes)}
  \param{addDefaultRoutes}
    {add default routes if all routes from a source node go
     through the same gateway (used only if addStaticRoutes is true)}
  \param{addSubnetRoutes}
   {add subnet routes instead of destination interface routes
    (only where applicable; used only if addStaticRoutes is true)}
  \param{optimizeRoutes}
   {optimize routing tables by merging routes, the resulting routing table might
    route more packets than the original (used only if addStaticRoutes is true)}
  \param{dumpTopology}
   {if true, then the module prints extracted network topology}
  \param{dumpAddresses}
   {if true, then the module prints assigned IP addresses for all interfaces}
  \param{dumpRoutes}
   {if true, then the module prints configured and optimized routing tables for all nodes to
    the module output}
  \param{dumpConfig}
   {name of the file, write configuration into the given config file that can be fed back
    to speed up subsequent runs (network configurations)}
\end{params}

\subsection{Ipv4FlatNetworkConfigurator}

The \nedtype{Ipv4FlatNetworkConfigurator} module configures
IP addresses and routes of IP nodes of a network.
All assigned addresses share a common subnet prefix,
the network topology will be ignored. Shortest path
routes are also generated from any node to any other
node of the network. The Gateway (next hop) field of the routes
is not filled in by these configurator, so it relies
on proxy ARP if the network spans several LANs.

% no optimization of routing tables

The \nedtype{Ipv4FlatNetworkConfigurator} module configures
the network when it is initialized. The configuration
is performed in stage 2, after interface tables are
filled in. Do not use a \nedtype{Ipv4FlatNetworkConfigurator}
module together with static routing files, because they
can iterfere with the configurator.

The \nedtype{Ipv4FlatNetworkConfigurator} searches each IP nodes of the network.
(IP nodes are those modules that have the @node NED property and
has a \nedtype{Ipv4RoutingTable} submodule named ``routingTable'').
The configurator then assigns IP addresses to the IP nodes, controlled
by the following module parameters:
\begin{itemize}
  \item \fpar{netmask} common netmask of the addresses (default is 255.255.0.0)
  \item \fpar{networkAddress} higher bits are the network part of the addresses,
        lower bits should be 0. (default is 192.168.0.0)
\end{itemize}

With the default parameters the assigned addresses are in the range
192.168.0.1 - 192.168.255.254, so there can be maximum 65534 nodes in the
network. The same IP address will be assigned to each interface
of the node, except the loopback interface which always has address 127.0.0.1
(with 255.0.0.0 mask).

After assigning the IP addresses, the configurator fills in the routing tables.
There are two kind of routes:
\begin{itemize}
  \item default routes: for nodes that has only one non-loopback interface
        a route is added that matches with any destination address
        (the entry has 0.0.0.0 \ttt{host} and \ttt{netmask} fields).
        These are remote routes, but the gateway address is left unspecified.
        The delivery of the datagrams rely on the proxy ARP feature of the
        routers.
  \item direct routes following the shortest paths: for nodes that has more
        than one non-loopback interface a separate route is added to each
        IP node of the network. The outgoing interface is chosen by the
        shortest path to the target node. These routes are
        added as direct routes, even if there is no direct link with the
        destination. In this case proxy ARP is needed to deliver the datagrams.
\end{itemize}

\begin{note}
This configurator does not try to optimize the routing tables.
If the network contains $n$ nodes, the size of all routing tables
will be proportional to $n^2$, and the time of the lookup of the
best matching route will be proportional to $n$.
\end{note}

% FIXME weird FlatNetworkConfigurator behaviour.
%       Assigned IP addresses does not mirror the hierachy of networks (e.g. each node in an Ethernet LAN handled as a one-element subnet).
%       No gateway address is set in the routes, delivery relies on proxy ARPing.
%       Direct routes created to each node, even if there is no direct link to it.
%       Different interfaces of a node should have different IP address.
%       Broadcast capable interfaces should have a real netmast (not 255.255.255.255) to support subnet directed IP broadcasts.

\subsection{Old routing files}
\label{subsec:routing_files}

Routing files are files with \ttt{.irt} or \ttt{.mrt} extension,
and their names are passed in the \fpar{routingFile} parameter
to \nedtype{Ipv4RoutingTable} modules.

Routing files may contain network interface configuration and static
routes. Both are optional. Network interface entries in the file
configure existing interfaces; static routes are added to the route table.

Interfaces themselves are represented in the simulation by modules
(such as the PPP module). Modules automatically register themselves
with appropriate defaults in the IPv4RoutingTable, and entries in the
routing file refine (overwrite) these settings.
Interfaces are identified by names (e.g. ppp0, ppp1, eth0) which
are normally derived from the module's name: a module called
\ttt{"ppp[2]"} in the NED file registers itself as interface ppp2.

An example routing file (copied here from one of the example simulations):

\begin{verbatim}
ifconfig:

# ethernet card 0 to router
name: eth0   inet_addr: 172.0.0.3   MTU: 1500   Metric: 1  BROADCAST MULTICAST
Groups: 225.0.0.1:225.0.1.2:225.0.2.1

# Point to Point link 1 to Host 1
name: ppp0   inet_addr: 172.0.0.4   MTU: 576   Metric: 1

ifconfigend.

route:
172.0.0.2   *           255.255.255.255  H  0   ppp0
172.0.0.4   *           255.255.255.255  H  0   ppp0
default:    10.0.0.13   0.0.0.0          G  0   eth0

225.0.0.1   *           255.255.255.255  H  0   ppp0
225.0.1.2   *           255.255.255.255  H  0   ppp0
225.0.2.1   *           255.255.255.255  H  0   ppp0

225.0.0.0   10.0.0.13   255.0.0.0        G  0   eth0

routeend.
\end{verbatim}

The \ttt{ifconfig...ifconfigend.} part configures interfaces,
and \ttt{route..routeend.} part contains static routes.
The format of these sections roughly corresponds to the output
of the \ttt{ifconfig} and \ttt{netstat -rn} Unix commands.

An interface entry begins with a \ttt{name:} field, and lasts until
the next \ttt{name:} (or until \ttt{ifconfigend.}). It may
be broken into several lines.

Accepted interface fields are:

\begin{itemize}
  \item \ttt{name:} - arbitrary interface name (e.g. eth0, ppp0)
  \item \ttt{inet\_addr:} - IP address
  \item \ttt{Mask:} - netmask
  \item \ttt{Groups:} Multicast groups. 224.0.0.1 is added automatically,
     and 224.0.0.2 also if the node is a router (IPForward==true).
  \item \ttt{MTU:} - MTU on the link (e.g. Ethernet: 1500)
  \item \ttt{Metric:} - integer route metric
  \item flags: \ttt{BROADCAST}, \ttt{MULTICAST}, \ttt{POINTTOPOINT}
\end{itemize}

The following fields are parsed but ignored: \ttt{Bcast},\ttt{encap},
\ttt{HWaddr}.

Interface modules set a good default for MTU, Metric (as $2*10^9$/bitrate) and
flags, but leave \fvar{inet\_addr} and \fvar{Mask} empty. \fvar{inet\_addr} and
\fvar{mask} should be set either from the routing file or by a dynamic network
configuration module.

The route fields are:

\begin{verbatim}
Destination  Gateway  Netmask  Flags  Metric Interface
\end{verbatim}

\fvar{Destination}, \fvar{Gateway} and \fvar{Netmask} have the usual meaning.
The \fvar{Destination} field should either be an IP address or ``default''
(to designate the default route). For \fvar{Gateway}, \ttt{*} is also
accepted with the meaning \ttt{0.0.0.0}.

\fvar{Flags} denotes route type:

\begin{itemize}
  \item \textit{H} ``host'': direct route (directly attached to the router), and
  \item \textit{G} ``gateway'': remote route (reached through another router)
\end{itemize}

\fvar{Interface} is the interface name, e.g. \ttt{eth0}.

\begin{important}
The meaning of the routes where the destination is a multicast address
has been changed in version 1.99.4. Earlier these entries was used
both to select the outgoing interfaces of multicast datagrams
sent by the higher layer (if multicast interface was otherwise unspecified)
and to select the outgoing interfaces of datagrams that are received from
the network and forwarded by the node.

From version 1.99.4 multicast routing applies reverse path forwarding.
This requires a separate routing table, that can not be populated from
the old routing table entries. Therefore simulations that use multicast
forwarding can not use the old configuration files, they should be
migrated to use an \nedtype{Ipv4NetworkConfigurator} instead.

Some change is needed in models that use link-local multicast too.
Earlier if the IP module received a datagram from the higher layer
and multiple routes was given for the multicast group,
then IP sent a copy of the datagram on each interface of that routes.
From version 1.99.4, only the first matching interface is used (considering
longest match). If the application wants to send the multicast datagram
on each interface, then it must explicitly loop and specify the multicast
interface.
\end{important}

% FIXME 'H' and 'G' flags should be independent. Now they excludes each other, the parser sets route.type to the last one.
%       H = host/network
%       G = indirect/direct

% TODO warn that multicast configuration has changed


%%% Local Variables:
%%% mode: latex
%%% TeX-master: "usman"
%%% End:


\cleardoublepage

\chapter{IPv6 and Mobile IPv6}
\label{cha:ipv6}


\section{Overview}

IPv6 support is implemented by several cooperating modules. The IPv6 module
implements IPv6 datagram handling (sending, forwarding etc). It relies on
\nedtype{Ipv6RoutingTable} to get access to the routes. \nedtype{Ipv6RoutingTable} also contains the
neighbour discovery data structures (destination cache, neighbour cache,
prefix list -- the latter effectively merged into the route table). Interface
configuration (address, state, timeouts etc) is held in the \nedtype{InterfaceTable},
in \cppclass{Ipv6InterfaceData} objects attached to \cppclass{InterfaceEntry}
as its \ttt{ipv6()} member.

The module \nedtype{Ipv6NeighbourDiscovery} implements all tasks associated with
neighbour discovery and stateless address autoconfiguration. The data
structures themselves (destination cache, neighbour cache, prefix list)
are kept in \nedtype{Ipv6RoutingTable}, and are accessed via public C++ methods.
Neighbour discovery packets are only sent and processed by this module --
when IPv6 receives one, it forwards the packet to \nedtype{Ipv6NeighbourDiscovery}.

The rest of ICMPv6 (ICMP errors, echo request/reply etc) is implemented in
the module \nedtype{Icmpv6}, just like with IPv4. ICMP errors are sent into
\nedtype{Ipv6ErrorHandling}, which the user can extend or replace to get errors
handled in any way they like.


%%% Local Variables:
%%% mode: latex
%%% TeX-master: "usman"
%%% End:


\cleardoublepage

\chapter{The UDP Model}
\label{cha:udp}

\section{The UDP module}

The state of the sockets are stored within the UDP module and the application
can configure the socket by sending command messages to the UDP module.
These command messages are distinguished by their kind and the type of their
control info. The control info identifies the socket and holds the parameters
of the command.

Applications don't have to send messages directly to the UDP module,
as they can use the \cppclass{UdpSocket} utility class, which encapsulates the messaging and
provides a socket like interface to applications.

\subsection{Sending UDP datagrams}

If the application want to send datagrams, it optionally can connect to the destination.
It does this be sending a message with UDP\_C\_CONNECT kind and \cppclass{UdpConnectCommand}
control info containing the remote address and port of the connection.
The UDP protocol is in fact connectionless, so it does not send any packets as a result
of the connect call. When the UDP module receives the connect request,
it simply remembers the destination address and port and use it as default destination
for later sends. The application can send several connect commands to the same socket.

% FIXME currently connect() or bind() is mandatory as the first command,
%       the application cannot send packets or set options otherwise

% FIXME connect() should allow unspecified dest address and -1 port (interpreted as disconnect())

For sending an UDP packet, the application should attach an \cppclass{UDPSendCommand}
control info to the packet, and send it to \nedtype{Udp}. The control info may contain
the destination address and port. If the destination address or port
is unspecified in the control info then the packet is sent to the connected target.

The \nedtype{Udp} module encapsulates the application's packet into an \msgtype{UDPPacket},
creates an appropriate IP control info and send it over ipOut or ipv6Out depending on
the destination address.

The destination address can be the IPv4 local broadcast address (255.255.255.255)
or a multicast address. Before sending broadcast messages, the socket must be configured
for broadcasting. This is done by sending an message to the UDP module. The message
kind is UDP\_C\_SETOPTION and its control info (an \cppclass{UdpSetBroadcastCommand})
tells if the broadcast is enabled. You can limit the multicast to the local network
by setting the TTL of the IP packets to 1. The TTL can be configured per socket,
by sending a message to the UDP with an \cppclass{UDPSetTimeToLive} control info
containing the value. If the node has multiple interfaces, the application can
choose which is used for multicast messages. This is also a socket option, the
id of the interface (as registered in the interface table) can be given in an
\cppclass{UdpSetMulticastInterfaceCommand} control info.

% FIXME currently sending broadcast messages is enabled without setting SO_BROADCAST to true,
%       this is not so in UNIX

% FIXME there should be a separate TTL for multicast (not used for unicast), default value is 1
%       see IP_MULTICAST_TTL in `man 7 ip`

\begin{note}
The \nedtype{Udp} module supports only local broadcasts (using the special 255.255.255.255 address).
Packages that are broadcasted to a remote subnet are handled as undeliverable messages.
\end{note}

If the UDP packet cannot be delivered because nobody listens on the destination port,
the application will receive a notification about the failure. The notification is
a message with UDP\_I\_ERROR kind having attached an \cppclass{UdpErrorIndication}
control info. The control info contains the local and destination address/port,
but not the original packet.

After the application finished using a socket, it should close it by sending a message
UDP\_C\_CLOSE kind and \cppclass{UdpCloseCommand} control info. The control info
contains only the socket identifier. This command frees the resources associated
with the given socket, for example its socket identifier or bound address/port.

\subsection{Receiving UDP datagrams}

Before receiving UDP datagrams applications should first ``bind'' to the given UDP port.
This can be done by sending a message with message kind UDP\_C\_BIND attached with an
\cppclass{UdpBindCommand} control info. The control info contains the socket identifier
and the local address and port the application want to receive UDP packets.
Both the address and port is optional. If the address is unspecified, than the UDP
packets with any destination address is passed to the application. If the port is
-1, then an unused port is selected automatically by the UDP module.
The localAddress/localPort combination must be unique.

When a packet arrives from the network, first its error bit is checked. Erronous messages
are dropped by the UDP component. Otherwise the application bound to the destination port
is looked up, and the decapsulated packet passed to it. If no application is bound to
the destination port, an ICMP error is sent to the source of the packet. If the socket is
connected, then only those packets are delivered to the application, that received from
the connected remote address and port.

The control info of the decapsulated packet is an \cppclass{UDPDataIndication}
and contains information about the source and destination address/port, the TTL,
and the identifier of the interface card on which the packet was received.

The applications are bound to the unspecified local address, then they receive any packets
targeted to their port. UDP also supports multicast and broadcast addresses; if they
are used as destination address, all nodes in the multicast group or subnet receives the packet.
The socket receives the broadcast packets only if it is configured for broadcast.
To receive multicast messages, the socket must join to the group of the multicast address.
This is done be sending the UDP module an UDP\_C\_SETOPTION message with
\cppclass{UdpJoinMulticastGroupsCommand} control info. The control info specifies the
multicast addresses and the interface identifiers. If the interface identifier is given
only those multicast packets are received that arrived at that interface.
The socket can stop receiving multicast messages if it leaves the multicast group.
For this purpose the application should send the UDP another UDP\_C\_SETOPTION
message in their control info (\cppclass{UdpLeaveMulticastGroupsCommand}) specifying
the multicast addresses of the groups.

% TODO clarify: multicast packets should not be delivered to connected sockets?

\subsection{Signals}

The \nedtype{Udp} module emits the following signals:
\begin{itemize}
  \item \fsignal{sentPk} when an UDP packet sent to the IP, the packet
  \item \fsignal{rcvdPk} when an UDP packet received from the IP, the packet
  \item \fsignal{passedUpPk} when a packet passed up to the application, the packet
  \item \fsignal{droppedPkWrongPort} when an undeliverable UDP packet received, the packet
  \item \fsignal{droppedPkBadChecksum} when an erronous UDP packet received, the packet
\end{itemize}

\section{UDP sockets}

UDPSocket is a convenience class, to make it easier to send and receive
UDP packets from your application models. You'd have one (or more)
UDPSocket object(s) in your application simple module class, and call
its member functions (bind(), connect(), sendTo(), etc.) to create and
configure a socket, and to send datagrams.

UDPSocket chooses and remembers the sockId for you, assembles and sends command
packets such as UDP\_C\_BIND to UDP, and can also help you deal with packets and
notification messages arriving from UDP.

Here is a code fragment that creates an UDP socket and sends a 1K packet
over it (the code can be placed in your handleMessage() or activity()):

\begin{cpp}
UDPSocket socket;
socket.setOutputGate(gate("udpOut"));
socket.connect(L3AddressResolver().resolve("10.0.0.2"), 2000);

cPacket *pk = new cPacket("dgram");
pk->setByteLength(1024);
socket.send(pk);

socket.close();
\end{cpp}

% when the localAddr is unspecified by the socket (~ INADDR_ANY), then the kernel sets the
% source address field of the outgoing packet according to the outgoing interface

Processing messages sent up by the UDP module is relatively straightforward.
You only need to distinguish between data packets and error notifications,
by checking the message kind (should be either UDP\_I\_DATA or UDP\_I\_ERROR),
and casting the control info to UDPDataIndication or UDPErrorIndication.
USPSocket provides some help for this with the \ffunc{belongsToSocket()} and
\ffunc{belongsToAnyUDPSocket()} methods.

\begin{cpp}
void MyApp::handleMessage(cMessage *msg)
{
    if (msg->getKind() == UDP_I_DATA)
    {
        if (socket.belongsToSocket())
            processUDPPacket(PK(msg));
    }
    else if (msg->getKind() == UDP_I_ERROR)
    {
        processUDPError(msg);
    }
    else
    {
        error("Unrecognized message (%s)", msg->getClassName());
    }
}
\end{cpp}

\cleardoublepage

\chapter{The TCP Models}
\label{cha:tcp}

\section{The TCP Module}

The \nedtype{Tcp} model relies on sending and receiving \cppclass{IPControlInfo} objects
attached to TCP segment objects as control info (see \ffunc{cMessage::setControlInfo()}).

The \nedtype{Tcp} module manages several \cppclass{TcpConnection} object each
holding the state of one connection. The connections are identified
by a connection identifier which is choosen by the application.
If the connection is established it can also be identified by
the local and remote addresses and ports. The TCP module simply
dispatches the incoming application commands and packets to
the corresponding object.

\subsection{TCP packets}
\label{subsec:tcp_packets}

The INET framework models the TCP header with the \msgtype{TcpHeader} message class.
This contains the fields of a TCP frame, except:
\begin{compactitem}
  \item \emph{Data Offset}: represented by \ffunc{cMessage::length()}
  \item \emph{Reserved}
  \item \emph{Checksum}: modelled by \ffunc{cMessage::hasBitError()}
  \item \emph{Options}: only EOL, NOP, MSS, WS, SACK\_PERMITTED, SACK and TS are possible
  \item \emph{Padding}
\end{compactitem}

The Data field can either be represented by (see \cppclass{TcpDataTransferMode}):
\begin{compactitem}
  \item encapsulated C++ packet objects,
  \item raw bytes as a \cppclass{ByteArray} instance,
  \item its byte count only,
\end{compactitem}
corresponding to transfer modes OBJECT, BYTESTREAM, BYTECOUNT resp.


\subsection{TCP commands}

The application and the TCP module communicates with each other
by sending \cppclass{cMessage} objects. These messages are specified
in the \ffilename{TCPCommand.msg} file.

The \cppclass{TCPCommandCode} enumeration defines the message kinds
that are sent by the application to the TCP:
\begin{itemize}
  \item TCP\_C\_OPEN\_ACTIVE: active open
  \item TCP\_C\_OPEN\_PASSIVE: passive open
  \item TCP\_C\_SEND: send data
  \item TCP\_C\_CLOSE: no more data to send
  \item TCP\_C\_ABORT: abort connection
  \item TCP\_C\_STATUS: request status info from TCP
\end{itemize}

Each command message should have an attached control info of type \cppclass{TcpCommand}.
Some commands (TCP\_C\_OPEN\_xxx, TCP\_C\_SEND) use subclasses.
The \cppclass{TcpCommand} object has a \fvar{connId} field that identifies the
connection locally within the application. \fvar{connId} is to be chosen by the
application in the open command.

When the application receives a message from the TCP, the message kind is
set to one of the \cppclass{TCPStatusInd} values:
\begin{itemize}
  \item TCP\_I\_ESTABLISHED: connection established
  \item TCP\_I\_CONNECTION\_REFUSED: connection refused
  \item TCP\_I\_CONNECTION\_RESET: connection reset
  \item TCP\_I\_TIME\_OUT: connection establish timer went off, or max retransmission count reached
  \item TCP\_I\_DATA: data packet
  \item TCP\_I\_URGENT\_DATA: urgent data packet
  \item TCP\_I\_PEER\_CLOSED: FIN received from remote TCP
  \item TCP\_I\_CLOSED: connection closed normally
  \item TCP\_I\_STATUS: status info
\end{itemize}

These messages also have an attached control info with \cppclass{TcpCommand}
or derived type (TCPConnectInfo, TCPStatusInfo, TCPErrorInfo).

% receive() calls are not modeled, incoming data passed to the application right away
% how accurate the modeling of the receiver window?

\subsection{TCP parameters}

The \nedtype{Tcp} module has the following parameters:
\begin{itemize}
  \item \fpar{advertisedWindow} in bytes, corresponds with the maximal receiver buffer capacity (Note: normally, NIC queues should be at least this size, default is  14*mss)
  \item \fpar{delayedAcksEnabled} delayed ACK algorithm (RFC 1122) enabled/disabled
  \item \fpar{nagleEnabled} Nagle's algorithm (RFC 896) enabled/disabled
  \item \fpar{limitedTransmitEnabled} Limited Transmit algorithm (RFC 3042) enabled/disabled (can be used for TCPReno/TCPTahoe/TCPNewReno/TCPNoCongestionControl)
  \item \fpar{increasedIWEnabled} Increased Initial Window (RFC 3390) enabled/disabled
  \item \fpar{sackSupport} Selective Acknowledgment (RFC 2018, 2883, 3517) support (header option) (SACK will be enabled for a connection if both endpoints support it)
  \item \fpar{windowScalingSupport} Window Scale (RFC 1323) support (header option) (WS will be enabled for a connection if both endpoints support it)
  \item \fpar{timestampSupport} Timestamps (RFC 1323) support (header option) (TS will be enabled for a connection if both endpoints support it)
  \item \fpar{mss} Maximum Segment Size (RFC 793) (header option, default is 536)
  \item \fpar{tcpAlgorithmClass} the name of TCP flavour

             Possible values are ``TCPReno'' (default), ``TCPNewReno'', ``TCPTahoe'', ``TCPNoCongestionControl'' and ``DumpTCP''.
             In the future, other classes can be written which implement Vegas, LinuxTCP  or other variants.
             See section \ref{sec:tcp_algorithms} for detailed description of implemented flavours.

             Note that TCPOpenCommand allows tcpAlgorithmClass to be chosen per-connection.

  \item \fpar{recordStats} if set to false it disables writing excessive amount of output vectors
\end{itemize}

\section{TCP connections}

Most part of the TCP specification is implemented in the
\cppclass{TcpConnection} class: takes care of the state machine,
stores the state variables (TCB), sends/receives SYN, FIN, RST, ACKs, etc.
TCPConnection itself implements the basic TCP ``machinery'',
the details of congestion control are factored out to
\cppclass{TcpAlgorithm} classes.

There are two additional objects the \cppclass{TcpConnection}
relies on internally: instances of \cppclass{TcpSendQueue} and
\cppclass{TcpReceiveQueue}. These polymorph classes manage the actual data stream,
so \cppclass{TcpConnection} itself only works with sequence number variables.
This makes it possible to easily accomodate need for various types of
simulated data transfer: real byte stream, "virtual" bytes (byte counts
only), and sequence of \cppclass{cMessage} objects (where every message object is
mapped to a TCP sequence number range).

\subsection{Data transfer modes}

Different applications have different needs how to represent
the messages they communicate with. Sometimes it is enough to
simulate the amount of data transmitted (``200 MB''), contents
does not matter. In other scenarios contents matters a lot.
The messages can be represented as a stream of bytes, but
sometimes it is easier for the applications to pass message
objects to each other (e.g. HTTP request represented by a
\msgtype{HTTPRequest} message class).

The TCP modules in the INET framework support 3 data transfer modes:

\begin{itemize}
  \item \ttt{TCP\_TRANSFER\_BYTECOUNT}: only byte counts are
        represented, no actual payload in \msgtype{TcpHeader}s.
        The TCP sends as many TCP segments as needed
  \item \ttt{TCP\_TRANSFER\_BYTESTREAM}: the application can pass
        byte arrays to the TCP. The sending TCP breaks down the bytes
        into MSS sized chunks and transmits them as the payload
        of the TCP segments. The receiving application can read the
        chunks of the data.
  \item \ttt{TCP\_TRANSFER\_OBJECT}: the application pass a
        \cppclass{cMessage} object to the TCP. The sending
        TCP sends as many TCP segments as needed according to
        the message length. The \cppclass{cMessage} object
        is also passed as the payload of the first segment. % check: first?
        The receiving application receives the object only
        when its last byte is received.
\end{itemize}

These values are defined in \ffilename{TCPCommand.msg} as
the \cppclass{TcpDataTransferMode} enumeration. The application
can set the data transfer mode per connection when the connection
is opened. The client and the server application must specify
the same data transfer mode.


\subsection{Opening connections}

Applications can open a local port for incoming connections by sending
the TCP a TCP\_C\_PASSIVE\_OPEN message. The attached control info
(an \cppclass{TcpOpenCommand}) contains the local address and port.
The application can specify that it wants to handle
only one connection at a time, or multiple simultanous connections. If the
\fvar{fork} field is true, it emulates the Unix accept(2) semantics: a new
connection structure is created for the connection (with a new \fvar{connId}),
and the connection with the old connection id remains listening.
If \fvar{fork} is false, then the first connection is accepted
(with the original \fvar{connId}),
and further incoming connections will be refused by the TCP by sending an RST segment.
The \fvar{dataTransferMode} field in \cppclass{TcpOpenCommand} specifies
whether the application data is transmitted as C++ objects, real bytes or byte
counts only. The congestion control algorithm can also be specified
on a per connection basis by setting \fvar{tcpAlgorithmClass} field to the
name of the algorithm.

The application opens a connection to a remote server by sending the TCP
a TCP\_C\_OPEN\_ACTIVE command. The TCP creates a \cppclass{TcpConnection}
object an sends a SYN segment. The initial sequence number selected according
to the simulation time: 0 at time 0, and increased by 1 in each 4$\mu$s.
If there is no response to the SYN segment, it retry after 3s, 9s, 21s and
45s. After 75s a connection establishment timeout (TCP\_I\_TIMEOUT) reported
to the application and the connection is closed.

When the connection gets established, TCP sends a TCP\_I\_ESTABLISHED
notification to the application. The attached control info
(a \cppclass{TcpConnectInfo} instance)
will contain the local and remote addresses and ports of the connection.
If the connection is refused by the remote peer (e.g. the port is not open),
then the application receives a TCP\_I\_CONNECTION\_REFUSED message.

\begin{note}
If you do active OPEN, then send data and close before the connection
has reached ESTABLISHED, the connection will go from SYN\_SENT to CLOSED
without actually sending the buffered data. This is consistent with
RFC 793 but may not be what you would expect.
\end{note}

\begin{note}
Handling segments with SYN+FIN bits set (esp. with data too) is
inconsistent across TCPs, so check this one if it is of importance.
\end{note}

\subsection{Sending Data}

The application can write data into the connection
by sending a message with TCP\_C\_SEND kind to the TCP.
The attached control info must be of type \cppclass{TCPSendCommand}.

The TCP will add the message to the \emph{send queue}.
There are three type of send queues corresponding to the
three data transfer mode. If the payload is transmitted as a message
object, then \cppclass{TCPMsgBasedSendQueue};
if the payload is a byte array then \cppclass{TCPDataStreamSendQueue};
if only the message lengths are represented then \cppclass{TCPVirtualDataSendQueue}
are the classes of send queues. The appropriate queue is created based
on the value of the \fpar{dataTransferMode} parameter of the Open command, no
further configuration is needed.

The message is handed over to the IP when there is
enough room in the windows. If Nagle's algorithm is
enabled, the TCP will collect 1 SMSS data and sends
them toghether.

\begin{note}
There is no way to set the PUSH and URGENT flags, when sending data.
\end{note}

% FIXME urgBit is never set
% FIXME model TCP_NODELAY, there is no PUSH flag in socket.send() (TCP_PUSH option ?)

\subsection{Receiving Data}

The TCP connection stores the incoming segments in the
\emph{receive queue}. The receive queue also has three flavours:
\cppclass{TCPMsgBasedRcvQueue}, \cppclass{TCPDataStreamRcvQueue}
and \cppclass{TCPVirtualDataRcvQueue}. The queue is created
when the connection is opened according to the \fvar{dataTransferMode}
of the connection.

Finite receive buffer size is modeled by the \fpar{advertisedWindow}
parameter. If receive buffer is exhausted (by out-of-order
segments) and the payload length of a new received segment
is higher than the free receiver buffer, the new segment will be dropped.
Such drops are recorded in \emph{tcpRcvQueueDrops} vector.

If the \emph{Sequence Number} of the received segment is the next
expected one, then the data is passed
to the application immediately. The \ffunc{recv()} call of
Unix is not modeled.

The data of the segment, which can be either a \cppclass{cMessage}
object, a \cppclass{ByteArray} object, or a simply byte count,
is passed to the application in a message that has
TCP\_I\_DATA kind.

% when the cMessage object is passed to the app? when last byte received?

\begin{note}
The TCP module does not handle the segments with PUSH or URGENT
flags specially. The data of the segment passed to the application
as soon as possible, but the application can not find out if that
data is urgent or pushed.
\end{note}

\subsection{RESET handling}

When an error occures at the TCP level, an RST segment is sent to
the communication partner and the connection is aborted.
Such error can be:
\begin{compactitem}
  \item arrival of a segment in CLOSED state
  \item an incoming segment acknowledges something not yet sent.
\end{compactitem}

The receiver of the RST it will abort the connection.
If the connection is not yet established, then the passive
end will go back to the LISTEN state and waits for another
incoming connection instead of aborting.

\subsection{Closing connections}

When the application does not have more data to send, it closes the
connection by sending a TCP\_C\_CLOSE command to the TCP. The TCP
will transmit all data from its buffer and in the last segment sets
the FIN flag. If the FIN is not acknowledged in time it will be
retransmitted with exponential backoff.

The TCP receiving a FIN segment will notify the application that
there is no more data from the communication partner. It sends
a TCP\_I\_PEER\_CLOSED message to the application containing
the connection identifier in the control info.

When both parties have closed the connection, the applications
receive a TCP\_I\_CLOSED message and the connection object is
deleted. (Actually one of the TCPs waits for $2 MSL$ before
deleting the connection, so it is not possible to reconnect
with the same addresses and port numbers immediately.)

\subsection{Aborting connections}

The application can also abort the connection. This means that
it does not wait for incoming data, but drops the data associated
with the connection immediately. For this purpose the application
sends a TCP\_C\_ABORT message specifying the connection identifier
in the attached control info. The TCP will send a RST to the
communication partner and deletes the connection object. The application
should not reconnect with the same local and remote addresses and
ports within MSL (maximum segment lifetime), because segments
from the old connection might be accepted in the new one.

\subsection{Status Requests}

Applications can get detailed status information about an existing
connection. For this purpose they send the TCP module a TCP\_C\_STATUS
message attaching an \cppclass{TcpCommand} info with the identifier
of the connection. The TCP will respond with a TCP\_I\_STATUS message
with a \cppclass{TcpStatusInfo} attachement. This control info
contains the current state, local and remote addresses and ports,
the initial sequence numbers, windows of the receiver and sender, etc.

% \section{TCP queues}
%
% Three queues belong to each TCP connection. The \emph{send queue} holds
% the segments not yet transmitted or not yet acknowledged.
% The \emph{receive queue} holds the segments received by the TCP,
% but not yet passed to the application. (This happens only when the segment
% is received out-of-order.). The \emph{retransmit queue} holds additional
% information about the segments in the send queue.
%
% As mentioned in section \ref{subsec:tcp_packets}, there are three methods
% to represent the application data in the TCP segment. Consequently the above
% queues comes in three flavours. If the payload is transmitted as a message
% object, then \cppclass{TCPMsgBasedRcvQueue} and \cppclass{TCPMsgBasedSendQueue};
% if the payload is a byte array then \cppclass{TCPDataStreamRcvQueue} and
% \cppclass{TCPDataStreamSendQueue}; if only the message lengths are represented
% then \cppclass{TCPVirtualDataRcvQueue} and \cppclass{TCPVirtualDataSendQueue}
% are the classes of receive/send queues. The appropriate queue is created based
% on the value of the \fpar{dataTransferMode} parameter of the Open command, no
% further configuration is needed. The retransmit queue is always an
% instance of \cppclass{TcpSackRexmitQueue}.
%
% The interfaces of the receive/send queues are defined by the
% \cppclass{TcpReceiveQueue} and \cppclass{TcpSendQueue} classes.
%
% % mapping segments into the sequence space
%

\section{TCP algorithms}
\label{sec:tcp_algorithms}

The \cppclass{TcpAlgorithm} object controls
retransmissions, congestion control and ACK sending: delayed acks, slow start,
fast retransmit, etc. They are all extends the \cppclass{TcpAlgorithm} class.
This simplifies the design of \cppclass{TcpConnection} and makes it a lot easier to
implement TCP variations such as Tahoe, NewReno, Vegas or LinuxTCP.

Currently implemented algorithm classes are \cppclass{TcpReno},
\cppclass{TcpTahoe}, \cppclass{TcpNewReno}, \cppclass{TcpNoCongestionControl}
and \cppclass{DumbTcp}. It is also possible to add new TCP variations
by implementing \cppclass{TcpAlgorithm}.

\includegraphics{figures/tcp_algorithms}

The concrete TCP algorithm class to use can be chosen per connection (in OPEN)
or in a module parameter.

\subsection{DumbTcp}

A very-very basic \cppclass{TcpAlgorithm} implementation, with hardcoded
retransmission timeout (2 seconds) and no other sophistication. It can be
used to demonstrate what happened if there was no adaptive
timeout calculation, delayed acks, silly window avoidance,
congestion control, etc. Because this algorithm does not
send duplicate ACKs when receives out-of-order segments,
it does not work well together with other algorithms.

\subsection{TcpBaseAlg}

The \cppclass{TcpBaseAlg} is the base class of the INET implementation
of Tahoe, Reno and New Reno. It implements basic TCP
algorithms for adaptive retransmissions, persistence timers,
delayed ACKs, Nagle's algorithm, Increased Initial Window
-- EXCLUDING congestion control. Congestion control
is implemented in subclasses.

\subsubsection*{Delayed ACK}

When the \fpar{delayedAcksEnabled} parameter is set to \fkeyword{true},
\cppclass{TcpBaseAlg} applies a 200ms delay before sending ACKs.

\subsubsection*{Nagle's algorithm}

When the \fpar{nagleEnabled} parameter is \fkeyword{true}, then
the algorithm does not send small segments if there is outstanding
data. See also \ref{subsec:trans_policies}.

\subsubsection*{Persistence Timer}

The algorithm implements \emph{Persistence Timer} (see \ref{subsec:flow_control}).
When a zero-sized window is received it starts the timer with 5s timeout.
If the timer expires before the window is increased, a 1-byte probe is
sent. Further probes are sent after 5, 6, 12, 24, 48, 60, 60, 60, ...
seconds until the window becomes positive.

\subsubsection*{Initial Congestion Window}

Congestion window is set to 1 SMSS when the connection is established.
If the \fpar{increasedIWEnabled} parameter is true, then the initial
window is increased to 4380 bytes, but at least 2 SMSS and at most 4 SMSS.
The congestion window is not updated afterwards; subclasses can
add congestion control by redefining virtual methods of the
\cppclass{TcpBaseAlg} class.

\subsubsection*{Duplicate ACKs}

The algorithm sends a duplicate ACK when an out-of-order
segment is received or when the incoming segment fills in all
or part of a gap in the sequence space.

\subsubsection*{RTO calculation}

Retransmission timeout ($RTO$) is calculated according to
Jacobson algorithm (with $\alpha=7/8$), and Karn's modification is also applied.
The initial value of the $RTO$ is 3s, its minimum is 1s,
maximum is 240s (2 MSL).

% FIXME according to RFC1222, MIN_REXMIT_TIMEOUT should be a fraction of second
%       to accomodate high speed LANs. In the linux kernel (net/tcp.h)
%       TCP_RTO_MIN is HZ/5 = 200ms. Consider 0ms lower bound.

\subsection{TCPNoCongestion}

TCP with no congestion control (i.e. congestion window kept very large).
Can be used to demonstrate effect of lack of congestion control.

% FIXME 65536 is not 'very large' nowadays, with window scaling
%       the receive window can be as large as 2^30 bytes.
%       Consequently the initial ssthresh is too small for Tahoe/Reno/NewReno,
%       Slow Start is stopped too early first time.

\subsection{TcpTahoe}

The \cppclass{TcpTahoe} algorithm class extends \cppclass{TcpBaseAlg}
with \emph{Slow Start}, \emph{Congestion Avoidance} and
\emph{Fast Retransmit} congestion control algorithms.
This algorithm initiates a \emph{Slow Start} when a packet
loss is detected.

\subsubsection*{Slow Start}

The congestion window is initially set to 1 SMSS or in case of
\fpar{increasedIWEnabled} is \fkeyword{true} to 4380 bytes
(but no less than 2 SMSS and no more than 4 SMSS). The window
is increased on each incoming ACK by 1 SMSS, so it is approximately
doubled in each RTT.

\subsubsection*{Congestion Avoidance}

When the congestion window exceeded $ssthresh$, the window
is increased by $SMSS^2/cwnd$ on each incoming ACK event, so
it is approximately increased by 1 SMSS per RTT.

\subsubsection*{Fast Retransmit}

When the 3rd duplicate ACK received, a packet loss is detected
and the packet is retransmitted immediately. Simultanously
the $ssthresh$ variable is set to half of the $cwnd$ (but at least 2 SMSS)
and $cwnd$ is set to 1 SMSS, so it enters slow start again.

Retransmission timeouts are handled the same way:
$ssthresh$ will be $cwnd/2$, $cwnd$ will be 1 SMSS.

\subsection{TcpReno}

The \cppclass{TcpReno} algorithm extends the behaviour \cppclass{TcpTahoe}
with \emph{Fast Recovery}. This algorithm can also use the information
transmitted in SACK options, which enables a much more accurate
congestion control.

\subsubsection*{Fast Recovery}

When a packet loss is detected by receiveing 3 duplicate ACKs,
$ssthresh$ set to half of the current window as in Tahoe. However
$cwnd$ is set to $ssthresh + 3*SMSS$ so it remains in congestion
avoidance mode. Then it will send one new segment for each incoming
duplicate ACK trying to keep the pipe full of data. This requires
the congestion window to be inflated on each incoming duplicate
ACK; it will be deflated to $ssthresh$ when new data gets
acknowledged.

However a hard packet loss (i.e. RTO events) cause a
slow start by setting $cwnd$ to 1 SMSS.

\subsubsection*{SACK congestion control}

This algorithm can be used with the SACK extension.
Set the \fpar{sackSupport} parameter to \fkeyword{true} to
enable sending and receiving \emph{SACK} options.

\subsection{TcpNewReno}

This class implements the TCP variant known as New Reno.
New Reno recovers more efficiently from multiple packet losses within one RTT
than Reno does.

It does not exit fast-recovery phase until all data which was out-standing
at the time it entered fast-recovery is acknowledged. Thus avoids
reducing the $cwnd$ multiple times.

\section{TCP socket}

%The \cppclass{TcpSocket} C++ class is provided to simplify managing TCP connections
%from applications. \cppclass{TcpSocket} handles the job of assembling and sending
%command messages (OPEN, CLOSE, etc) to \nedtype{Tcp}, and it also simplifies
%the task of dealing with packets and notification messages coming from \nedtype{Tcp}.

\cppclass{TcpSocket} is a convenience class, to make it easier to manage TCP connections
from your application models. You'd have one (or more) \cppclass{TcpSocket} object(s)
in your application simple module class, and call its member functions
(bind(), listen(), connect(), etc.) to open, close or abort a TCP connection.

TCPSocket chooses and remembers the connId for you, assembles and sends command
packets (such as OPEN\_ACTIVE, OPEN\_PASSIVE, CLOSE, ABORT, etc.) to TCP,
and can also help you deal with packets and notification messages arriving
from TCP.

A session which opens a connection from local port 1000 to 10.0.0.2:2000,
sends 16K of data and closes the connection may be as simple as this
(the code can be placed in your \ffunc{handleMessage()} or
\ffunc{activity()}):

\begin{cpp}
TCPSocket socket;
socket.connect(L3AddressResolver().resolve("10.0.0.2"), 2000);

msg = new cMessage("data1");
msg->setByteLength(16*1024);  16K
socket.send(msg);

socket.close();
\end{cpp}

% FIXME missing setOutputGate() call

Dealing with packets and notification messages coming from TCP is somewhat
more cumbersome. Basically you have two choices: you either process those
messages yourself, or let TCPSocket do part of the job. For the latter,
you give TCPSocket a callback object on which it'll invoke the appropriate
member functions: \ffunc{socketEstablished()}, \ffunc{socketDataArrived()},
\ffunc{socketFailure()}, \ffunc{socketPeerClosed()},
etc (these are methods of \cppclass{TCPSocket::CallbackInterface}).,
The callback object can be your simple module class too.

This code skeleton example shows how to set up a TCPSocket to use the module
itself as callback object:

\begin{cpp}
class MyModule : public cSimpleModule, public TCPSocket::CallbackInterface
{
    TCPSocket socket;
    virtual void socketDataArrived(int connId, void *yourPtr,
                                   cPacket *msg, bool urgent);
    virtual void socketFailure(int connId, void *yourPtr, int code);
    ...
};

void MyModule::initialize() {
    socket.setCallbackObject(this,NULL);
}

void MyModule::handleMessage(cMessage *msg) {
    if (socket.belongsToSocket(msg))
        socket.processMessage(msg); dispatch to socketXXXX() methods
    else
        ...
}

void MyModule::socketDataArrived(int, void *, cPacket *msg, bool) {
    ev << "Received TCP data, " << msg->getByteLength() << " bytes\\n";
    delete msg;
}

void MyModule::socketFailure(int, void *, int code) {
    if (code==TCP_I_CONNECTION_RESET)
        ev << "Connection reset!\\n";
    else if (code==TCP_I_CONNECTION_REFUSED)
        ev << "Connection refused!\\n";
    else if (code==TCP_I_TIMEOUT)
        ev << "Connection timed out!\\n";
}
\end{cpp}

If you need to manage a large number of sockets (e.g. in a server
application which handles multiple incoming connections), the
\cppclass{TcpSocketMap} class may be useful. The following code
fragment to handle incoming connections is from the LDP module:

\begin{cpp}
TCPSocket *socket = socketMap.findSocketFor(msg);
if (!socket)
{
    not yet in socketMap, must be new incoming connection: add to socketMap
    socket = new TCPSocket(msg);
    socket->setOutputGate(gate("tcpOut"));
    socket->setCallbackObject(this, NULL);
    socketMap.addSocket(socket);
}
dispatch to socketEstablished(), socketDataArrived(), socketPeerClosed()
or socketFailure()
socket->processMessage(msg);
\end{cpp}

%%% Local Variables:
%%% mode: latex
%%% TeX-master: "usman"
%%% End:


\cleardoublepage

\ifdraft TODO
\include{ch-sctp}
\cleardoublepage
\fi

\ifdraft TODO
\chapter{Internet Routing}
\label{cha:routing}

\section{Overview}

INET Framework has models for several internet routing protocols, including
RIP, OSPF and BGP. 

The easiest way to add routing to a network is to use the \nedtype{Router} 
NED type for routers. \nedtype{Router} contains a conditional instance
for each of the above protocols. These submodules can be enabled by
setting the \ttt{hasRIP}, \ttt{hasOSPF} and/or \ttt{hasBGP} parameters to
\ttt{true}.

Example:

\begin{verbatim}
**.hasRIP = true
\end{verbatim}

There are also NED types called \nedtype{RipRouter}, \nedtype{OspfRouter},
\nedtype{BgpRouter}, which are all \nedtype{Router}s with appropriate
routing protocol enabled.

\section{RIP}
\label{sec:rip}

RIP (Routing Information Protocol) is a distance-vector routing protocol
which employs the hop count as a routing metric. RIP prevents routing loops
by implementing a limit on the number of hops allowed in a path from source 
to destination.

The \nedtype{Rip} module implements distance vector routing as
specified in RFC 2453 (RIPv2) and RFC 2080 (RIPng). Per-interface
configuration can be specified in an XML file.

\nedtype{RipRouter} is a \nedtype{Router} with the RIP protocol enabled.


\section{OSPF}
\label{sec:ospf}

OSPF (Open Shortest Path First) is a routing protocol for IP networks. 
It uses a link state routing (LSR) algorithm and falls into the group 
of interior gateway protocols (IGPs), operating within a single 
autonomous system (AS).

The \nedtype{Ospf} module implements the OSPF Version 2. Areas and routers
can be configured using an XML file.

\nedtype{OspfRouter} is a \nedtype{Router} with the OSPF protocol enabled.


\section{BGP}
\label{sec:bgp}

BGP (Border Gateway Protocol) is a standardized exterior gateway protocol
designed to exchange routing and reachability information among 
autonomous systems (AS) on the Internet.

The \nedtype{Bgp} module implements BGP Version 4. The model implements 
RFC 4271, with some limitations. Autonomous Systems and rules can be
configured in an XML file.

\nedtype{BgpRouter} is a \nedtype{Router} with the BGP protocol enabled.

%%% Local Variables:
%%% mode: latex
%%% TeX-master: "usman"
%%% End:


\cleardoublepage
\fi

\ifdraft TODO
\chapter{Ad Hoc Routing}
\label{cha:adhoc-routing}

\section{Overview}

In ad hoc networks, nodes are not familiar with the topology of 
their networks. Instead, they have to discover it: typically, 
a new node announces its presence and listens for announcements 
broadcast by its neighbors. Each node learns about others nearby 
and how to reach them, and may announce that it too can reach them.
The difficulty of routing may be compounded by the fact that
nodes may be mobile, which results in a changing topology.

Ad hoc routing protocols fall in two broad categories: proactive
and reactive. \textit{Proactive} or \textit{table-driven} protocols 
maintain fresh lists of destinations and their routes by periodically
distributing routing tables throughout the network.  
\textit{Reactive} or \textit{on-demand} protocols find a route on demand
by flooding the network with Route Request packets.

The INET Framework contains the implementation of several ad hoc routing
protocols including AODV, DSDV, DYMO and GPSR. 

The easiest way to add routing to an ad hoc network is to use the
\nedtype{ManetRouter} NED type for nodes. \nedtype{ManetRouter}
contains a submodule named \ttt{routing} whose type is a parameter,
so it can be configured to be an AODV router, a DYMO router, or a
router of any other supported routing protocol. For example, you
can configure \nedtype{ManetRouter} nodes in the network to use
AODV with the following ini file line:

\begin{verbatim}
**.routingProtocolType = "Aodv"
\end{verbatim}

There are also NED types called \nedtype{AodvRouter}, \nedtype{DymoRouter},
\nedtype{DsvRouter}, \nedtype{GpsrRouter}, which are all 
\nedtype{ManetRouter}s with the routing protocol submodule type 
set appropriately.


\section{AODV}
\label{sec:aodv}

AODV (Ad hoc On-Demand Distance Vector Routing) is a routing protocol 
for mobile ad hoc networks and other wireless ad hoc networks.
It offers quick adaptation to dynamic link conditions, low processing and
memory overhead, low network utilization, and determines unicast
routes to destinations within the ad hoc network.

The \nedtype{Aodv} module type implements AODV, based on RFC 3561. 

\nedtype{AodvRouter} is a \nedtype{ManetRouter} with the routing module type
set to \nedtype{Aodv}. 


\section{DSDV}
\label{sec:dsdv}

DSDV (Destination-Sequenced Distance-Vector Routing) is a table-driven 
routing scheme for ad hoc mobile networks based on the Bellman-Ford algorithm.

The \nedtype{Dsdv} module type implements DSDV. It is currently a partial
implementation.

\nedtype{DsdvRouter} is a \nedtype{ManetRouter} with the routing module type
set to \nedtype{Dsdv}. 


\section{DYMO}
\label{sec:dymo}

The DYMO (Dynamic MANET On-demand) routing protocol is successor to the 
AODV routing protocol. DYMO can work as both a pro-active and as a reactive 
routing protocol, i.e. routes can be discovered just when they are needed.

The \nedtype{Dymo} module type implements DYMO, based on the IETF draft
\textit{draft-ietf-manet-dymo-24}.

\nedtype{DymoRouter} is a \nedtype{ManetRouter} with the routing module type
set to \nedtype{Dymo}. 


\section{GPSR}
\label{sec:gpsr}

GPSR (Greedy Perimeter Stateless Routing) is a routing protocol for 
mobile wireless networks that uses the geographic positions of nodes 
to make packet forwarding decisions. 

The \nedtype{Gpsr} module type implements GPSR, based 
on the paper ``GPSR: Greedy Perimeter Stateless Routing for Wireless
Networks'' by Brad Karp and H. T. Kung, 2000. The implementation 
supports both GG and RNG planarization algorithms.

\nedtype{GpsrRouter} is a \nedtype{ManetRouter} with the routing module type
set to \nedtype{Gpsr}. 


%%% Local Variables:
%%% mode: latex
%%% TeX-master: "usman"
%%% End:


\cleardoublepage
\fi

\chapter{Differentiated Services}
\label{cha:diffserv}


\section{Overview}

In the early days of the Internet, only best effort service was defined.
The Internet delivers individually each packet, and delivery time is not
guaranteed, moreover packets may even be dropped due to congestion at
the routers of the network. It was assumed that transport protocols,
and applications can overcome these deficiencies. This worked until
FTP and email was the main applications of the Internet, but the newer
applications such as Internet telephony and video conferencing cannot
tolerate delay jitter and loss of data.

% TypeOfService field

The first attempt to add QoS capabilities to the IP routing was
Integrated Services. Integrated services provide resource assurance
through resource reservation for individual application flows.
An application flow is identified by the source and destination
addresses and ports and the protocol id. Before data packets are
sent the necessary resources must be allocated along the path
from the source to the destination. At the hops from the source
to the destination each router must examine the packets, and decide
if it belongs to a reserved application flow. This could cause a
memory and processing demand in the routers.
Other drawback is that
the reservation must be periodically refreshed, so there is an overhead
during the data transmission too.

Differentiated Services is a more scalable approach to offer a better than
best-effort service. Differentiated Services do not require resource reservation
setup. Instead of making per-flow reservations, Differentiated
Services divides the traffic into a small number of \emph{forwarding classes}.
The forwarding class is directly encoded into the packet header. After packets are
marked with their forwarding classes at the edge of the network, the interior nodes
of the network can use this information to differentiate the treatment of packets.
The forwarding classes may indicate drop priority and resource priority. For example,
when a link is congested, the network will drop packets with the highest drop priority
first.

In the Differentiated Service architecture, the network is partitioned into
DiffServ domains. Within each domain the resources of the domain are allocated
to forwarding classes, taking into account the available resources and the
traffic flows. There are \emph{service level agggreements} (SLA) between the users
and service providers, and between the domains that describe the mapping of
packets to forwarding classes and the allowed traffic profile for each class.
The routers at the edge of the network are responsible for marking the packets
and protect the domain from misbehaving traffic sources. Nonconforming traffic
may be dropped, delayed, or marked with a different forwarding class.


\subsection{Implemented Standards}

The implementation follows these RFCs below:

\begin{itemize}
  \item RFC 2474: Definition of the Differentiated Services Field (DS Field) in the IPv4 and IPv6 Headers
  \item RFC 2475: An Architecture for Differentiated Services
  \item RFC 2597: Assured Forwarding PHB Group
  \item RFC 2697: A Single Rate Three Color Marker
  \item RFC 2698: A Two Rate Three Color Marker
  \item RFC 3246: An Expedited Forwarding PHB (Per-Hop Behavior)
  \item RFC 3290: An Informal Management Model for Diffserv Routers
\end{itemize}

\section{Architecture of NICs}

Network Interface Card (NIC) modules, such as \nedtype{PppInterface} and
\nedtype{EthernetInterface}, may contain traffic conditioners in
their input and output data path. Traffic conditioners have one input
and one output gate as defined in the \nedtype{ITrafficConditioner}
interface. They can transform the incoming traffic by dropping or
delaying packets. They can also set the DSCP field of the packet,
or mark them other way, for differentiated handling in the queues.

The NICs may also contain an external queue component. If the \fpar{queueType}
parameter is set, it must contain a module type implementing the \nedtype{IOutputQueue}
module interface. If it is not specified, then \nedtype{Ppp} and \nedtype{EtherMac}
use an internal drop tail queue to buffer the packets until the line is busy.

\subsection{Traffic Conditioners}

Traffic conditioners have one input
and one output gate as defined in the \nedtype{ITrafficConditioner}
interface. They can transform the incoming traffic by dropping or
delaying packets. They can also set the DSCP field of the packet,
or mark them other way, for differentiated handling in the queues.

Traffic conditioners perform the following actions:
\begin{itemize}
 \item classify the incoming packets
 \item meter the traffic in each class
 \item marks/drops packets depending on the result of metering
 \item shape the traffic by delaying packets to conform to the
       desired traffic profile
\end{itemize}

INET provides classifier, meter, and marker modules, that can be
composed to build a traffic conditioner as a compound module.

\subsection{Output Queues}

The queue component also has one input and one output gate. These components
must implement a passive queue behaviour: they only deliver a packet,
when the module connected to its output explicitly asks them.
In terms of C++ it means, that the simple module owning the \fgate{out} gate,
or which is connected to the \fgate{out} gate of the compound module,
must implement the \cppclass{IPassiveQueue} interface. The next module
asks a packet by calling the \ffunc{requestPacket()} method of this interface.


\section{Simple modules}

This section describes the primitive elements from which traffic
conditioners and output queues can be built. The next sections
shows some examples, how these queues, schedulers, droppers,
classifiers, meters, markers can be combined.

The type of the components are:
\begin{itemize}
  \item \ttt{queue}: container of packets, accessed as FIFO
  \item \ttt{dropper}: attached to one or more queue, it can
    limit the queue length below some threshold
    by selectively dropping packets
  \item \ttt{scheduler}: decide which packet is transmitted first,
     when more packets are available on their inputs
  \item \ttt{classifier}: classify the received packets
     according to their content (e.g. source/destination,
     address and port, protocol, dscp field of IP datagrams)
     and forward them to the corresponding output gate.
  \item \ttt{meter}: classify the received packets
      according to the temporal characteristic of their
      traffic stream
  \item \ttt{marker}: marks packets by setting their fields
      to control their further processing
\end{itemize}

\subsection{Queues}

When packets arrive at higher rate, than the interface can trasmit,
they are getting queued.


Queue elements store packets until they can be transmitted.
They have one input and one output gate.
Queues may have one or more thresholds associated with them.

 Received packets
are enqueued and stored until the module connected to their
output asks a packet by calling the \ffunc{requestPacket()}
method.

They should be able to notify the module connected to its output
about the arrival of new packets.

\subsubsection{FIFO Queue}

The \nedtype{FifoQueue} module implements a passive
FIFO queue with unlimited buffer space. It can be combined
with algorithmic droppers and schedulers to form an
IOutputQueue compound module.

The C++ class implements the \cppclass{IQueueAccess} and
\cppclass{IPassiveQueue} interfaces.

\subsubsection{DropTailQueue}

The other primitive queue module is \nedtype{DropTailQueue}.
Its capacity can be specified by the \fpar{frameCapacity}
parameter. When the number of stored packet reached the capacity
of the queue, further packets are dropped.
Because this module contains a built-in dropping strategy, it
cannot be combined with algorithmic droppers as \nedtype{FifoQueue}
can be. However its output can be connected to schedulers.

This module implements the \nedtype{IOutputQueue} interface,
so it can be used as the queue component of interface card per se.

\subsection{Droppers}

Algorithmic droppers selectively drop received packets based on some condition.
The condition can be either deterministic (e.g. to bound the queue length),
or probabilistic (e.g. RED queues).

Other kind of droppers are absolute droppers; they drop each received
packet. They can be used to discard excess traffic, i.e. packets whose
arrival rate exceeds the allowed maximum. In INET the \nedtype{Sink}
module can be used as an absolute dropper.

The algorithmic droppers in INET are \nedtype{ThresholdDropper} and
\nedtype{RedDropper}. These modules has multiple input and multiple
output gates. Packets that arrive on gate \fgate{in[i]} are forwarded
to gate \fgate{out[i]} (unless they are dropped). However the queues
attached to the output gates are viewed as a whole, i.e. the queue
length parameter of the dropping algorithm is the sum of the individual
queue lengths. This way we can emulate shared buffers of the queues.
Note, that it is also possible to connect each output to the same
queue module.

\subsubsection{Threshold Dropper}

The \nedtype{ThresholdDropper} module selectively drops packets,
based on the available buffer space of the queues attached to its output.
The buffer space can be specified as the count of packets, or as the size
in bytes.

The module sums the buffer lengths of its outputs
and if enqueuing a packet would exceed the configured
capacities, then the packet will be dropped instead.

By attaching a \nedtype{ThresholdDropper} to the input of a FIFO
queue, you can compose a drop tail queue. Shared buffer
space can be modeled by attaching more FIFO queues
to the output.

\subsubsection*{RED Dropper}

The \nedtype{RedDropper} module implements Random Early Detection
(\cite{Floyd93randomearly}).

It has $n$ input and $n$ output gates (specified by the
\fpar{numGates} parameter). Packets that arrive at the $i^{th}$ input
gate are forwarded to the $i^{th}$ output gate, or dropped.
The output gates must be connected to simple modules implementing
the \nedtype{IQueueAccess} C++ interface (e.g. \nedtype{FifoQueue}).

The module sums the used buffer space of the queues attached
to the output gates. If it is below a minimum threshold,
the packet won't be dropped, if above a maximum threshold,
it will be dropped, if it is between the minimum and
maximum threshold, it will be dropped by a given probability.
This probability determined by a linear function which is
0 at the minth and maxp at maxth.

\begin{center}
\setlength{\unitlength}{1cm}
\begin{picture}(7,4)(-1,-1)
\put(-0.5,0){\vector(1,0){6.5}}
\put(0,-0.5){\vector(0,1){3.5}}
\put(5.8,-0.3){$qlen$}
\put(-0.5,3){$p$}
\put(1,0){\line(3,1){3}}
\put(4,1){\line(0,1){1}}
\put(4,2){\line(1,0){1.5}}
\put(-0.5,1.9){$1$}
%\put(-0.2,2){\line(1,0){0.2}}
\multiput(0,2)(0.4,0){10}{\line(1,0){0.2}}
%\put(-0.2,1){\line(1,0){0.2}}
\multiput(0,1)(0.4,0){10}{\line(1,0){0.2}}
\put(-1,0.9){$p_{max}$}
\multiput(4,0)(0,0.4){3}{\line(0,1){0.2}}
\put(0.9,-0.3){$th_{min}$}
\put(3.9,-0.3){$th_{max}$}
\end{picture}
\end{center}

The queue length can be smoothed by specifying the \fpar{wq}
parameter. The average queue length used in the tests
are computed by the formula:

 $$avg = (1-wq)*avg + wq*qlen$$

The \fpar{minth}, \fpar{maxth}, and \fpar{maxp} parameters
can be specified separately for each input gate, so this module
can be used to implement different packet drop priorities.

\subsection{Schedulers}

Scheduler modules decide which queue can send a packet, when the
interface is ready to transmit one. They have several input gates
and one output gate.

Modules that are connected to the inputs of a scheduler must
implement the \cppclass{IPassiveQueue} C++ interface.
Schedulers also implement \cppclass{IPassiveQueue}, so
they can be cascaded to other schedulers, and can be used
as the output module of \nedtype{IOutputQueue}s.

There are several possible scheduling discipline (first come/first served,
priority, weighted fair, weighted round-robin, deadline-based,
rate-based). INET contains implementation
of priority and weighted round-robin schedulers.

\subsubsection{Priority Scheduler}

The \nedtype{PriorityScheduler} module implements a strict priority
scheduler. Packets that arrived on \fgate{in[0]} has the highest priority,
then packets arrived on \fgate{in[1]}, and so on. If more packets
available when one is requested, then the one with highest priority
is chosen. Packets with lower priority are transmitted only when
there are no packets on the inputs with higher priorities.

\nedtype{PriorityScheduler} must be used with care, because a
large volume of higher packets can starve lower priority packets.
Therefore it is necessary to limit the rate of higher priority
packets to a fraction of the output datarate.

\nedtype{PriorityScheduler} can be used to implement
the \ttt{EF} PHB.

\subsubsection*{Weighted Round Robin Scheduler}

The \nedtype{WrrScheduler} module implements a weighted
round-robin scheduler. The scheduler visits the input gates
in turn and selects the number of packets for transmission
based on their weight.

For example if the module has three input gates, and the weights
are 3, 2, and 1, then packets are transmitted in this order:
\begin{verbatim}
A, A, A, B, B, C, A, A, A, B, B, C, ...
\end{verbatim}
where A packets arrived on \fgate{in[0]}, B packets on \fgate{in[1]},
and C packets on \fgate{in[2]}. If there are no packets in the current
one when a packet is requested, then the next one is chosen that has
enough tokens.

If the size of the packets are equal, then \nedtype{WrrScheduler}
divides the available bandwith according to the weights. In each
case, it allocates the bandwith fairly. Each flow receives a guaranteed
minimum bandwith, which is ensured even if other flows exceed
their share (flow isolation). It is also efficiently uses the
channel, because if some traffic is smaller than its share of
bandwidth, then the rest is allocated to the other flows.

\nedtype{WrrScheduler} can be used to implement the \ttt{AFxy} PHBs.

\subsection{Classifiers}

Classifier modules have one input and many output gates.
They examine the received packets, and forward them to the
appropriate output gate based on the content of some portion
of the packet header. You can read more about classifiers
in RFC 2475 2.3.1 and RFC 3290 4.

The \nedtype{inet.networklayer.diffserv} package contains two
classifiers: \nedtype{MultiFieldClassifier} to classify
the packets at the edge routers of the DiffServ domain, and
\nedtype{BehaviorAggregateClassifier} to classify the packets
at the core routers.


\subsubsection*{Multi-field Classifier}

The \nedtype{MultiFieldClassifier} module can be used to identify
micro-flows in the incoming traffic. The flow is identified
by the source and destination addresses, the protocol id,
and the source and destination ports of the IP packet.

The classifier can be configured by specifying a list of filters.
Each filter can specify a source/destination address mask, protocol,
source/destination port range, and bits of TypeOfService/TrafficClass
field to be matched. They also specify the index of the output gate
matching packet should be forwarded to. The first matching filter
determines the output gate, if there are no matching filters,
then \fgate{defaultOut} is chosen.

The configuration of the module is given as an XML document.
The document element must contain a list of \ttt{<filter>} elements.
The filter element has a mandatory \ttt{@gate} attribute that gives
the index of the gate for packets matching the filter. Other attributes
are optional and specify the condition of matching:
\begin{compactitem}
  \item \ttt{@srcAddress}, \ttt{@srcPrefixLength}: to match the source
    address of the IP
  \item \ttt{@destAddress}, \ttt{@destPrefixLength}:
  \item \ttt{@protocol}: matches the protocol field of the IP packet.
    Its value can be a name (e.g. ``udp'', ``tcp''),
    or the numeric code of the protocol.
  \item \ttt{@tos},{@tosMask}: matches bits of the TypeOfService/TrafficClass
    field of the IP packet.
  \item \ttt{@srcPort}: matches the source port of the TCP or UDP packet.
  \item \ttt{@srcPortMin}, \ttt{@srcPortMax}: matches a range of source ports.
  \item \ttt{@destPort}: matches the destination port of the TCP or UDP packet.
  \item \ttt{@destPortMin}, \ttt{@destPortMax}: matches a range of
     destination ports.
\end{compactitem}

The following example configuration specifies
\begin{compactitem}
  \item to transmit packets received from the 192.168.1.x subnet on gate 0,
  \item to transmit packets addressed to port 5060 on gate 1,
  \item to transmit packets having CS7 in their DSCP field on gate 2,
  \item to transmit other packets on \fgate{defaultGate}.
\end{compactitem}

\begin{verbatim}
<filters>
  <filter srcAddress="192.168.1.0" srcPrefixLength="24" gate="0"/>
  <filter protocol="udp" destPort="5060" gate="1"/>
  <filter tos="0b00111000" tosMask="0x3f" gate="2"/>
</filters>
\end{verbatim}

\subsubsection*{Behavior Aggregate Classifier}

The \nedtype{BehaviorAggregateClassifier} module can be used to read
the DSCP field from the IP datagram, and direct the packet to
the corresponding output gate. The DSCP value is the lower
six bits of the TypeOfService/TrafficClass field. Core routers
usually use this classifier to guide the packet to the appropriate
queue.

DSCP values are enumerated in the \fpar{dscps} parameter.
The first value is for gate \fgate{out[0]}, the second for
\fgate{out[1]}, so on. If the received packet has a DSCP
value not enumerated in the \fpar{dscps} parameter, it will
be forwarded to the \nedtype{defaultOut} gate.

\subsection{Meters}

Meters classify the packets based on the temporal characteristics
of their arrival. The arrival rate of packets is compared to an
allowed traffic profile, and packets are decided to be green
(in-profile) or red (out-of-profile). Some meters apply more than two
conformance level, e.g. in three color meters the partially conforming
packets are classified as yellow.

The allowed traffic profile is usually specified by a token bucket.
In this model, a bucket is filled in with tokens with a specified rate,
until it reaches its maximum capacity. When a packet arrives, the
bucket is examined. If it contains at least as many tokens as the
length of the packet, then that tokens are removed, and the packet
marked as conforming to the traffic profile. If the bucket contains
less tokens than needed, it left unchanged, but the packet marked
as non-conforming.

Meters has two modes: color-blind and color-aware.
In color-blind mode, the color assigned by a previous meter does not
affect the classification of the packet in subsequent meters.
In color-aware mode, the color of the packet can not be changed to a less
conforming color: if a packet is classified as non-conforming by a meter,
it also handled as non-conforming in later meters in the data path.

\begin{important}
Meters take into account the length of the IP packet only, L2 headers are omitted
from the length calculation. If they receive a packet which is not
an IP datagram and does not encapsulate an IP datagram, an error occurs.
\end{important}

\subsubsection*{TokenBucketMeter}

The \nedtype{TokenBucketMeter} module implements a simple token bucket meter.
The module has two output, one for green packets, and one for red packets.
When a packet arrives, the gained tokens are added to the bucket, and
the number of tokens equal to the size of the packet are subtracted.

Packets are classified according to two parameters,
Committed Information Rate ($cir$), Committed Burst Size ($cbs$),
to be either green, or red.

Green traffic is guaranteed to be under $cir*(t_1-t_0)+8*cbs$ in
every $[t_0,t_1]$ interval.

\subsubsection*{SingleRateThreeColorMeter}

The \nedtype{SingleRateThreeColorMeter} module implements a
Single Rate Three Color Meter (RFC 2697).
The module has three output for green, yellow, and red packets.

Packets are classified according to three parameters,
Committed Information Rate ($cir$), Committed Burst Size ($cbs$),
and Excess Burst Size ($ebs$), to be either green, yellow or red.
The green traffic is guaranteed to be under $cir*(t_1-t_0)+8*cbs$,
while the green+yellow traffic to be under $cir*(t_1-t_0)+8*(cbs+ebs)$
in every $[t_0,t_1]$ interval.


\subsubsection*{TwoRateThreeColorMeter}

The \nedtype{TwoRateThreeColorMeter} module implements a
Two Rate Three Color Meter (RFC 2698). The module has three output
gates for the green, yellow, and red packets.

It classifies the packets based on two rates, Peak Information Rate ($pir$)
and Committed Information Rate ($cir$), and their associated burst sizes
($pbs$ and $cbs$) to be either green, yellow or red. The green traffic
is under $pir*(t_1-t_0)+8*pbs$ and $cir*(t_1-t_0)+8*cbs$, the yellow traffic
is under $pir*(t_1-t_0)+8*pbs$ in every $[t_0,t_1]$ interval.

\subsection{Markers}

DSCP markers sets the codepoint of the crossing packets.
The codepoint determines the further processing of the packet
in the router or in the core of the DiffServ domain.

The \nedtype{DscpMarker} module sets the DSCP field
(lower six bit of TypeOfService/TrafficClass) of IP datagrams
to the value specified by the \fpar{dscps} parameter.
The \fpar{dscps} parameter is a space separated list
of codepoints. You can specify a different value
for each input gate; packets arrived at the $i^{th}$
input gate are marked with the $i^{th}$ value.
If there are fewer values, than gates, then the last
one is used for extra gates.

The DSCP values are enumerated in the \ffilename{DSCP.msg} file.
You can use both names and integer values in the \fpar{dscps}
parameter.

For example the following lines are equivalent:
\begin{inifile}
**.dscps = "EF 0x0a 0b00001000"
**.dscps = "46 AF11 8"
\end{inifile}

\section{Compound modules}

\subsection{AFxyQueue}

The \nedtype{AFxyQueue} module is an example queue, that implements
one class of the Assured Forwarding PHB group (RFC 2597).

Packets with the same AFx class, but different drop priorities
arrive at the \fgate{afx1In}, \fgate{afx2In}, and \fgate{afx3In} gates.
The received packets are stored in the same queue. Before the packet
is enqueued, a RED dropping algorithm may decide to selectively
drop them, based on the average length of the queue and the RED parameters
of the drop priority of the packet.

The afxyMinth, afxyMaxth, and afxyMaxp parameters must have values that
ensure that packets with lower drop priorities are dropped with lower
or equal probability than packets with higher drop priorities.

\subsection{DiffservQeueue}

The \nedtype{DiffservQueue} is an example queue, that can be used in
interfaces of DS core and edge nodes to support
the AFxy (RFC 2597) and EF (RFC 3246) PHBs.

\begin{center}
\includegraphics[scale=0.7]{figures/DiffservQueue.png}
\end{center}

The incoming packets are first classified according to
their DSCP field. DSCPs other than AFxy and EF are handled
as BE (best effort).

EF packets are stored in a dedicated queue, and served first
when a packet is requested. Because they can preempt the other
queues, the rate of the EF packets should be limited to a fraction
of the bandwith of the link. This is achieved by metering the EF
traffic with a token bucket meter and dropping packets that
does not conform to the traffic profile.

There are other queues for AFx classes and BE. The AFx queues
use RED to implement 3 different drop priorities within the class.
BE packets are stored in a drop tail queue.
Packets from AFxy and BE queues are sheduled by a WRR scheduler,
which ensures that the remaining bandwith is allocated among the classes
according to the specified weights.

%%% Local Variables:
%%% mode: latex
%%% TeX-master: "usman"
%%% End:



\cleardoublepage

\include{ch-mpls}
\cleardoublepage

\ifdraft TODO
\chapter{Applications}
\label{cha:apps}


\section{Overview}

This chapter describes application models and traffic generators.

\section{TCP applications}

This sections describes the applications using the TCP protocol.
Each application must implement the \nedtype{ITCPApp} module interface
to ease configuring the \nedtype{StandardHost} module.

The applications described here are all contained by the
\nedtype{inet.applications.tcpapp} package. These applications use
\msgtype{GenericAppMsg} objects to represent the data sent between the client
and server. The client message contains the expected reply length, the
processing delay, and a flag indicating that the connection should be closed
after sending the reply. This way intelligence (behaviour specific to the
modelled application, e.g. HTTP, SMB, database protocol) needs only to be
present in the client, and the server model can be kept simple and dumb.


\subsection{TcpBasicClientApp}

Client for a generic request-response style protocol over TCP.
May be used as a rough model of HTTP or FTP users.

The model communicates with the server in sessions. During a session,
the client opens a single TCP connection to the server, sends several
requests (always waiting for the complete reply to arrive before
sending a new request), and closes the connection.

The server app should be \nedtype{TcpGenericServerApp}; the model sends
\msgtype{GenericAppMsg} messages.

Example settings:

FTP:

\begin{inifile}
numRequestsPerSession = exponential(3)
requestLength = truncnormal(20,5)
replyLength = exponential(1000000)
\end{inifile}

HTTP:

\begin{inifile}
numRequestsPerSession = 1 # HTTP 1.0
numRequestsPerSession = exponential(5)  # HTTP 1.1, with keepalive
requestLength = truncnormal(350,20)
replyLength = exponential(2000)
\end{inifile}

Note that since most web pages contain images and may contain frames,
applets etc, possibly from various servers, and browsers usually download
these items in parallel to the main HTML document, this module cannot
serve as a realistic web client.

Also, with HTTP 1.0 it is the server that closes the connection after
sending the response, while in this model it is the client.

\subsection{TcpSinkApp}

Accepts any number of incoming TCP connections, and discards whatever
arrives on them.

The module parameter \fpar{dataTransferMode} should be set the transfer mode in TCP layer.
Its possible values (``bytecount'', ``object'', ``bytestream'') are described in ...

\subsection{TcpGenericServerApp}

Generic server application for modelling TCP-based request-reply style
protocols or applications.

Requires message object preserving sendQueue/receiveQueue classes
to be used with \nedtype{Tcp} (that is, TCPMsgBasedSendQueue and TCPMsgBasedRcvQueue;
TCPVirtualBytesSendQueue/RcvQueue are not good).

The module accepts any number of incoming TCP connections, and expects
to receive messages of class \msgtype{GenericAppMsg} on them. A message should
contain how large the reply should be (number of bytes). \nedtype{TcpGenericServerApp}
will just change the length of the received message accordingly, and send
back the same message object. The reply can be delayed by a constant time
(replyDelay parameter).

\subsection{TcpEchoApp}

The \nedtype{TcpEchoApp} application accepts any number of incoming TCP
connections, and sends back the messages that arrive on them, The lengths of the
messages are multiplied by \fpar{echoFactor} before sending them back (echoFactor=1
will result in sending back the same message unmodified.) The reply can also be
delayed by a constant time (\fpar{echoDelay} parameter).

When \nedtype{TcpEchoApp} receives data packets from TCP (and such, when they can be
echoed) depends on the dataTransferMode setting.
With "bytecount" and "bytestream", TCP passes up data to us
as soon as a segment arrives, so it can be echoed immediately.
With "object" mode, our local TCP reproduces the same
messages that the sender app passed down to its TCP -- so if the sender
app sent a single 100 MB message, it will be echoed only when all
100 megabytes have arrived.

\subsection{TcpSessionApp}

Single-connection TCP application: it opens a connection, sends
the given number of bytes, and closes. Sending may be one-off,
or may be controlled by a "script" which is a series of
(time, number of bytes) pairs. May act either as client or as server,
and works with TCPVirtualBytesSendQueue/RcvQueue as sendQueue/receiveQueue
setting for ~TCP.
Compatible with both IPv4 (~IPv4) and ~IPv6.

\subsubsection*{Opening the connection}

Regarding the type of opening the connection, the application may
be either a client or a server. When active=false, the application
will listen on the given local localPort, and wait for an incoming connection.
When active=true, the application will bind to given local localAddress:localPort,
and connect to the connectAddress:connectPort. To use an ephemeral port
as local port, set the localPort parameter to -1.

Even when in server mode (active=false), the application will only
serve one incoming connection. Further connect attempts will be
refused by TCP (it will send RST) for lack of LISTENing connections.

The time of opening the connection is in the tOpen parameter.

\subsubsection*{Sending data}

Regardless of the type of OPEN, the application can be made to send
data. One way of specifying sending is via the tSend, sendBytes
parameters, the other way is sendScript. With the former, sendBytes
bytes will be sent at tSend. With sendScript, the format is
"<time> <numBytes>;<time> <numBytes>;..."

\subsubsection*{Closing the connection}

The application will issue a TCP CLOSE at time tClose. If tClose=-1, no
CLOSE will be issued.



\subsection{TelnetApp}

Models Telnet sessions with a specific user behaviour.
The server app should be \nedtype{TcpGenericServerApp}.

In this model the client repeats the following activity
between \fpar{startTime} and \fpar{stopTime}:

\begin{enumerate}
\item opens a telnet connection
\item sends \fpar{numCommands} commands. The command is \fpar{commandLength} bytes
      long. The command is transmitted as entered by the user character by character,
      there is \fpar{keyPressDelay} time between the characters. The server echoes
      each character. When the last character of the command is sent (new line),
      the server responds with a \fpar{commandOutputLength} bytes long message.
      The user waits \fpar{thinkTime} interval between the commands.
\item closes the connection and waits \fpar{idleInterval} seconds
\item if the connection is broken it is noticed after \fpar{reconnectInterval}
      and the connection is reopened
\end{enumerate}

Each parameter in the above description is ``volatile'', so you can
use distributions to emulate random behaviour.

Additional parameters:
addresses,ports
dataTransferMode

\begin{note}
This module emulates a very specific user behaviour, and as such,
it should be viewed as an example rather than a generic Telnet model.
If you want to model realistic Telnet traffic, you are encouraged
to gather statistics from packet traces on a real network, and
write your model accordingly.
\end{note}

\subsection{TcpServerHostApp}

This module hosts TCP-based server applications. It dynamically creates
and launches a new "thread" object for each incoming connection.

Server threads should be subclassed from the \cppclass{TcpServerThreadBase}
C++ class, registered in the C++ code using the Register\_Class() macro,
and the class name should be specified in the serverThreadClass
parameter of \nedtype{TcpServerHostApp}. The thread object will receive events
via a callback interface (methods like established(), dataArrived(),
peerClosed(), timerExpired()), and can send packets via TCPSocket's send()
method.

Example server thread class: \cppclass{TcpGenericServerThread}.

\begin{important}
Before you try to use this module, make sure you actually need it!
In most cases, \nedtype{TcpGenericServerApp} and \msgtype{GenericAppMsg} will be completely
enough, and they are a lot easier to handle. You'll want to subclass your
client from \cppclass{TCPGenericCliAppBase} then; check \nedtype{TelnetApp} and
\nedtype{TcpBasicClientApp} for examples.
\end{important}


\section{UDP applications}

All UDP applications should be derived from the \nedtype{IUDPApp} module interface,
so that the application of \nedtype{StandardHost} could be configured without changing its NED file.

The following applications are implemented in INET:
\begin{itemize}
\item \nedtype{UdpBasicApp} sends UDP packets to a given IP address at a given interval
\item \nedtype{UdpBasicBurst} sends UDP packets to the given IP address(es) in bursts, or acts as a packet sink.
\item \nedtype{UdpEchoApp} similar to \nedtype{UdpBasicApp}, but it sends back the packet after reception
\item \nedtype{UdpSink} consumes and prints packets received from the \nedtype{Udp} module
\item \nedtype{UdpVideoStreamClient},\nedtype{UdpVideoStreamServer} simulates UDP streaming
\end{itemize}

The next sections describe these applications in details.

\subsection{UdpBasicApp}

The \nedtype{UdpBasicApp} sends UDP packets to a the IP addresses given in the
\fpar{destAddresses} parameter. The application sends a message to one of the
targets in each \fpar{sendInterval} interval. The interval between message and
the message length can be given as a random variable. Before the packet is
sent, it is emitted in the \fsignal{sentPk} signal.

The application simply prints the received UDP datagrams. The \fsignal{rcvdPk}
signal can be used to detect the received packets.

The number of sent and received messages are saved as scalars at the end of the
simulation.

% could be a simple packet generator without the ability to receive packets?

\subsection{UdpSink}

This module binds an UDP socket to a given local port, and prints the
source and destination and the length of each received packet.

% TODO does not accept broadcast messages

\subsection{UdpEchoApp}

Similar to \nedtype{UdpBasicApp}, but it sends back the packet after reception.
It accepts only packets with \msgtype{UDPEchoAppMsg} type, i.e. packets that
are generated by another \nedtype{UdpEchoApp}.

When an echo response received, it emits an \fsignal{roundTripTime} signal.

\subsection{UdpVideoStreamClient}

This module is a video streaming client. It send one ``video streaming request'' to
the server at time \fpar{startTime} and receives stream from \nedtype{UdpVideoStreamServer}.

The received packets are emitted by the \fsignal{rcvdPk} signal.

\subsection{UdpVideoStreamServer}

This is the video stream server to be used with \nedtype{UdpVideoStreamClient}.

The server will wait for incoming "video streaming requests".
When a request arrives, it draws a random video stream size
using the \fpar{videoSize} parameter, and starts streaming to the client.
During streaming, it will send UDP packets of size \fpar{packetLen} at every
\fpar{sendInterval}, until \fpar{videoSize} is reached. The parameters \fpar{packetLen}
and \fpar{sendInterval} can be set to constant values to create CBR traffic,
or to random values (e.g. sendInterval=uniform(1e-6, 1.01e-6)) to
accomodate jitter.

The server can serve several clients, and several streams per client.

% FIXME why streamVector? VideoStreamData could be deleted immediately after last byte sent
% TODO this is video-on-demand, support multicast/broadcast video streaming too

\subsection{UdpBasicBurst}

Sends UDP packets to the given IP address(es) in bursts, or acts as a
packet sink. Compatible with both IPv4 and IPv6.

\subsubsection*{Addressing}

The \fpar{destAddresses} parameter can contain zero, one or more destination
addresses, separated by spaces. If there is no destination address given,
the module will act as packet sink. If there are more than one addresses,
one of them is randomly chosen, either for the whole simulation run,
or for each burst, or for each packet, depending on the value of the
\fpar{chooseDestAddrMode} parameter. The \fpar{destAddrRNG} parameter controls which
(local) RNG is used for randomized address selection.
The own addresses will be ignored.

An address may be given in the dotted decimal notation, or with the module
name. (The \cppclass{L3AddressResolver} class is used to resolve the address.)
You can use the "Broadcast" string as address for sending broadcast messages.

INET also defines several NED functions that can be useful:
\begin{itemize}
\item[-] moduleListByPath("pattern",...): \\
         Returns a space-separated list of the modulenames.
         All modules whole getFullPath() matches one of the pattern parameters will get included.
         The patterns may contain wilcards in the same syntax as in ini files.
         See cTopology::extractByModulePath() function
         example: destaddresses = moduleListByPath("**.host[*]", "**.fixhost[*]")
\item[-] moduleListByNedType("fully.qualified.ned.type",...): \\
         Returns a space-separated list of the modulenames with the given NED type(s).
         All modules whose getNedTypeName() is listed in the given parameters will get included.
         The NED type name is fully qualified.
         See cTopology::extractByNedTypeName() function
         example: destaddresses = moduleListByNedType("inet.nodes.inet.StandardHost")
\end{itemize}

The peer can be UDPSink or another UDPBasicBurst.

\subsubsection*{Bursts}

The first burst starts at \fpar{startTime}. Bursts start by immediately sending
a packet; subsequent packets are sent at \fpar{sendInterval} intervals. The
sendInterval parameter can be a random value, e.g. exponential(10ms).
A constant interval with jitter can be specified as 1s+uniform(-0.01s,0.01s)
or uniform(0.99s,1.01s). The length of the burst is controlled by the
\fpar{burstDuration} parameter. (Note that if \fpar{sendInterval} is greater than
\fpar{burstDuration}, the burst will consist of one packet only.) The time between
burst is the \fpar{sleepDuration} parameter; this can be zero (zero is not
allowed for \fpar{sendInterval}.) The zero \fpar{burstDuration} is interpreted as infinity.

\subsubsection*{Packets}

Packet length is controlled by the \fpar{messageLength} parameter.

The module adds two parameters to packets before sending:
\begin{itemize}
\item[-] sourceID: source module ID
\item[-] msgId: incremented by 1 after send any packet.
\end{itemize}
When received packet has this parameters, the module checks the order of received packets.

\subsubsection*{Operation as sink}

When \fpar{destAddresses} parameter is empty, the module receives packets and makes statistics only.

\subsubsection*{Statistics}

Statistics are collected on outgoing packets:
\begin{itemize}
\item[-] sentPk: packet object
\end{itemize}

Statistics are collected on incoming packets:
\begin{itemize}
\item[-] outOfOrderPk: statistics of out of order packets.
       The packet is out of order, when has msgId and sourceId parameters and module
       received bigger msgId from same sourceID.
\item[-] dropPk: statistics of dropped packets.
       The packet is dropped when not out-of-order packet and delay time is larger than
       delayLimit parameter. The delayLimit=0 is infinity.
\item[-] rcvdPk: statistics of not dropped, not out-of-order packets.
\item[-] endToEndDelay: end to end delay statistics of not dropped, not out-of-order packets.
\end{itemize}


\section{SCTP applications}

TODO


\section{IPv4/IPv6 traffic generators}

The applications described in this section use the services of the network
layer only, they do not need transport layer protocols.
They can be used with both IPv4 and IPv6.

\nedtype{IIPvXTraffixGenerator} (prototype) sends IP or IPv6 datagrams to the
given address at the given \fpar{sendInterval}.
The \fpar{sendInterval} parameter can be a constant or a random value (e.g. exponential(1)).
If the \fpar{destAddresses} parameter contains more than one address, one
of them is randomly for each packet. An address may be given in the
dotted decimal notation (or, for IPv6, in the usual notation with colons),
or with the module name. (The \cppclass{L3AddressResolver} class is used to resolve
the address.) To disable the model, set destAddresses to "".

The \nedtype{IpvxTrafGen} sends messages with length \fpar{packetLength}.
The sent packet is emitted in the \fsignal{sentPk} signal.
The length of the sent packets can be recorded as scalars and vectors.

The \nedtype{IpvxTrafSink} can be used as a receiver of the packets
generated by the traffic generator. This module emits the packet
in the \fsignal{rcvdPacket} signal and drops it. The \ttt{rcvdPkBytes}
and \ttt{endToEndDelay} statistics are generated from this signal.

The \nedtype{IpvxTrafGen} can also be the peer of the traffic generators;
it handles the received packets exactly like \nedtype{IpvxTrafSink}.

\section{The PingApp application}

The \nedtype{PingApp} application
generates ping requests and calculates the packet loss and round trip
parameters of the replies.

Start/stop time, sendInterval etc. can be specified via parameters. An address
may be given in the dotted decimal notation (or, for IPv6, in the usual
notation with colons), or with the module name.
(The \cppclass{L3AddressResolver} class is used to resolve the address.)
To disable send, specify empty destAddr.

Every ping request is sent out with a sequence number, and replies are
expected to arrive in the same order. Whenever there's a jump in the
in the received ping responses' sequence number (e.g. 1, 2, 3, 5), then
the missing pings (number 4 in this example) is counted as lost.
Then if it still arrives later (that is, a reply with a sequence number
smaller than the largest one received so far) it will be counted as
out-of-sequence arrival, and at the same time the number of losses is
decremented. (It is assumed that the packet arrived was counted earlier as a loss,
which is true if there are no duplicate packets.)

Uses \msgtype{PingPayload} as payload for the ICMP(v6) Echo Request/Reply packets.

\subsection*{Parameters}

\begin{itemize}
  \item \fpar{destAddr}: destination address
  \item \fpar{srcAddr}: source address (useful with multi-homing)
  \item \fpar{packetSize}: of ping payload, in bytes (default is 56)
  \item \fpar{sendInterval}: time to wait between pings (can be random, default is 1s)
  \item \fpar{hopLimit}: TTL or hopLimit for IP packets (default is 32)
  \item \fpar{count}: stop after \fpar{count} ping request, 0 means continuously
  \item \fpar{startTime}: send first ping request at \fpar{startTime}
  \item \fpar{stopTime}: time of finish sending, 0 means forever
  \item \fpar{printPing}: dump on stdout (default is \fkeyword{false})
\end{itemize}

\subsection*{Signals and Statistics}

\begin{itemize}
  \item \fsignal{rtt} value of the round trip time
  \item \fsignal{numLost} number of lost packets
  \item \fsignal{outOfOrderArrivals} number of packets arrived out-of-order
  \item \fsignal{pingTxSeq} sequence number of the sent ping request
  \item \fsignal{pingRxSeq} sequence number of the received ping response
\end{itemize}

% FIXME seqNo should be part of ICMPMessage


\section{Ethernet applications}

The \nedtype{inet.applications.ethernet} package contains modules
for a simple client-server application. The \nedtype{EtherAppClient} is a simple
traffic generator that peridically sends \msgtype{EtherAppReq} messages
whose length can be configured. destAddress, startTime,waitType, reqLength, respLength

The server component of the model (\nedtype{EtherAppServer}) responds with a
\msgtype{EtherAppResp} message of the requested length. If the response does
not fit into one ethernet frame, the client receives the data in multiple
chunks.

% FIXME reqLength>1500 causes an error in the LLC module
% FIXME numFrames field of EtherAppRes is not used
% FIXME server always sends 1497 byte chunks, it should depend on the framing (1497 is for LLC)
% FIXME if registerSAP is false (default), the and EtherLLC used, then the client won't receive messages (auto config?)
% FIXME Ieee802Nic -> EthernetInterface in the NED comment

Both applications have a \fpar{registerSAP} boolean parameter.
This parameter should be set to \ttt{true} if the application is connected
to the \nedtype{EtherLlc} module which requires registration of the SAP
before sending frames.

Both applications collects the following statistics: sentPkBytes, rcvdPkBytes,
endToEndDelay.

The client and server application works with any model that accepts
Ieee802Ctrl control info on the packets (e.g. the 802.11 model).
The applications should be connected directly to the \nedtype{EtherLlc}
or an EthernetInterface NIC module.

The model also contains a host component that groups the applications
and the LLC and MAC components together (\nedtype{EtherHost}). This node does
not contain higher layer protocols, it generates Ethernet traffic directly.
By default it is configured to use half duplex MAC (CSMA/CD).



%%% Local Variables:
%%% mode: latex
%%% TeX-master: "usman"
%%% End:


\cleardoublepage
\fi

\ifdraft TODO
\chapter{Visualization}
\label{cha:visualization}

\section{Overview}

The INET Framework is able to visualize a wide range of events and conditions
in the network: packet drops, data link connectivity, wireless signal path loss,
transport connections, routing table routes, and many more. Visualization is
implemented as a collection of configurable INET modules that can be added
to simulations at will.

\section{Visualizing Network Communication}

\subsection{Visualizing Packet Drops}

Several network problems manifest themselves as excessive packet drops, for
example poor connectivity, congestion, or misconfiguration. Visualizing packet
drops helps identifying such problems in simulations, thereby reducing time
spent on debugging and analysis. Poor connectivity in a wireless network can
cause senders to drop unacknowledged packets after the retry limit is exceeded.
Congestion can cause queues to overflow in a bottleneck router, again resulting
in packet drops.

Packet drops can be visualized by including a \nedtype{PacketDropVisualizer}
module in the simulation. The \nedtype{PacketDropVisualizer} module indicates
packet drops by displaying an animation effect at the node where the packet drop
occurs. In the animation, a packet icon gets thrown out from the node icon, and
fades away.

The visualization of packet drops can be enabled with the visualizer's
\fpar{displayPacketDrops} parameter. By default, packet drops at all nodes are
visualized. This selection can be narrowed with the \fpar{nodeFilter},
\fpar{interfaceFilter} and \fpar{packetFilter} parameters.

One can click on the packet drop icon to display information about the packet
drop in the inspector panel.

Packets are dropped for the following reasons:

\begin{itemize}
  \item queue overflow
  \item retry limit exceeded
  \item unroutable packet
  \item network address resolution failed
  \item interface down
\end{itemize}


\subsection{Visualizing Transport Path Activity}

With INET simulations, it is often useful to be able to visualize network
traffic. INET provides several visualizers for this task, operating at various
levels of the network stack.

Transport path activity can be visualized by including a
\nedtype{TransportRouteVisualizer} module in the simulation. \nedtype{TransportRouteVisualizer}
that can provide graphical feedback about transport traffic, i.e. traffic that passes
through the transport layers of two endpoints. Adding an \nedtype{IntegratedVisualizer} is
also an option, because it also contains a \nedtype{TransportRouteVisualizer}. Transport
path activity visualization is disabled by default, it can be enabled by setting
the visualizer's \fpar{displayRoutes} parameter to true.

\nedtype{TransportRouteVisualizer} observes packets that pass through the transport layer,
i.e. carry data from/to higher layers.

The activity between two nodes is represented visually by a polyline arrow which
points from the source node to the destination node. \nedtype{TransportRouteVisualizer}
follows packets throughout their path so that the polyline goes through all
nodes which are the part of the path of packets. The arrow appears after the
first packet has been received, then gradually fades out unless it is reinforced
by further packets. Color, fading time and other graphical properties can be
changed with parameters of the visualizer.

By default, all packets and nodes are considered for the visualization. This
selection can be narrowed with the visualizer's packetFilter and nodeFilter
parameters.

\subsection{Visualizing Network Path Activity}

With INET simulations, it is often useful to be able to visualize network
traffic. INET offers several visualizers for this task, operating at various
levels of the network stack. In this showcase, we examine \nedtype{NetworkRouteVisualizer}
that can provide graphical feedback about network layer level traffic.

Network path activity can be visualized by including a \nedtype{NetworkRouteVisualizer}
module in the simulation. Adding an \nedtype{IntegratedVisualizer} module is also an
option, because it also contains a \nedtype{NetworkRouteVisualizer} module. Network path
activity visualization is disabled by default, it can be enabled by setting the
visualizer's \fpar{displayRoutes} parameter to true.

\nedtype{NetworkRouteVisualizer} currently observes packets that pass through the network
layer (i.e. carry data from/to higher layers), but not those that are internal
to the operation of the network layer protocol. That is, packets such as ARP,
although potentially useful, will not trigger the visualization.

The activity between two nodes is represented visually by a polyline arrow which
points from the source node to the destination node. \nedtype{NetworkRouteVisualizer}
follows packet throughout its path so the polyline goes through all nodes that
are part of the packet's path. The arrow appears after the first packet has been
received, then gradually fades out unless it is reinforced by further packets.
Color, fading time and other graphical properties can be changed with parameters
of the visualizer.

By default, all packets and nodes are considered for the visualization. This
selection can be narrowed with the visualizer's packetFilter and nodeFilter
parameters.


\subsection{Visualizing Data Link Activity}

With INET simulations, it is often useful to be able to visualize network
traffic. INET offers several visualizers for this task, operating at various
levels of the network stack. In this showcase, we examine \nedtype{DataLinkVisualizer}
that can provide graphical feedback about data link level traffic.

Data link activity can be visualized by including a \nedtype{DataLinkVisualizer} module in
the simulation. Adding an \nedtype{IntegratedVisualizer} module is also an option, because
it also contains a \nedtype{DataLinkVisualizer} module. Data link visualization is
disabled by default, it can be enabled by setting the visualizer's displayLinks
parameter to true.

\nedtype{DataLinkVisualizer} currently observes packets that pass through the data link
layer (i.e. carry data from/to higher layers), but not those that are internal
to the operation of the data link layer protocol. That is, frames such as ACK,
RTS/CTS, Beacon or Authentication/Association frames of IEEE 802.11, although
potentially useful, will not trigger the visualization. Visualizing such frames
may be implemented in future INET revisions.

The activity between two nodes is represented visually by an arrow that points
from the sender node to the receiver node. The arrow appears after the first
packet has been received, then gradually fades out unless it is refreshed by
further packets. The style, color, fading time and other graphical properties
can be changed with parameters of the visualizer.

By default, all packets, interfaces and nodes are considered for the
visualization. This selection can be narrowed to certain packets and/or nodes
with the visualizer's \fpar{packetFilter}, \fpar{interfaceFilter}, and
\fpar{nodeFilter} parameters.


\subsection{Visualizing Physical Link Activity}

With INET simulations, it is often useful to be able to visualize network
traffic. For this task, there are several visualizers in INET, operating at
various levels of the network stack. In this showcase, we demonstrate working of
\nedtype{PhysicalLinkVisualizer} that can provide graphical feedback about physical layer
traffic.

Physical link activity can be visualized by including a \nedtype{PhysicalLinkVisualizer}
module in the simulation. Adding an \nedtype{IntegratedVisualizer} module is also an
option, because it also contains a \nedtype{PhysicalLinkVisualizer} module. Physical link
activity visualization is disabled by default, it can be enabled by setting the
visualizer's \fpar{displayLinks} parameter to true.

\nedtype{PhysicalLinkVisualizer} observes frames that pass through the physical layer,
i.e. are received correctly.

The activity between two nodes is represented visually by a dotted arrow which
points from the sender node to the receiver node. The arrow appears after the
first frame has been received, then gradually fades out unless it is refreshed
by further frames. Color, fading time and other graphical properties can be
changed with parameters of the visualizer.

By default, all packets, interfaces and nodes are considered for the
visualization. This selection can be narrowed with the visualizer's
\fpar{packetFilter}, \fpar{interfaceFilter}, and \fpar{nodeFilter} parameters.


\subsection{Visualizing Routing Tables}

In a complex network topology, it is difficult to see how a packet would be
routed because the relevant data is scattered among network nodes and hidden in
their routing tables. INET contains support for visualization of routing tables,
and can display routing information graphically in a concise way. Using
visualization, it is often possible to understand routing in a simulation
without looking into individual routing tables. The visualization currently
supports IPv4.

The \nedtype{RoutingTableVisualizer} module (included in the network as part of
\nedtype{IntegratedVisualizer}) is responsible for visualizing routing table entries.

The visualizer basically annotates network links with labeled arrows that
connect source nodes to next hop nodes. The module visualizes those routing
table entries that participate in the routing of a given set of destination
addresses, by default the addresses of all interfaces of all nodes in the
network. That is, it selects the best (longest prefix) matching routes for all
destination addresses from each routing table, and shows them as arrows that
point to the next hop. Note that one arrow might need to represent several
routing entries, for example when distinct prefixes are routed towards the same
next hop.

Routing table entries are represented visually by solid arrows. An arrow going
from a source node represents a routing table entry in the source node's routing
table. The endpoint node of the arrow is the next hop in the visualized routing
table entry. By default, the routing entry is displayed on the arrows in
following format:

\begin{verbatim}
destination/mask -> gateway (interface)
\end{verbatim}

The format can be changed by setting the visualizer's \fpar{labelFormat} parameter.

Filtering is also possible. The \fpar{nodeFilter} parameter controls which nodes'
routing tables should be visualized (by default, all nodes), and the
\fpar{destinationFilter} parameter selects the set of destination nodes to consider
(again, by default all nodes.)

The visualizer reacts to changes. For example, when a routing protocol changes a
routing entry, or an IP address gets assigned to an interface by DHCP, the
visualizer automatically updates the visualizations according to the specified
filters. This is very useful e.g. for the simulation of mobile ad-hoc networks.

\subsection{Displaying IP Addresses and Other Interface Information}

In the simulation of complex networks, it is often useful to be able to display
node IP addresses, interface names, etc. above the node icons or on the links.
For example, when automatic address assignment is used in a hierarchical network
(e.g. using \nedtype{Ipv4NetworkConfigurator}), visual inspection can help to
verify that the result matches the expectations. While it is true that addresses and other
interface data can also be accessed in the GUI by diving into the interface
tables of each node, that is tedious, and unsuitable for getting an overview.

The \nedtype{InterfaceTableVisualizer} module (included in the network as part of
\nedtype{IntegratedVisualizer}) displays data about network nodes' interfaces.
(Interfaces are contained in interface tables, hence the name.) By default, the
visualization is turned off. When it is enabled using the
\fpar{displayInterfaceTables} parameter, the default is that interface names, IP
addresses and netmask length are displayed, above the nodes (for wireless
interfaces) and on the links (for wired interfaces). By clicking on an interface
label, details are displayed in the inspector panel.

The visualizer has several configuration parameters. The \fpar{format} parameter
specifies what information is displayed about interfaces. It takes a format
string, which can contain the following directives:

\begin{itemize}
  \item \%N: interface name
  \item \%4: IPv4 address
  \item \%6: IPv6 address
  \item \%n: network address. This is either the IPv4 or the IPv6 address
  \item \%l: netmask length
  \item \%M: MAC address
  \item \%\textbackslash: conditional newline for wired interfaces. The '\textbackslash'
  needs to be escaped with another '\textbackslash', i.e. '\%\textbackslash\textbackslash'
  \item \%i and \%s: the info() and str() functions for the interfaceEntry class, respectively
\end{itemize}

The default format string is
\texttt{"\%N \%\textbackslash\textbackslash\%n/\%l"}, i.e. interface name, IP address and
netmask length.

The set of visualized interfaces can be selected with the configurator's
\fpar{nodeFilter} and \fpar{interfaceFilter} parameters. By default, all
interfaces of all nodes are visualized, except for loopback addresses (the default for the
\fpar{interfaceFilter} parameter is \texttt{"not lo\textbackslash*"}.)

It is possible to display the labels for wired interfaces above the node icons,
instead of on the links. This can be done by setting the
\fpar{displayWiredInterfacesAtConnections} parameter to false.

There are also several parameters for styling, such as color and font selection.


\subsection{Visualizing IEEE 802.11 Network Membership}

When simulating wifi networks that overlap in space, it is difficult to see
which node is a member of which network. The membership may even change over
time. It would be useful to be able to display e.g. the SSID above node icons.

IEEE 802.11 network membership can be visualized by including a
\nedtype{Ieee80211Visualizer} module in the simulation. Adding an \nedtype{IntegratedVisualizer} is
also an option, because it also contains a \nedtype{Ieee80211Visualizer}. Displaying
network membership is disabled by default, it can be enabled by setting the
visualizer's \fpar{displayAssociations} parameter to true.

The \nedtype{Ieee80211Visualizer} displays an icon and the SSID above network nodes which
are part of a wifi network. The icons are color-coded according to the SSID. The
icon, colors, and other visual properties can be configured via parameters of
the visualizer.

The visualizer's \fpar{nodeFilter} parameter selects which nodes' memberships are
visualized. The \fpar{interfaceFilter} parameter selects which interfaces are
considered in the visualization. By default, all interfaces of all nodes are
considered.


\subsection{Visualizing Transport Connections}

In a large network with a complex topology, there might be many transport layer
applications and many nodes communicating. In such a case, it might be difficult
to see which nodes communicate with which, or if there is any communication at
all. Transport connection visualization makes it easy to get information about
the active transport connections in the network at a glance. Visualization makes
it easy to identify connections by their two endpoints, and to tell different
connections apart. It also gives a quick overview about the number of
connections in individual nodes and the whole network.

The \nedtype{TransportConnectionVisualizer} module (also part of \nedtype{IntegratedVisualizer})
displays color-coded icons above the two endpoints of an active, established
transport layer level connection. The icons will appear when the connection is
established, and disappear when it is closed. Naturally, there can be multiple
connections open at a node, thus there can be multiple icons. Icons have the
same color at both ends of the connection. In addition to colors, letter codes
(A, B, AA, …) may also be displayed to help in identifying connections. Note
that this visualizer does not display the paths the packets take. If you are
interested in that, take a look at \nedtype{TransportRouteVisualizer}, covered in the
Visualizing Transport Path Activity showcase.

The visualization is turned off by default, it can be turned on by setting the
\fpar{displayTransportConnections} parameter of the visualizer to true.

It is possible to filter the connections being visualized. By default, all
connections are included. Filtering by hosts and port numbers can be achieved by
setting the \fpar{sourcePortFilter}, \fpar{destinationPortFilter},
\fpar{sourceNodeFilter} and \fpar{destinationNodeFilter} parameters.

The icon, colors and other visual properties can be configured by setting the
visualizer's parameters.


\section{Visualizing The Infrastructure}

\subsection{Visualizing the Physical Environment}

The physical environment has a profound effect on the communication of wireless
devices. For example, physical objects like walls inside buildings constraint
mobility. They also obstruct radio signals often resulting in packet loss. It's
difficult to make sense of the simulation without actually seeing where physical
objects are.

The visualization of physical objects present in the physical environment is
essential.

The \nedtype{PhysicalEnvironmentVisualizer} (also part of \nedtype{IntegratedVisualizer}) is
responsible for displaying the physical objects. The objects themselves are
provided by the PhysicalEnvironment module; their geometry, physical and visual
properties are defined in the XML configuration of the PhysicalEnvironment
module.

The two-dimensional projection of physical objects is determined by the
\nedtype{SceneCanvasVisualizer} module. (This is because the projection is also needed by
other visualizers, for example \nedtype{MobilityVisualizer}.) The default view is top view
(z axis), but you can also configure side view (x and y axes), or isometric or
ortographic projection.

The visualizer also supports OpenGL-based 3D rendering using the OpenSceneGraph
(OSG) library. If the OMNeT++ installation has been compiled with OSG
support, you can switch to 3D view using the Qtenv toolbar.

\subsection{Visualizing Node Mobility}

In INET simulations, the movement of mobile nodes is often as important as the
communication among them. However, as mobile nodes roam, it is often difficult
to visually follow their movement. INET provides a visualizer that not only
makes visually tracking mobile nodes easier, but also indicates other properties
like speed and direction.

Node mobility of nodes can be visualized by \nedtype{MobilityVisualizer} module
(included in the network as part of \nedtype{IntegratedVisualizer}). By default,
mobility visualization is enabled, it can be disabled by setting
\fpar{displayMovements} parameter to false.

By default, all mobilities are considered for the visualization. This selection
can be narrowed with the visualizer's \fpar{moduleFilter} parameter.

The visualizer has several important features:

\begin{itemize}
  \item Movement Trail: It displays a line along the recent path of movements.
        The trail gradually fades out as time passes. Color, trail length and
        other graphical properties can be changed with parameters of the
        visualizer.
  \item Velocity Vector: Velocity is represented visually by an arrow. Its
        starting point is the node, and its direction coincides with the
        movement's direction. The arrow's length is proportional to the node's
       speed.
  \item Orientation Arc: Node orientation is represented by an arc whose size
       is specified by the \fpar{orientationArcSize} parameter. This value is the
       relative size of the arc compared to a full circle. The arc's default
       value is 0.25, i.e. a quarter of a circle.
\end{itemize}

These features are disabled by default; they can be enabled by setting the
visualizer's \fpar{displayMovementTrails}, \fpar{displayVelocities} and
\fpar{displayOrientations} parameters to true.


\section{Generic}

The following pages describe generic features that are common to many visualizers.

TODO Styling and Appearance (from the showcase)


%%% Local Variables:
%%% mode: latex
%%% TeX-master: "usman"
%%% End:


\cleardoublepage
\fi

\include{ch-history}
\cleardoublepage

\bibliographystyle{alpha}
\bibliography{inet-users-guide}


%% no need for the following since 'tocbibind' package
%% \phantomsection
%% \addcontentsline{toc}{chapter}{\indexname}
\printindex

\end{document}

%%% Local Variables:
%%% mode: latex
%%% TeX-master: t
%%% End:
